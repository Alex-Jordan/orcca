% arara: pdflatex
% arara: pdflatex
% arara: makechapters: {makeChapGlossaries: yes, items: [module1,module2,module3,module4,module5,module6,module7,module8,module9,module10] }
% !arara: makechapters: {items: [module4,module5,module6], makeChapGlossaries: no}
% !arara: makechapters: {items: [module1,module2,module3,module4,module5,module6,module7,module8,module9,module10]}
% !arara: indent: { overwrite: on, trace: yes, localSettings: on}
\documentclass[12pt,a4paper,anypage]{report}
% ================================================================================================
%
%		MTH 65 notes
%       Last edited October 11th 2013
%       Version 1.2
%           C. M. Hughes
%
% ================================================================================================
\usepackage[	textheight=25cm,
	left=3.65cm,right=1.65cm,
	top=2.5cm,
footskip=1.5cm]{geometry}                   % page set up
\usepackage{parskip}
\usepackage[sc,hang,font=small]{caption}           % figures/tables captions
\usepackage{subcaption}
\usepackage{minitoc}                                % mini-table of contents
\usepackage{amssymb}                                % mathematical content
\usepackage{amsmath}                                % mathematical content
\usepackage[standard,thmmarks,amsmath]{ntheorem}	% needed for theorems, examples- MUST load AFTER amsmath
\usepackage{enumitem}
\usepackage{placeins} 					            % enables \FloatBarrier, useful for positioning figures/tables more precisely.
\usepackage{multicol}                               % multi columns
\usepackage{booktabs}
\usepackage{siunitx}
\usepackage{datetime}
\usepackage[framemethod=tikz]{mdframed}
\usepackage{pgfplots}                               % drawing graphs
\usetikzlibrary{positioning}
\usetikzlibrary{shapes.misc}
%\usepackage{refcheck}                              % useful for checking references
\usepackage{longtable}                              % tables that run over a page
\usepackage{fancyhdr}                               % headers and footers
\usepackage{needspace}                              % needed to keep examples looking good (with \hrule above and below)
\usepackage[explicit]{titlesec}                     % customize section headings
%\usepackage{kpfonts}
\usepackage[charter]{mathdesign}                     % changes font
\usepackage[expansion=false,kerning=true]{microtype} % better kerning
\usepackage{adjustbox}
\usepackage{varioref}
\usepackage{hyperref}                               % to allow hyper refs in the final pdf document
\usepackage{cleveref}

% glossaries settings
\usepackage[nonumberlist,section=chapter]{glossaries}
%\usepackage[section=chapter]{glossaries}
% make the links black
\renewcommand*{\glstextformat}[1]{\textcolor{black}{#1}} 
% basic vocab
\newglossaryentry{equation}{name=equation,description={an equation usually involves at least one variable, and 
has two sides; typically we will try and solve an equation for one of the unknown variables}}
\newglossaryentry{expression}{name=expression,description={an expression usually involves at least one variable--
	in contrast to an equation, an expression does not have two sides, and will often need to be simplified (not solved)
in some way}}
\newglossaryentry{variable}{name=variable,description={represents an unknown quantity in an expression or 
equation; the most-commonly used variable is $x$}}
\newglossaryentry{solution}{name=solution,description={is a value of the variable that satisfies the given equation. In 
the context of systems of equations, the solution is an ordered pair}}
\newglossaryentry{solve}{name=solve,description={the process we use to find the 
	value (or values) of a variable that satisfy an equation; we use solve when working 
	with an \emph{equation} and \emph{simplify} when working with an \emph{expression}}}
\newglossaryentry{simplify}{name=simplify,description={the process we use to make an expression
more manageable or convenient-- combining like terms is an example of simplifying}}
\newglossaryentry{linear}{name=linear,description={means that the highest exponent in the expression or equation is $1$. A few examples: 
	$x=5$, $y=5x-3$, $4x$. Examples of non-linear expressions: $x^2$, $y^3+xy$}}
\newglossaryentry{quadratic}{name=quadratic,description={quadratic expressions have degree $2$. For example, $4x^2+x-3$}}
% graphical-based entries
\newglossaryentry{system}{name=system,description={a system of linear equations is (for our purposes) a pair of linear equations. It can be solved
in a variety of different ways, including: graphing, substitution, and elimination. Some systems do not have a solution}}
\newglossaryentry{slope}{name=slope,description={describes the way a line slants; slope can be positive, 
negative, zero, or undefined. $m$ is most commonly used to represent slope}}
\newglossaryentry{intercept}{name=intercept,description={an intercept is where a line (or curve) crosses an axis; we will 
discuss $x$-intercepts and $y$-intercepts. We might also call them horizontal and vertical intercepts respectively}}
\newglossaryentry{point}{name=point,description={a point is a way of visualizing an ordered pair $(x,y)$ on a graph; we need 
points to plot graphs}}
\newglossaryentry{vertex}{name=vertex,description={the highest (if $a>0$) or lowest (if $a<0$) point on the parabola
representing a quadratic function}}
% polynomial stuff
\newglossaryentry{degree}{name=degree,description={the degree of a polynomial is the highest power present; when there is only
one variable, simply find the highest power; if there is more than one variable, find the highest sum of powers}}
\newglossaryentry{coefficient}{name=coefficient,description={the value attached to a variable; for example in $2x$ the coefficient
of $x$ is $2$. In the expression $-5y^3$ the coefficient is $-5$}}
\newglossaryentry{monomial}{name=monomial,description={a \emph{single} algebraic term, for example $3t^2$, $-\frac{3}{4}x^2$, $r$}}
\newglossaryentry{binomial}{name=binomial,description={\emph{two} algebraic terms added or subtracted together, for example $5x^2-4y^3$}}
\newglossaryentry{trinomial}{name=trinomial,description={\emph{three} algebraic terms added or subtracted together, for example $y^2+y-1$}}
\newglossaryentry{polynomial}{name=polynomial,description={an arrangement of one or more algebraic terms, for example $2x^5-x+1$}}
\newglossaryentry{FOIL}{name=FOIL,description={\textbf{F}irst \textbf{O}utside \textbf{I}nside \textbf{L}ast is a useful
acronym to help us expand binomial multiplication; for example $(x+1)(x-2)$}}
\newglossaryentry{factor}{name=factor,description={factoring is the process of 
	writing an expression as the \emph{product} of irreducible factors; examples of factored expressions are $2(x+1)$ and $4(y+3)(y-5)$}}
\newglossaryentry{prime}{name=prime,description={numbers are prime when they have no factors other than themselves and the 
	number $1$; for example: $2$, $3$, $5$, $7$, $11$, $13$ are all prime numbers. Expressions are prime when they can not 
be factored; $x^2+1$ can not be factored in the real numbers}}
\makeglossaries

% arrow style
\tikzset{>=stealth}

% cycle list- truly awesome; see section 4.6.7, pg 129 of pgfplots
\pgfplotscreateplotcyclelist{pccstylelist}{%
	color=red,mark=none,<->\\%
	color=violet,mark=none,<->\\%
}

% axis settings
\pgfplotsset{
	every axis/.append style={width=10cm,
		axis x line=middle,
		axis y line=middle,
		line width=1pt,
		xlabel={$x$},
		ylabel={$y$},
		axis line style={<->},
		scale only axis,       % otherwise width won't be as intended: http://tex.stackexchange.com/questions/36297/pgfplots-how-can-i-scale-to-text-width
		cycle list name=pccstylelist,
	},
	% framed
	framed/.style={axis background/.style ={draw=black}},
	% grid style
	grid style={gray!50},
	% line style
	cmhplot/.style={color=red,thick,mark=none,<->},
	soldot/.style={only marks,mark=*},
	holdot/.style={color=black,fill=white,only marks,mark=*},
}

% enumerate settings
\setlist{itemsep=0.05em,topsep=-0.1em}
\setlist[enumerate]{label=(\alph*)}
\setlist[enumerate,2]{label=(\roman*)}

% newlist: steps
\newlist{steps}{enumerate}{3}
\setlist[steps]{label=(\arabic*)}%

% we might want to change the steps as we proceed through 
% the different topics- for example, in polynomial functions
% we may want P1, P2, ..., and in rational functions we 
% might want R1, R2, ..., etc
% 
% The following command should ease the process
\newcommand{\reformatstepslist}[1]{\setlist[steps]{label*=(${#1}_\arabic*$)}}

% hyperref settings- it seemed to work better here than
% as options to the \usepackage call above
\hypersetup{colorlinks=true,
	linkcolor=blue
}

% rename chapter as Module
\renewcommand{\chaptername}{Module}

% custom chapter
\titleformat{\chapter}[display]
{\normalfont\Large\filcenter\bf}
{\titlerule[1pt]%
	\vspace{1pt}%
	\titlerule
	\vspace{1pc}%
	\LARGE\MakeUppercase{\chaptertitlename} \thechapter}
{1pc}
{\titlerule
	\vspace{1pc}%
	\Huge}

% \chapter*
\titleformat{name=\chapter,numberless}
{\normalfont\huge\bfseries}{}{-20pt}{\Huge#1}

% custom section
\titleformat{\section}
{\normalfont\Large\bfseries}
{\llap{\thesection\hskip 9pt}#1}
{0pt}
{}

% custom subsection
\titleformat{\subsection}
{\normalfont\large\bfseries}
{\llap{\thesubsection\hskip 9pt}#1}
{0pt}
{}

% From the titlesec package
% \titlespacing{command}{left spacing}{before spacing}{after spacing}[right]
% spacing: how to read {12pt plus 4pt minus 2pt}
%           12pt is what we would like the spacing to be
%           plus 4pt means that TeX can stretch it by at most 4pt
%           minus 2pt means that TeX can shrink it by at most 2pt
%       This is one example of the concept of, 'glue', in TeX
\titlespacing{\chapter}{0pt}{*4}{-0.1cm}
\titlespacing\section{0pt}{12pt plus 4pt minus 2pt}{-5pt plus 2pt minus 2pt}
\titlespacing\subsection{0pt}{12pt plus 4pt minus 2pt}{-6pt plus 2pt minus 2pt}
\titlespacing\subsubsection{0pt}{12pt plus 4pt minus 2pt}{-6pt plus 2pt minus 2pt}

% some useful commands for commenting and displaying mathematics respectively
\newcommand{\com}[1]{\iffalse{#1}\fi}
\newcommand{\dd}{\displaystyle}

\makeatletter
\newtheoremstyle{margincmh}%
{\item[\theorem@headerfont \llap{##1 ##2}]}%
{\item[\theorem@headerfont \llap{##1 ##2} -- ##3\theorem@separator\hskip\labelsep]}%
%{\item[\theorem@headerfont \llap{##1 ##2} -- ##3\IfEndWith{##3}{!}{}{\theorem@separator}\hskip\labelsep]}%
\newtheoremstyle{margincmhsoln}%
{\item[\theorem@headerfont \llap{##1}]}%
{\item[\theorem@headerfont \llap{##1} (##3): ]}%
\makeatother


% example
\theoremstyle{margincmh}
%\theorembodyfont{\itshape}}
\theorembodyfont{}
\theoremsymbol{}
%\theoremprework{\bigskip\needspace{\baselineskip}\hrule\medskip}
\theoremprework{\medskip}
\theorempostwork{}
\theoremseparator{:}
\newtheorem{myexample}{example}[chapter]

% solution
\theoremstyle{margincmhsoln}
\theorembodyfont{}
%\theorempostwork{\medskip\hrule\needspace{\baselineskip}}
\theoremprework{\medskip}
\theoremsymbol{\rlap{$\blacksquare$}}
\theoremseparator{}
\newtheorem{myProof}{Solution}

%% myDefinition
\pgfdeclarehorizontalshading{exersicebackground}{100bp}
{color(0bp)=(blue!40);
	color(100bp)=(black!5)}
%\newmdenv[
%	outerlinewidth=1pt,
%	innerlinewidth=0pt,
%	roundcorner=10pt,
%	tikzsetting={shading=exersicebackground},
%	innertopmargin=.5cm,
%	%skipabove={\dimexpr0.5\baselineskip+\topskip\relax},
%	%skipabove=0pt,
%	%needspace=3\baselineskip,
%]{myDefinition}
\newmdenv[outerlinewidth=3,
        innerlinewidth=2,linecolor=gray,
        backgroundcolor=blue!20,%
    	tikzsetting={shading=exersicebackground},
        innerlinecolor=yellow!50,outerlinecolor=red!50,innertopmargin=0pt,%
        splittopskip=\topskip,skipbelow=0pt,%
        ]{myDefinition}


% Margin Paragraph
\reversemarginpar
\setlength{\marginparwidth}{1.0in}
\let\oldmarginpar\marginpar
\renewcommand\marginpar[1]{\-\oldmarginpar[\raggedleft\footnotesize #1]%
	{\raggedright\footnotesize #1}}

% Define 'Drill and skill' command
\newcommand{\drillandskill}
{%
	\marginpar{\raisebox{-1cm}{\em \color{red} \small Drill and skill!}}%
}

% solution for drill and skill problems
\newcommand{\solution}[1]{{\color{blue}{#1}}}

% Define text reference command
% 	- it puts a comment in the margin
%	- it writes to a file with the references you have used
\newwrite\sectionRefwrite
\openout\sectionRefwrite sectionRefs.log\relax

\newcommand{\textref}[2]%
{\marginpar{\em\color{red}\small See \S #1, pg #2}%
	\write\sectionRefwrite{Module \thechapter, p \thepage: Section #1, page #2^^J}%
}

% Define fix command
% 	- it puts a comment in the margin
%	- it writes to a file with a list of things that need fixing
\newwrite\sortwrite
\openout\sortwrite fixThis.log\relax

\newcommand{\fixthis}[1]
{%
	\marginpar{\huge \color{red} \framebox{FIX}}%
	\write\sortwrite{p\thepage : #1^^J}%
}

% tight center- a center environment with no topsep
\newenvironment{tightcenter}{%
	\setlength\topsep{0pt}
	\setlength\parskip{0pt}
	\begin{center}
		}{%
	\end{center}
}

% standard environments
\crefname{table}{Table}{Tables}
\Crefname{table}{Table}{Tables}
\crefname{figure}{Figure}{Figures}
\Crefname{figure}{Figure}{Figures}
\crefname{section}{Section}{Sections}
\Crefname{section}{Section}{Sections}
\crefname{equation}{Equation}{Equations}
\Crefname{equation}{Equation}{Equations}
% custom environments
\crefname{myexample}{example}{examples}
\Crefname{myexample}{Example}{Examples}

%% this bit is useful because it helps make modules
%\ifdefined\myfile
%\includeonly{\myfile}
%\else
%% include everything !
%\fi
%\includeonly{module2}

% ================================================================================================
%
% 				BEGIN DOCUMENT
%
% ================================================================================================
\begin{document}

% minitoc commands
\dominitoc
\faketableofcontents

\pagestyle{fancy}

\lhead{\tiny\rightmark}
\rhead{\tiny\leftmark}
\lfoot{\tiny v1.2: \today, \currenttime}

%+*** mainfile.tex
% !arara: pdflatex: { files: [ mainfile.tex ] }
% arara: makechapters: { files:[mainfile], items: [module1], makeChapGlossaries: yes}
% !arara: indent: { overwrite: on, trace: yes, localSettings: on}
\chapter{Systems of equations}
\minitoc
\section{Review of Graphing lines}\label{sec:reviewgraphlines}
Our starting points in Math 65 requires us to have a good foundation of graphing of lines. It may
have been a while since you have studied them, and this section is to remind you about them.

There are 3 main pieces of terminology that we will need for this section:
\begin{enumerate}
	\item \gls{slope} describes the way a line slants ($m$)
	\item the $x$-\gls{intercept} is where the line crosses the $x$ - axis.
	\item the $y$-intercept is where the line crosses the $y$ - axis; $(0, b)$
\end{enumerate}
There are various ways to write the \gls{equation} of a line
\begin{itemize}[leftmargin=6cm,font=\bfseries]
	\item[Standard form	]	$Ax+By=C$, where $A$, $B$, and $C$ are integers	
	\item[Slope-intercept form]	\parbox[t]{7cm}{$y=mx+b$, where $m$ is slope and $b$ is the $y$ coordinate of the $y$ intercept. $(0,b)$ is the $y$ intercept} 	
	\item[Point-slope form]	\parbox[t]{7cm}{$y-y_1=m(x-x_1)$ where $m$ is the slope and $(x_1,y_1)$ is a \gls{point} on the line}
\end{itemize}
The most common form is the slope-intercept form, $y = mx + b$. In this case the given equation
is in terms of $y$; the slope $m$, and the $y$ -intercept $(0,b)$ can readily be determined.
There are many ways to find the slope of a line which is needed to find any equation of the line.
The various ways to find slope are given and illustrated below.

Given two point on a line, we can find the slope by using the slope formula
\[
	m = \frac{y_2-y_1}{x_2-x_1}
\]
where the two point $(x_1,y_1)$ and $(x_2,y_2)$ are given.
	
\begin{myexample}
Find the slope of the line that connects $(-5, 4)$ and $(-1, 0)$
\end{myexample}
\begin{myProof}
	We begin with the formula for the slope
	\[
		m = \frac{y_2-y_1}{x_2-x_1}
	\]
	and we substitute $(x_1,y_1) = (-5,4)$, and $(x_2,y_2) = (-1,0)$ and obtain
	\begin{align*}
		m & =	 	\frac{0-4}{-1-(-5)} \\
		  & =		\frac{-4}{4}         \\
		  & =		-1                   
	\end{align*}
	Notice how we have lined up the $=$ symbols, and have worked vertically- this is the way
	that mathematics should be presented, and it is required for you to adopt this.
\end{myProof}
				
\begin{myexample}
Find the slope for the line represented by the equation $2x+3y=6$
\end{myexample}
\begin{myProof}
	In this example we use the equation of the line to find the slope. We first need to rearrange the equation for $y$ by moving the $2x$ to the other side by subtracting $2x$ from 
	both sides to give
	\[
		3y = 6- 2x
	\]
	Now we divide all terms by $3$ which gives
	\[
		\frac{3y}{3} = \frac{6-2x}{3}
	\]
	which can be simplified to
	\begin{align*}
		y = 2-\frac{2x}{3} ~\textrm{or}~ y = -\frac{2x}{3}+2 ~\textrm{or}~ y=-\frac{2}{3}x+2 
	\end{align*}
	Of the three forms given here, the most preferable is the last one, that is
	\[
		y = -\frac{2}{3}x +2
	\]
	The reason this is most preferable is that it separates the {\em \gls{coefficient}} of $x$ from the \gls{variable}. 
	This allows us to read off that this line has a slope of $-\frac{2}{3}$.
\end{myProof}
									
\begin{myexample}
Another way to find the slope of the line $2x+3y=6$
\end{myexample}
\begin{myProof}
	When you find three point that are on the line and are getting ready to graph it, you
	then have a couple more choices on how to find the slope. 
																							
	We begin by finding 3 point on the line by choosing values for $x$ and then finding the corresponding
	value of $y$
	\begin{tightcenter}
		\begin{tabular}{SS}
			\toprule
            {$x$} & {$y$}  \\
			\midrule
			0 & 2  \\
			3 & 0  \\
			6 & -2 \\
			\bottomrule
		\end{tabular}
	\end{tightcenter}
	Remember that slope is also known as $\frac{rise}{run}$. The change in values in the second
	column is the rise and the corresponding change of values in the first column is the run. Therefore, 
	the slope of this line is $\dd\frac{-2}{3}$ or $\dd-\frac{2}{3}$, which is what we found in the last example.
\end{myProof}
									
\begin{myexample}\label{ex:firstgraph}
And another way to find the slope of the line $2x+3y=6$.
\end{myexample}
\begin{myProof}
	Find $3$ point on the line by choosing values for $x$ and then finding $y$
	\begin{tightcenter}
		\begin{tabular}{SS}
			\toprule
            {$x$} & {$y$}  \\
			\midrule
			0 & 2  \\
			3 & 0  \\
			6 & -2 \\
			\bottomrule
		\end{tabular}
	\end{tightcenter}
	Now graph these point, and connect the point to graph your line as shown in \cref{fig:firstline}.
	\begin{figure}
		\begin{subfigure}{.5\textwidth}
			\centering
			\begin{tikzpicture}
				\begin{axis}[%
						framed,
						xmin=-5,xmax=7,
						ymin=-5,ymax=5,
						grid=major,
						xtick={-4,...,6},
						ytick={-4,...,4},
						width=.9\textwidth,
					]
					\addplot+[soldot] coordinates{	(0,2)	(3,0) (6,-2) };
				\end{axis}
			\end{tikzpicture}
			\caption{Graph the point}
		\end{subfigure}%
		\begin{subfigure}{.5\textwidth}
			\centering
			\begin{tikzpicture}
				\begin{axis}[
						framed,
						xmin=-5,xmax=7,
						ymin=-5,ymax=5,
						grid=major,
						xtick={-4,...,6},
						ytick={-4,...,4},
						width=.9\textwidth,
					]
					\addplot expression[domain=-4.5:7]{2-2/3*x};
					\legend{$3y+2x=6$};
					\addplot[cmhplot,soldot] 	coordinates{	(0,2) (3,0) (6,-2) };
					\draw (axis cs: -3,4) -- node[anchor=east]{Rise} (axis cs: -3,2);
					\draw (axis cs: -3,2) -- node[anchor=north]{Run} (axis cs: 0,2);
				\end{axis}
			\end{tikzpicture}
			\caption{Draw a line through them}
		\end{subfigure}%
		\caption{The line defined in \cref{ex:firstgraph}}
		\label{fig:firstline}
	\end{figure}
																							
	Next make any right triangle using any two point on the line, and remember that the rise is
	how many units you go up (or down, in which case the rise is negative) from the first point
	to the second point, and the run is how many units you go the right from the first point.
																							
	In this case, we see that we go
	\begin{tightcenter}
		down 2
	\end{tightcenter}
	which means that the rise is $-2$. 
																										
	We also see that we go
	\begin{tightcenter}right 3\end{tightcenter}
	which means that the run is 3. 
																														
	Therefore
	\begin{align*}
		m & =		\frac{\mathrm{rise}}{\mathrm{run}} \\
		  & =		-\frac{2}{3}                       
	\end{align*}
	So in the last three examples, we obtained the same slope using three different methods. We also reviewed
	how to graph a line. 
																														
	Please use your text to review more about graphing.
\end{myProof}
																			
\begin{myexample}\label{ex:graphbyintercepts}
Graph 
\[
	y = 3x - 5
\]
by finding the $x$ and $y$ intercept.
\end{myexample}
\begin{myProof}
	We need to find the horizontal and vertical intercept. 
	\begin{itemize}[leftmargin=4cm]
		\item[$x$-intercept]
		We find the $x$ intercept by setting $y=0$ and solving for $x$
		\[
			0 = 3 x - 5
		\]
		which gives 
		\[
			x = \frac{5}{3}
		\]
		The horizontal intercept is therefore $\left(\frac{5}{3},0\right)$.
		\item[$y$-intercept]
		We find the $y$ intercept by setting $x=0$ and solving for $y$
		\[
			y = -5
		\]
		The vertical intercept is therefore $(0,-5)$. 
	\end{itemize} 
	We plot the intercept  and line in \cref{fig:graphintercepts}.
	\begin{figure}
		\begin{subfigure}{.5\textwidth}
			\centering
			\begin{tikzpicture}
				\begin{axis}[
						framed,
						xmin=-10,xmax=10,
						ymin=-10,ymax=10,
						xtick={-8,-6,...,8},
						minor xtick={-9,-7,...,9},
						ytick={-8,-6,...,8},
						minor ytick={-9,-7,...,9},
						grid=both,
						width=0.9\textwidth,
					]
					\addplot+[soldot]coordinates{ (1.66,0)  (0,-5)  };
				\end{axis}
			\end{tikzpicture}
			\caption{Graph the intercept}
		\end{subfigure}
		\begin{subfigure}{.5\textwidth}
			\centering
			\begin{tikzpicture}
				\begin{axis}[
						framed,
						xmin=-10,xmax=10,
						ymin=-10,ymax=10,
						xtick={-8,-6,...,8},
						minor xtick={-9,-7,...,9},
						ytick={-8,-6,...,8},
						minor ytick={-9,-7,...,9},
						grid=both,
						legend pos=north west,
						width=0.9\textwidth,
					]
					\addplot expression[domain=-1.5:5]{3*x-5};
					\legend{$y=3x-5$};
					\addplot[cmhplot,soldot] coordinates{ (1.66,0)  (0,-5)  };
				\end{axis}
			\end{tikzpicture}
			\caption{Draw a line through them}
		\end{subfigure}
		\caption{The line defined in \cref{ex:graphbyintercepts}.}
		\label{fig:graphintercepts}
	\end{figure}
																																								
	\mbox{}
\end{myProof} 
																													
\section{Solving systems of equations by graphing}
\textref{5.1}{290}%
A \gls{system} of \gls{linear} equation (in two variable) consists of two equation in two unknowns, $x$ and
$y$ say. Examples of such system are
\begin{equation*}
	\begin{cases}
		3x+2y = 1 &   \\
		4x-y=2    &   
	\end{cases}
\end{equation*}
Our goal in this first part of the course is to \gls{solve} such a system by finding unique values for $x$ and
$y$ that satisfy both equation. There are a number of ways that we can do this; the method that we
describe in the section is by graphing.
																												
Solving system of equation using graphing is a good way to represent the idea visually. For
those of you who are visual learners, this will indeed convince you of the various possibilities, and
add to the understanding of why the answers appear as they do. In general the way to do this is as
follows:
\begin{enumerate}
	\item Graph the first equation using your preferred method (see \cref{sec:reviewgraphlines}).
	\item Graph the second equation using your preferred method (see \cref{sec:reviewgraphlines}).
	\item See if the lines are:
	\begin{enumerate}
		\item intersecting in one point $–$ this is the \gls{solution}
		\item parallel - there is no solution because they are inconsistent
		\item the same line - All Real Numbers is the answer and they are dependent lines.
	\end{enumerate}	 
	\item If there is a solution of a point, label the coordinates on the graph. Then write the
	coordinates as your final answer.
\end{enumerate}
We demonstrate the techniques with examples.
																												
\begin{myexample}
Solve the system by graphing
\begin{equation}\label{ex:firstsystem}
	\begin{cases}
		x+y = 2 &   \\
		x-y = 4 &   
	\end{cases}
\end{equation}
\end{myexample}
\begin{myProof}
	We can choose {\em any} values for $x$ that we wish, we will just be finding different point on the line
	\begin{tightcenter}
		\begin{minipage}{4cm}
			{\color{violet}
				\begin{tabular}{SS}
					\toprule
					{$x$} & {$y = 2-x$} \\	
					\midrule
					0 & 2       \\
					2 & 0       \\
					4 & -2      \\
					\bottomrule
				\end{tabular}
			}
		\end{minipage}
		\begin{minipage}{4cm}
			{\color{red}
				\begin{tabular}{SS}
					\toprule
					{$x$} & {$y = x-4$} \\	
					\midrule
					0 &  -4     \\
					2 &  -2     \\
					5 &  1      \\
					\bottomrule
				\end{tabular}
			}
		\end{minipage}
	\end{tightcenter}
	We now graph these point on a graph, and connect the point. We label each line
	with its equation. Note, they can labelled in their original form if you prefer $x+y=2$ and
	$x-y=4$, or in slope intercept form as demonstrated.
																																									
	\begin{figure}[!h]
		\begin{subfigure}{.5\textwidth}
			\centering
			\begin{tikzpicture}
				\begin{axis}[
						framed,
						xmin=-5,xmax=6,
						ymin=-5,ymax=5,
						grid=major,
						xtick={-4,...,5},
						ytick={-4,...,4},
						width=0.9\textwidth,
					]
					\addplot+[soldot] coordinates{	(0,2)  (2,0)  (4,-2) };
					\addplot+[soldot] coordinates{	(0,-4)  (2,-2)  (5,1)  (3,-1)  };
				\end{axis}
			\end{tikzpicture}
			\caption{Graph the point}
		\end{subfigure}
		\begin{subfigure}{.5\textwidth}
			\centering
			\begin{tikzpicture}
				\begin{axis}[
						framed,
						xmin=-5,xmax=6,
						ymin=-5,ymax=5,
						grid=major,
						xtick={-4,...,5},
						ytick={-4,...,4},
						width=0.9\textwidth,
					]
					\addplot expression[domain=-1:6]{x-4};
					\addplot expression[domain=-3:6]{2-x};
					\legend{$y=x-4$, $y=2-x$};
					\addplot+[soldot] coordinates{ (0,-4)  (2,-2)  (5,1)  (3,-1) };
					\addplot+[soldot] coordinates{	(0,2)  (2,0)  (4,-2) };
				\end{axis}
			\end{tikzpicture}
			\caption{Draw a line through them}
		\end{subfigure}
		\caption{The system of equation defined in \cref{ex:firstsystem}.}
	\end{figure}
	\FloatBarrier
																																									
	We can see that these lines intersect at the point $(3,-1)$. This is the {\em only}
	point that both equation have in common, and $(3,-1)$ will make both of the equation
	true; in other words $(3,-1)$ is the solution to the original system of linear equation. It
	remains for us to check that this is true.
																																									
	Once we have found our solution, we must always check it by plugging the numbers
	back into the original system.
																																									
	If $x=3$ and $y=-1$ then in the first equation
	\begin{align*}
		x+y & =	3+(-1) \\
		    & =	2,     
	\end{align*}
	which is as required. In the second equation
	\begin{align*}
		x-y & =	3-(-1) \\
		    & =	3+1    \\
		    & =	4,     
	\end{align*}
	which is again as required. 
																																									
	We must always {\em answer the question in a sentence}. The solution to the system $(3,-1)$. 
	As there is only one solution, we say that the system is {\em consistent}.
\end{myProof}
																											
\begin{myexample}
Solve the system by graphing
\begin{equation}\label{ex:secondsystem}
	\begin{cases}
		y	=	2x+1  &   \\
		y	=	-2x-3 &   
	\end{cases}
\end{equation}
\end{myexample}
\begin{myProof}
	We proceed as in the previous example by making a table of values for each of the two equation
	\begin{tightcenter}
		\begin{minipage}{4cm}
			{\color{red}
				\begin{tabular}{SS}
					\toprule
					{$x$}  & {$y = 2x+1$} \\	
					\midrule
					0  & 1        \\
					2  & 5        \\
					-2 & -3       \\
					\bottomrule
				\end{tabular}
			}
		\end{minipage}
		\begin{minipage}{4cm}
			{\color{violet}
				\begin{tabular}{SS}
					\toprule
					{$x$}  & {$y = -2x-3$} \\	
					\midrule
					0  & -3        \\
					1  & -5        \\
					-2 & 1         \\
					\bottomrule
				\end{tabular}
			}
		\end{minipage}
	\end{tightcenter}
	Note: the $y$ value that corresponds to $x=0$ is the vertical intercept of the graph. 
																																									
	We graph the point, then connect them, and label the equation. We also mark the 
	point of intersection, find the coordinates and write them as our answer.
																																									
	\begin{figure}[!h]
		\centering
		\begin{tikzpicture}
			\begin{axis}[
					framed,
					xmin=-5,xmax=5,
					ymin=-6,ymax=6,
					grid=major,
					xtick={-4,...,4},
					ytick={-5,...,5},
					legend pos=south east,
				]
				\addplot expression[domain=-3.5:2.5]{2*x+1};
				\addplot expression[domain=-4.5:1.5]{-2*x-3};
				\legend{$y=2x+1$, $y=-2x-3$};
				\addplot+[soldot] coordinates{	(-2,-3)  (0,1)  (2,5) };
				\addplot+[soldot] coordinates{	(-2,1)  (0,-3)  (1,-5)  (-1,-1)  };
			\end{axis}
		\end{tikzpicture}
		\caption{The system defined in \cref{ex:secondsystem}.}
	\end{figure}
	\FloatBarrier
																																									
	The point of intersection is $(-1,-1)$, and this is the solution to the system. As before, 
	we need to check to make sure that this point satisfies both of the original equation (an exercise for you).
																																									
	As there is only one solution, we say that the system is {\em consistent}.
\end{myProof}
																											
\begin{myexample}
Solve the following system by graphing
\begin{equation}\label{eq:thirdsystem}
	\begin{cases}
		y		=	3x-1 &   \\
		y		=	3x+2 &   
	\end{cases}
\end{equation}
{}
\end{myexample}
																											
\begin{myProof}
	We proceed using the established routine from the previous two examples. 
																																									
	We find some sample point:
	\begin{tightcenter}
		\begin{minipage}{4cm}
			{\color{violet}
				\begin{tabular}{SS}
					\toprule
					{$x$}  & {$y = 3x-1$} \\
					\midrule
					-1 & -4       \\
					0  & -1       \\
					1  & 2        \\
					\bottomrule
				\end{tabular}
			}
		\end{minipage}
		\begin{minipage}{4cm}
			{\color{red}
				\begin{tabular}{SS}
					\toprule
					{$x$}  & {$y = 3x+2$} \\	
					\midrule
					-1 & -1       \\
					0  & 2        \\
					1  & 5        \\
					\bottomrule
				\end{tabular}
			}
		\end{minipage}
	\end{tightcenter}
	We graph the point, then connect them, and label the equation. We also mark the point
	of intersection, find the coordinates and write them as our answer.
																																									
	There is no point of intersection because they are parallel lines. We should have
	noted this when we first saw the equation, as they both have the same slope $m=3$ .
																																									
	\begin{figure}[!h]
		\centering
		\begin{tikzpicture}
			\begin{axis}[
					framed,
					xmin=-5,xmax=5,
					ymin=-6,ymax=6,
					grid=major,
					xtick={-4,...,4},
					ytick={-5,...,5},
					legend pos=north west,
				]
				\addplot expression[domain=-2.6:1.3]{3*x+2};
				\addplot expression[domain=-1.6:2.1]{3*x-1};
				\legend{$y=3x+2$, $y=3x-1$};
				\addplot+[soldot] coordinates{	(-1,-1)  (0,2)  (1,5) };
				\addplot+[soldot] coordinates{	(-1,-4)  (0,-1)  (1,2)  };
			\end{axis}
		\end{tikzpicture}
		\caption{The system of equation defined in \cref{eq:thirdsystem}.}
	\end{figure}
	\FloatBarrier
																																									
	Therefore this system has {\em no solution}. As there is no solution, we say that the
	system is {\em inconsistent}.
\end{myProof}
																											
\begin{myexample}
Solve the following system by graphing
\begin{equation}\label{eq:consistentsys}
	\begin{cases}
		2x-y		=	0 &   \\
		y		=	2x   &   
	\end{cases}
\end{equation}
\end{myexample}
\begin{myProof}
	Proceeding as in the previous examples, we achieve the graph shown in \cref{fig:consistentsyst}.
																																									
	\begin{figure}[!h]
		\centering
		\begin{tikzpicture}
			\begin{axis}[
					framed,
					xmin=-5,xmax=5,
					ymin=-5,ymax=5,
					grid=major,
					xtick={-4,...,4},
					ytick={-4,...,4},
					legend pos=north west
				]
				\addplot expression[domain=-10:10]{2*x};
				\addplot expression[domain=-2.5:2.5]{2*x};
				\legend{$y=2x$, $2x-y=0$};
			\end{axis}
		\end{tikzpicture}
		\caption{The system defined in \cref{eq:consistentsys}. Notice that there is only 1 line showing,
		as both equation represent the same line.}
		\label{fig:consistentsyst}
	\end{figure}
	%\FloatBarrier
																																									
	In this case, we notice that the equation represent the same line, in which case
	any ordered pair that satisfies one of the equation will automatically satisfy the other. In
	this case the system has infinitely many solution. In this case, the system is consistent,
	but there is not a {\em unique} solution.
\end{myProof}
																											
\section{Solving systems using substitution}
\textref{5.2}{301}%
We saw in the previous section that we can solve a system of linear equation by graphing. Another
way to solve a system of equation is by using the substitution method. The main point in this
method is to eliminate one variable from one of the equation, and substitute it into the other.
Sometimes it is easy to see why the substitution method would be the best. Other times, a different
method might be preferable (see next section).
																											
Here is an outline of the steps to solving a system of equation by substitution.
\begin{steps}
	\item See if one of the variable in one of the equation has a coefficient of 1. This is the ideal
	situation and occurs quite often. If this does not occur, you still can continue with this
	method.
	\item Solve one of the equation for either variable- choose whichever seems like the most
	simple.
	\item Now you substitute this \gls{expression} into the OTHER equation.
	\item Solve the equation for the variable.
	\item Now substitute this number in either original equation and find the value for the other
	variable.
	\item Now that you have the values for both of the variable, write the answer as an ordered pair.
	\item Check to make sure that the ordered pair makes both of the original equation true.
\end{steps}
We demonstrate this technique with a series of examples.
																											
\begin{myexample}
Solve this system by the substitution method:
\begin{align*}
	x+y & =	2 \\
	x-y & =4  
\end{align*}
\end{myexample}
\begin{myProof}
	\begin{itemize}
		\item Note that the coefficients of all the variable are 1, which means that rearranging for
		either variable $x$ or $y$ will be straightforward.
		\item   We choose the second equation to solve for $x$ as this will leave no negative signs. (We
		could just as well use the first equation if we choose.) We add $y$ to both sides to obtain
		\begin{equation*}
			x= y +4
		\end{equation*}
		\item   We substitute this expression for $x$ into the first equation to obtain
		\begin{equation*}
			(y+4 )+ y=2 ,
		\end{equation*}
		which we \gls{simplify} to
		\begin{equation*}
			2 y+4=2 ,
		\end{equation*}
		and therefore, subtracting 4 from both sides gives
		\begin{equation*}
			2 y=- 2 ,
		\end{equation*}
		which on dividing by 2 gives
		\begin{equation*}
			y =- 1 .
		\end{equation*}
		\item   Now that we have a value for y we substitute this into the expression we have for $x$,
		which gives
		\begin{align*}
			x & = -1 +4 \\
			  & = 3     
		\end{align*}
		\item According to our work, the answer is $(3, -1)$. But we need to check that this is true by
		substituting it into the original equation (exercise).
	\end{itemize}
\end{myProof}
																											
\begin{myexample}
Solve this system by the substitution method:
\begin{align*}
	x+2y  & =	7  \\
	4x-2y & =	-2 
\end{align*}
\end{myexample}
\begin{myProof}
	\begin{itemize}
		\item   Note that in the first equation the coefficient of $x$ is 1; this means that rearranging for $x$ from
		this equation will be neater. If we had chosen to rearrange for $y$ in the first equation, then
		we would need to divide by 2, which would leave fractions in the expressions- this is not as
		neat as using our first observation.
		\item   We solve the first equation for $x$ by subtracting $2 y$ from both sides
		\begin{equation*}
			x = 7-2y
		\end{equation*}
		\item   We now substitute this into the second equation which gives
		\begin{align*}
			4(7-2y)-2y & =		-2  \\
			28-8y-2y   & =		-2  \\
			28-10y     & =		-2  \\
			30         & =		10y 
		\end{align*}
		and therefore
		\begin{equation*}
			y=3
		\end{equation*}
		\item   We obtain $x$ by substituting this value of $y$ back into our original expression for $x$
		\begin{align*}
			x & =		7-2(3) \\
			  & =		1      
		\end{align*}
		\item According to our work, the answer is $(1,3)$ . We need to check that this is true by
		substituting it into the original equation (exercise).
	\end{itemize}
\end{myProof}
																											
\section{Solving systems by addition}
\textref{5.3}{309}%
Remember that any time you add opposites (also known as additive inverses) you get 0. This will make it possible
to solve system of equation using a third method called the {\em addition method} (or the elimination method depending on the
text and/or instructor). Our goal in this method is to eliminate one of the variable by using addition. This may 
be achieved by using the equation in their given form, or by multiplying one (or both) equation(s) by a number to make
the elimination occur. This will be made clearer in the examples in this section.
																											
Here is an outline of the steps for using the addition method:
\begin{steps}
	\item Write the equation underneath each other in standard form, and line up like terms and the '=' symbols
	\item See if one of the variable will `add out' as the equation by seeing if either of the variable have
	coefficients that are opposites (for example $2x$ and $-2x$)
	\item If not, you will need to multiply one or both equation by a coefficient so that one variable term will `add out'
	\item Add the two equation together. (In some cases you might have to multiply one of the equation by -1 to get
	things to `add out')
	\item Solve for the remaining variable
	\item Substitute this value in either equation
	\item Solve for the other variable
	\item Check your result by plugging the numbers back into the original system
	\item Write your answer as an ordered pair.
\end{steps} 
																											
We demonstrate this technique with a series of examples.
																											
\begin{myexample}
Solve the system by the addition method. If there is no solution or an number of solution, state so
\[
	\begin{cases}
		x+y  =6  &   \\
		x-y  =-2 &   
	\end{cases}
\]
{}
\end{myexample}
\begin{myProof}
	\begin{steps}
		\item We begin by writing the equation underneath each other
		\begin{alignat*}{3}
			x+y & {}={} & 6  \\
			x-y & {}={} & -2 
		\end{alignat*}
		\item If we add them together, will either variable add out? {\color{red} Yes, so we need to do \#3}
		\item When we add them together we get
		\[
			2x = 4
		\]
		\item Solve
		\[
			x = 2
		\]
		\item Substitute this value in one equation
		\[
			2+y=6
		\]		
		\item Solve for the other variable
		\[
			y = 4
		\]
		\item Check in both equation
		\begin{align*}
			x-y & \stackrel{?}{=} -2 & x+y & \stackrel{?}{=}	 6 \\
			2-4 & \stackrel{?}{=} -2 & 2+4 & \stackrel{?}{=}	 6 \\
			-2  & \stackrel{?}{=} -2 & 6   & \stackrel{?}{=}	 6 
		\end{align*} 
		Both of which are true.
		\item Write the answer as an ordered pair: $(2,4)$.
	\end{steps} 
	Naturally, since you have already taken Math 60, you know the problems are not always going to be this clear.
	Practicing these types of problems is a good way to get the method firm in your mind.
\end{myProof} 
																																						
\begin{myexample}
Find the solution to the system
\[
	\begin{cases}
		2x+y  =1 &   \\
		3x+y  =0 &   
	\end{cases}
\]
\end{myexample}
\begin{myProof}
	\begin{steps}
		\item Write the equation underneath each other
		\begin{align*}
			3x+y & =0 \\
			2x+y & =1 
		\end{align*} 
		\item Will one of the variable `add out' as is? {\color{red} NO}
		\item However, the $y$ terms have the same coefficients, but not opposite signs. If we
		multiply the bottom equation by -1, they will add out easily. (Note, you could multiply the top equation
		by -1 instead)
		\begin{align*}
			-2x-y & =	-1 \\
			3x+y  & =	0  
		\end{align*} 
		\item Now add the 2 equation
		\[
			1x = -1
		\]	
																																																																						
		\item Solve
		\[
			x = -1
		\]
		\item Substitute into one equation
		\[
			3(-1)+y=0
		\]
		\item Solve for $y$
		\[
			y=3
		\]
		\item Check (in both equation)
		\begin{align*}
			3x+y=0  & \stackrel{?}{=} 0 & 2x+y    & \stackrel{?}{=}	 1 \\
			3(-1)+3 & \stackrel{?}{=} 0 & 2(-1)+3 & \stackrel{?}{=}	 1 \\
			-3+3    & \stackrel{?}{=} 0 & -2+3    & \stackrel{?}{=}	 1 \\
			0       & =  0              & 1       & =			 1             
		\end{align*} 
		\item Finally we write our answer as an ordered pair: $(-1,3)$.
	\end{steps} 
\end{myProof}
																																																	
\begin{myexample}
Solve the following system by using the addition method
\[
	\begin{cases}
		2x-y  =-7 &   \\
		3x+2y =0  &   
	\end{cases}
\]
\end{myexample}
\begin{myProof}
	\begin{steps}
		\item Write the equation underneath each other
		\begin{align*}
			2x-y  & =-7 \\
			3x+2y & =0  
		\end{align*} 
		\item No variable will add out as the equation stand.
		\item Since the $y$ terms already have opposite signs, we will work
		with them first. We multiply the top equation by 2.
		\[
			2(2x-y)=2(-7)
		\]
		and therefore
		\[
			4x-2y=-14
		\]
		\item We now add the equation together
		\begin{align*}
			4x-2y & =-14 \\
			3x+2y & =0   
		\end{align*}
		which gives
		\[
			7x = -14
		\]
		\item Solve:
		\[
			x = -2
		\]
		\item Substitute into the other equation
		\[
			3(-2)+2y=0
		\]
		\item and now solve for $y$
		\begin{align*}
			-6+2y & = 0 \\
			2y    & = 6 \\
			y     & = 3 
		\end{align*} 
		{}
		\item We now check the solution in the other equation
		\[
			2(-2)-3 = -7
		\]
		which is true.
		\item We conclude by saying that the solution to the system is $(-2,3)$.
	\end{steps} 
\end{myProof} 
																																																														
\begin{myexample}
Solve the following system by using the addition method
\[
\begin{cases}
		5x+3y  =27 &\\
		7x-2y  =13 &
      \end{cases}
\]
{}
\end{myexample}
\begin{myProof}
	\begin{steps}
		\item Write equation underneath each other
		\begin{align*}
			5x+3y & =27 \\
			7x-2y & =13 
		\end{align*} 
		\item Nothing adds out.
		\item Since the $y$ terms have opposite signs, we will 
		try to get them to add out. We multiply the top equation by $2$
		and multiply the second equation by $3$
		\begin{align*}
			2(5x+3y) & =2(27) \\
			3(7x-2y) & =3(13) 
		\end{align*} 
		which gives
		\begin{align*}
			10x+6y & = 54 \\ 
			21x-6y & = 39 
		\end{align*}
		\item Adding the equation will eliminate the $y$ term
		\[
			31x = 93
		\]
		\item And we can now solve for $x$
		\[
			x = 3
		\]
		\item Substitute into the other equation
		\[
			5(3)+3y = 27
		\]	
		\item And now solve for $y$
		\[
			y = 4
		\]
		\item Check in the first equation
		\[
			5(3)+3(4)\stackrel{?}{=} 27
		\]
		which is a true statement.
		\item We conclude by stating that the solution to the system is $(3,4)$.
	\end{steps} 
\end{myProof} 
																																																																										
\begin{myexample}
Consider the following miscellaneous examples. In each case, decide which method
you would use to solve the system.
\end{myexample}
\begin{tabular}{p{3cm}cp{8cm}}
	\toprule
	Example & Choice                         & Why                                                                      \\ 
	\midrule
	$
	\begin{cases}
	x+y =6 &	\\
	2x+y =8 & 
	\end{cases}
	$
	        & Substitution {\em or }Addition & Because it is easy to solve for $x$ in the 1st equation                  
	{\em or}
	Because it is easy to multiply the first equation by -1 \\ 
	$
	\begin{cases}
	3x-7y =13 &\\
	6x+5y =7  & 
	\end{cases}
	$
	        & Addition                       & Because there is no obvious way to use substitution                      \\ 
	$
	\begin{cases}
	x-7y  =13 & \\
	6x+5y =7 &   
	\end{cases}
	$
	        & Substitution                   & Because the first equation can easily be solved for one of the variable \\
	\bottomrule
\end{tabular}
																																																																										
We have discussed three approaches to solving a system of linear equation in two
variable:
\begin{enumerate}
	\item graphing
	\item substitution
	\item addition
\end{enumerate} 
From this point on you will be able to choose which method works best depending
on your equation, unless you are specifically instructed to use a certain method.

%+*** mainfile.tex
% !arara: pdflatex: { files: [ mainfile.tex ] }
% arara: makechapters: { files:[mainfile], items: [module2], makeChapGlossaries: yes}
% !arara: indent: { overwrite: on, trace: yes, localSettings: on}
\chapter{Polynomials}
\minitoc
\section{Problem solving Using systems of equations}
\textref{5.4}{318}%
Nearly every person has trouble doing word problems at one time or another. It is mostly true that the hardest part of word problems is setting them up- that is, translating the words of the problem into mathematics. Once this is done, it is `simply' a case of using the skills we have acquired to \gls{solve} them. There are a few guidelines that will help us with these types of problems.

Remember these are guidelines for solving word problems involving systems of \gls{linear} \gls{equation}. 
\begin{steps}
	\item READ and study the problem
	\item Make a key, chart or sketch in order to familiarize yourself with the problem.
	\item Write down the equation, being careful to introduce all variables.
	\item Choose the type of method for solving (by graphing, substitution, or elimination).
	\item Solve for the first \gls{variable}. 
	\item Solve for the second variable.
	\item Write your answers in a complete sentence. 
	\item Check the answers mathematically, and for common sense.
\end{steps}
We will demonstrate some techniques with the following examples, but remember that sometimes
there are other ways to solve the problem; such an example is described in 
\href{http://math.stackexchange.com/questions/478212/is-there-another-simpler-method-to-solve-this-elementary-school-math-problem}{A systems problem involving 
bicycle wheels}.

\begin{myexample}
A local flower shop has sold 8 roses and 10 carnations for \$110 to the first customer of the day. Later that day, another customer bought a dozen roses and a dozen carnations for \$156. What is the cost of one rose at this floral shop?
\end{myexample}
\begin{myProof}
	First, let's do some experimentation. We are asked to find the cost of one rose, and one carnation at this shop. If one rose costs \$5, and one carnation costs \$10, then the first customer would have had to pay
	\[
		8(5)+10(10)=140
	\]
	which is clearly incorrect, as we are told that this customer was only charged \$110. We could use similar reasoning for an experiment with roses costing \$7, and carnations costing \$11, which would lead to the first customer being charged
	\[
		8(7)+10(11) = 166
	\]
	which is again incorrect. We could obviously proceed by trial and error in this way (which is valid in its own right), but our goal is to set up this problem algebraically.
		
	Now that we have a feel for the problem, let us declare our variables; let
	\begin{tightcenter}
		$r$= the cost of 1 rose, $c$ = cost of 1 carnation
	\end{tightcenter}
	From our previous work, the equation will be
	\begin{align*}
		8r + 10c & =		110 \\
		12r +12c & =		156 
	\end{align*}
	Now that we have set up the equation, we can solve them using any of the techniques illustrated in Module 1. Here we will demonstrate the method of addition
	\begin{alignat*}{2}
		\text{Multiply the top equation by 	\color{red}{3}:}      &   & \color{red}{3}(8r+10c)     & = \color{red}{3}(110)     \\
		\text{Multiply the bottom equation by \color{green}{-2}:} &   & \color{green}{-2}(12r+12c) & =  \color{green}{-2}(156) 
	\end{alignat*}
	This gives
	\begin{align*}
		24r+30c  & =		330  \\
		-24r-24c & =		-312 
	\end{align*}
	We can now add the two equation together to give
	\[
		-6c=-18
	\]
	from which we can solve for $c$ to give
	\[
		c=3
	\]
	We conclude by writing our \gls{solution} in a complete sentence:
	\begin{tightcenter}
		{\em At this floral shop, a rose cost \$10 and a carnation costs \$3.}
	\end{tightcenter}
	It remains for us to check the solution in the original problem. With $r=10$ and $c=3$, the first
	customer pays
	\[
		8(10)+10(3)=110,
	\]
	and the second customer pays
	\[
		12(10)+12(3)=156,
	\]
	both of which are as required.
\end{myProof}

\begin{myexample}
At a certain school event, an adult ticket cost \$8 while a child ticket is \$4. If 450 tickets were sold for a total of \$3000, how many
children attended this event?
\end{myexample}
\begin{myProof}
	A good way to record information is to use a chart. This works well for a variety of
	word problems. 
		
	We can summarize the information given to us in the problem as follows
	\begin{tightcenter}
		\begin{tabular}{lccc}
			\toprule
			         & Amount & Unit cost & Total value \\ 
			\midrule
			Adults   & $a$    & $8$       & $8a$        \\
			Children & $c$    & $4$       & $4c$        \\
			Totals   & $450$  &           & $3000$      \\
			\bottomrule
		\end{tabular}
	\end{tightcenter}
	We can read off two pieces of information from this chart; one from the `Amount' column, and one
	from the `Total Value' column; these give the following equation (respectively)
	\begin{align*}
		a+c   & =		450  \\
		8a+4c & =		3000 
	\end{align*}
	If we multiply the first equation by $-4$, and leave the second equation alone, then
	we obtain
	\begin{align*}
		-4a-4c & =		-1800 \\
		8a+4c  & =		3000  
	\end{align*}
	Adding the two equation together gives
	\[
		4a = 1200
	\]	
	and therefore 
	\[
		a = 300
	\]
	We can work out the number of children that attended by calculating $450-300=150$. We conclude
	that
	\begin{tightcenter}
		{\em At this school event 300 adults and 150 children attended.}
	\end{tightcenter}
	It is left for us to check these answers in the same that we did for the previous example. We
	can clearly see that a total of
	\[
		150+300=450
	\]
	tickets were sold (satisfying one of the pieces of information), and that a total of
	\[
		8(300)+4(150)=3000
	\]
	dollars was made from the sale of tickets.
\end{myProof}

\section{Polynomials}
\textref{6.1}{342}%
The algebraic \gls{expression} $-0.75d^4+3d^4+5$ describes the billions of viral particles in 
our bodies after $d$ days of invasion
	The expression enables us to determine the day on which there is a maximum
	number of viral particles and, consequently, the day we feel sickest.
	The expression is an example of a \gls{polynomial}

\begin{myDefinition}
	\begin{description}
		\item[polynomial] A single term or the sum (remember that subtraction can be written as a sum) of two or
		more terms containing variables only in the numerator, and all exponents are natural numbers.
		\item[Degree] The \gls{degree} of a polynomial is the highest power of any term of the polynomial.  
	\end{description}
		 
\end{myDefinition}

\begin{myexample}
Consider the polynomial
\[
	9x^6-5x^4+7x^2+3
\]
\end{myexample}
We note that the highest order term is $9x^6$. The polynomial therefore has degree 6.
\begin{myexample}
Consider the polynomial 
\[
	-4 + 6x^2-4x^5-8x^3
\]
\begin{enumerate}
	\item How many terms are there?
	\item What is the degree of each term?
	\item What is the degree of the polynomial (the highest degree of all the terms of the polynomial)
	\item State the polynomial in standard form.
\end{enumerate} 
\end{myexample}
\begin{myProof}
	\begin{enumerate}
		\item There are 4 terms: $-4$, $6x^2$, $-4x^5$, $-8x^3$
		\item $-4$ has degree $0$, $6x^2$ has degree 2, $-4x^5$ has degree 5, $8x^3$ has a degree of 3
		\item The degree of the polynomial is 5, since the highest power of $x$ is 5
		\item The standard form for this polynomial is
		\[
			-4x^5-8x^3+6x^2-4
		\]
		Note that the standard form of a polynomial is written with the highest power
		of $x$ written first, and then decreasing powers of $x$.
	\end{enumerate} 
\end{myProof} 
\begin{myDefinition}
	{\bfseries Special polynomials}: There are three forms of polynomial that arise frequently, and are worthy of their
	own terminology:
	\begin{enumerate}
		\item \gls{monomial}- a polynomial with one term
		\item \gls{binomial}- a polynomial with two terms
		\item \Gls{trinomial}- a polynomial with three terms
	\end{enumerate} 
\end{myDefinition}
\section{Exponent Rules}
There are 3 basic rules for exponents that will help you multiply polynomial
\begin{enumerate}
	\item The Product Rule- How to multiply exponential expression
	\item The Power Rule- How to raise an exponential expression to a power (power to power)
	\item The Products to Powers Rule- How to raise exponential products to a power.
\end{enumerate} 
We discuss each of these in turn.
\subsection{Multiplication and Exponents}
Consider the following:
\begin{itemize}
	\item $5+5+5=3(5)$ or 15 : repeated addition of the same number becomes multiplication
	\item $5\cdot 5\cdot 5=5^3$ or 125 : repeated multiplication of the same number becomes exponentiation It is important
to realize that $3(5)\ne 5^3$
\end{itemize}
In general
\begin{itemize}
	\item $x+x+x=3x$: repeated addition of the same number becomes multiplication.
	\item $x\cdot x\cdot x=x^3$: repeated multiplication of the same number becomes exponentiation.
\end{itemize}

It is important to realize that $3x\ne x^3$ in general.

\begin{myexample}
Exponents are used to indicate repeated multiplication. Fill in the blanks, then check your answers:
\begin{enumerate}
	\item In $2^4$, $2$ is the \underline{\phantom{base}} and 4 is the \underline{\phantom{exponent}} which indicates 2
	occurs as a \underline{\phantom{\gls{factor}}} 4 times.
	\item 
	$\begin{aligned}[t]
	2^4 &= 2\cdot 2\cdot 2\cdot 2 \\
	& = \underline{\phantom{16}}
	\end{aligned}$
\end{enumerate} 
\end{myexample}
\begin{myProof}
	\begin{enumerate}
		\item $2$ is the {\color{red}{base}} and 4 is the {\color{red}{exponent}} which indicates 2
		occurs as a {\color{red}{factor}} 4 times.
		\item 
		$\begin{aligned}[t]
		2^4 & = 2\cdot 2\cdot 2\cdot 2 \\
		&= {\color{red}{{16}}}
		\end{aligned}$
	\end{enumerate} 
	{}
		
\end{myProof}	
\subsection{1st Rule for Exponents: The Product Rule}
This rule tells us how to multiply exponential expression. We will demonstrate
this by {example}.

Without evaluating, \gls{simplify} the expression $2^4*2^3$
\begin{align*}
	2^4\cdot 2^3 & =	(2\cdot 2\cdot 2\cdot 2)\cdot (2\cdot 2\cdot 2) \\
	             & =	2^7                                             
\end{align*} 
Check by evaluating the left side $2^4*2^3$ should have the same value as the right $2^7$:
\begin{tightcenter}
	$2^4=16 \qquad 2^3=8\qquad 2^4*2^3=16(8)=128$ and $2^7=128$
\end{tightcenter}
Note that in order to simplify exponential expression using the product rule the base must be the
same. For {example}, there is no simpler form for
\[
	3^2 4^5
\]
We could evaluate the expression, but there are only 2 factors of 3, and 5 factors of 4. It would make
no sense here to add (or multiply) the exponents (or the bases).
\begin{myDefinition}
	The Product Rule for exponents is defined as
	\[
		b^m\cdot b^n=b^{m+n}
	\]
	When multiplying exponential expression with the same base, add the exponents.
\end{myDefinition}
\begin{myexample}
Multiply each expression using the product rule:
\begin{multicols}{5}
	\begin{enumerate}
		\item $3^2 3^5$
		\item $n^7 n^4$
		\item $x x^8$
		\item $y(y^8)(y^4)$
		\item $x^2 x^8 y^3 y^4$
	\end{enumerate} 
\end{multicols}
\end{myexample}
\begin{myProof}
	\begin{enumerate}
		\item 
		$\begin{aligned}[t]
		3^2 3^5&=3^{2+5}\\
		&=3^7
		\end{aligned}$ 
		
		The bases are the same, so we can add the exponents.
		\item $n^7 n^4=n^{11}$ The bases are the same, so we can add the exponents.
		\item 
		$\begin{aligned}[t]
		x \cdot x^8&=x^1\cdot x^8\\
		& = x^9
		\end{aligned}$ 
		
		The bases are the same; remember that $x$ is the same as $x^1$
		\item 
		$\begin{aligned}[t]
		y(y^8)(y^4)&=y^{1+8+4}\\
		&=y^{13}
		\end{aligned}$
		
		The bases are the same, so we can add the exponents.
		\item 
		$\begin{aligned}[t]
		x^2 x^8 y^3 y^4&=x^{2+8}y^{3+4}\\
		&=x^{10}y^7
		\end{aligned}$
		
		Note that we can only add powers of the same base.
	\end{enumerate} 
	\mbox{}
\end{myProof} 
Beware- it is very easy to  confuse multiplication with addition; consider the following example.
\begin{myexample}
Compare 
\begin{enumerate}
	\item $2^4\cdot 2^3$ with $2^4 + 2^3$
	\item $x^4 \cdot x^3$ with $x^4+x^3$
	\item $x^4 \cdot x^4$ with $x^4+x^4$
\end{enumerate} 
{}
\end{myexample}
\begin{myProof}
	\begin{enumerate}
		\item $2^4\cdot  2^3=2^7$ but $2^4 + 2^3$ has no simpler form, except to evaluate.
		\item $x^4 \cdot  x^3=x^7$ but $x^4+x^3$ has no simpler form- they are not like terms!
		\item $x^4 \cdot  x^4=x^8$ but $x^4+x^4=2x^4$. These are like terms
	\end{enumerate} 
\end{myProof} 

\subsection{2nd Rule for Exponents- The Power Rule}
This rule enables us to raise an exponential expression to a power (power to a power).

Consider the expression
\[
	(3^2)^5
\]
Without doing any actual arithmetic, what does it mean? The expression inside the parenthesis
will be raised to the $5^{th}$ power. That means $3^2$ will be used as a factor 5 times
\begin{align*}
	(3^2)^5 & =		3^23^23^23^23^2                                                  \\
	        & =		3(3)\cdot 3(3)\cdot 3(3)\cdot 3(3)\cdot 3(3)                     \\
	        & =		3\cdot 3\cdot 3\cdot 3\cdot 3\cdot 3\cdot 3\cdot 3\cdot 3\cdot 3 \\
	        & =		3^{10}                                                           
\end{align*} 
Or
\begin{align*}
	(3^2)^5 & =		3^2\cdot 3^2\cdot 3^2\cdot 3^2\cdot 3^2 \\
	        & =		3^{2+2+2+2+2}                           \\
	        & =		3^{10}                                  
\end{align*}
To summarize, we have found that
\[
	(3^2)^5 = 3^{10}
\]
This leads us to the power rule.
\begin{myDefinition}
	The Power Rule: Let $b$ be any real number, with $m$ and $n$ integers (in future Math classes
	you will consider the case when $m$, $n$ are any real numbers), then
	\[
		(b^m)^n = b^{mn}
	\]
	When an exponential expression is raised to a power, we multiply the exponents. Place
	the product of the exponents on the base and remove the parenthesis.
\end{myDefinition}
\begin{myexample}
Multiply each expression using the power rule
\begin{multicols}{4}
	\begin{enumerate}
		\item $(3^2)^5$
		\item $(n^7)^4$
		\item $(x^4)^4$
		\item $(x^2)^8(y^3)^4$
	\end{enumerate}
\end{multicols}
\end{myexample}
\begin{myProof}
	\begin{enumerate}
		\item $(3^2)^5=3^{2(5)}=3^{10}$  \hfill When raising a power to a power, multiply the exponents
		\item $(n^7)^4 = n^{7(4)} = n^{28}$
		\item $(x^4)^4 = x^{16}$
		\item Remember that the bases need to be the same in order for simplification to occur.
		\begin{align*}
			(x^2)^8 (y^3)^4 & =   x^{2(8)} y^{3(4)} \\
			                & =   x^{16} y^{12}     
		\end{align*} 
	\end{enumerate}
\end{myProof}
\subsection{3rd Rule for Exponents- Products to Powers Rule}
Suppose we have a product that is raised to a power such as
\[
	(3d)^4
\]
Hopefully you realize that the product $3d$ will be used as a factor 4 times:
\begin{align*}
	(3d)^4 & =		(3d)(3d)(3d)(3d) \\
	       & =		3(3)(3)(3)dddd   \\
	       & =		3^4d^4           
\end{align*}
We see that $(3d)^4 = 3^4 d^4$. 
\begin{itemize}
	\item It is important to realize that the product $3d$ is grouped inside 
	parenthesis, which is why it appears as a factor $4$ times. If there were
	no parenthesis, $3d^4$ would mean something different, i.e $3d^4 = 3 dddd$. Notice
	that the $3$ is not repeated as a factor $4$ times- only the $d$ is.
	\item This is the same principal for $(-5)^2$ versus $-5^2$. In $(-5)^2$ the $-5$ is 
	a grouping, and so $-5$ will appear as a factor 2 times
	\[
		(-5)^2 = (-5)(-5 = 25)
	\]
	In $-5^2$ the $-5$ is not grouped, and so only $5$ will appear as a factor 2 times
	\[
		-5^2 = -5(5) = -25
	\]
\end{itemize}
\begin{myDefinition}
	Products to Powers Rule: Let $a$ and $b$ be any real number, and $n$ any integer,
	then
	\[
		(ab)^n = a^n b^n
	\]
	when a product is raised to a power, raise each factor to the power.
\end{myDefinition}
\begin{myexample}
Demonstrate the products to powers rule with
\begin{multicols}{2}
	\begin{enumerate}
		\item $(2 \cdot 3)^5$
		\item $(2\cdot 3^2)^5$
	\end{enumerate}	
\end{multicols}
\end{myexample}
\begin{myProof}
	\begin{enumerate}
		\item Using the 3rd law of exponents, we have
		\begin{align*}
			(2\cdot 3)^5 & =		2^5\cdot 3^5 \\
			             & =		32 \cdot 243 \\
			             & =		7776         
		\end{align*}	
		You should also verify on your calculator that this is equal to $6^5$.
		\item We will need to use the both the 3rd law and the 2nd law of exponents here
		\begin{align*}
			(2 \cdot 3^2)^5 & =		2^5(3^2)^5 \\	
			                & =		2^5 3^{10} 
		\end{align*}
		We have simplified this expression, and we could evaluate it further by evaluating $2^5$
		and multiplying it by $3^{10}$.
	\end{enumerate}
\end{myProof}
\begin{myexample}
Simplify each expression using the products-to-powers rule:
\begin{multicols}{2}
	\begin{enumerate}
		\item $(-2n)^4$
		\item $(-4x^2)^3$
	\end{enumerate}
\end{multicols}
\end{myexample}
\begin{myProof}
	\begin{enumerate}
		\item Remember that since the entire group is being raised to the 4th power, the 
		factor of $-2$ appears as a factor 4 times. 
		\begin{align*}
			(-2n)^4 & =		(-2)^4(n^4) \\
			        & =		16 n^4      
		\end{align*}
		That is why it must still be inside parenthesis after applying the products to powers rule. 
		Since there is an even number of negative factors, $(-2)^4$ becomes 16.
		\item Remember that since the entire group is being raised to the 3rd power, the factor
		of $-4$ appears as a factor 3 times. 
		\begin{align*}
			(-4x^2)^3 & =		(-4)^3 (x^2)^3 \\
			          & =		-64 x^6        
		\end{align*}
		Since there are an odd number of negative
		factors, $(-4)^3$ becomes $-64$.
	\end{enumerate}	
	Notice that $(-4)^3=-4^3 = -64$. When the exponent is odd, the parenthesis are not necessary
	around a single term. However, when the exponent is even, the parenthesis are necessary; for
	example, $-2^4 = -16$ and $(-2)^4 = 16$.
\end{myProof}

\section{Adding and Subtracting Polynomials}
In this section we will discuss adding and subtracting polynomial. This is essentially
an exercise in combining like terms. There are two main formats for adding and subtracting
polynomial: vertical format, and horizontal format. 

We will demonstrate each technique with examples.
\begin{myexample}\label{ex:addpolyhoriz}
Add the polynomial
\begin{align*}
	7y^3+5y-1 & \qquad and \quad 2y^2-6y+3 
\end{align*}
using horizontal format.
\end{myexample}
\begin{myProof}
	We arrange the terms with highest power of $y$ at the front of the expression
	\begin{align*}
		(7y^3+5y-1) + (2y^2-6y+3) & =		7y^3+2y^2+5y-6y-1+3 \\
		                          & =		7y^3+2y^2-y+2       
	\end{align*}
	Note that the grouping brackets are not strictly necessary as we are {\em adding} the 2
	polynomial, but if the problem involved subtraction then they would most certainly be so 
	(see \cref{ex:subpoly1,ex:subpoly2}).
\end{myProof}

\begin{myexample}\label{ex:polyvert}
Repeat \cref{ex:addpolyhoriz} using vertical format.
\end{myexample}
\begin{myProof}
	The main \gls{point} when using this technique is to line up like terms, leaving space for any terms
	that have a \gls{coefficient} of $0$.
	\begin{equation*}
		\begin{array}{ccccc}
			  & 7y^3 &       & +5y & -1 \\
			  &      & 2y^2  & -6y & +3 \\\hline
			= & 7y^3 & +2y^2 & -y  & +2 
		\end{array}
	\end{equation*}
\end{myProof}

\begin{myexample}\label{ex:subpoly1}
Subtract the polynomial
\[
	y^2-8y+9 \text{ from } 6y^3+2y^2-y-11 
\]
using horizontal format.
\end{myexample}
\begin{myProof}
	We are very careful to include grouping brackets around the 2nd polynomial
	\begin{align*}
		(6y^3+2y^2-y-11) - (y^2-8y+9) & =		6y^3+2y^2-y-11 {\color{red}-}y^2 {\color{red}+} 8y {\color{red}-} 9 \\
		                              & =		6y^3+2y^2-y-11-y^2+8y-9                                             \\
		                              & =		6y^3+2y^2-y^2-y+8y-9-11                                             \\
		                              & =		6y^3+y^2+7y-20                                                      
	\end{align*}
	You may find it helpful to rewrite the problem as
	\[
		(6y^3+2y^2-y-11) - {\color{red}1}(y^2-8y+9)
	\]
\end{myProof}

\begin{myexample}\label{ex:subpoly2}
Repeat \cref{ex:subpoly1} using vertical format.
\end{myexample}
\begin{myProof}
	We line up terms as we did in \cref{ex:polyvert}
	\begin{equation*}
		\begin{array}{crrrr}
			  & 6y^3 & +2y^2 & -y  & -11 \\
			- &      & (y^2  & -8y & +9) 
		\end{array}
	\end{equation*}
	We now need to change the problem to an addition problem. This requires
	changing all the terms of the subtrahend (the 2nd polynomial) to their opposites. Then combine
	all like terms, using the rules for addition
	\begin{equation*}
		\begin{array}{crrrr}
			  & 6y^3 & +2y^2 & -y  & -11 \\
			+ &      & (-y^2 & +8y & -9) \\ \hline
			  & 6y^3 & +y^2  & +7y & -20 
		\end{array}
	\end{equation*}
\end{myProof}

\begin{myexample}
Simplify each of the following.
\end{myexample}
\begin{myProof}
	\drillandskill
	{\em Addition practise}:
	\begin{enumerate}
		\item $3x+1+(4x-1)$ \solution{$=7x$}
		\item $(3x+1)+(4x-1)$ \solution{$=7x$}
		\item $-2x-1+(x^2-3x+2)$ \solution{$=x^2-5x+1$}
		\item $x^2+3x+1+(2x^2-4x-5)$ \solution{$=3x^2-x-4$}
		\item $-3y^4+y^3-2y^2+1+ (y^5-y^4+y^2-3)$ \solution{$=y^5-4y^4+y^3-y^2-2$}
	\end{enumerate}
	\solution{\em Subtraction practise}:
	\begin{enumerate}
		\item $2x-1 - (4x+3)$ \solution{$=-2x-4$}
		\item $-x+10 - (2x+3)$ \solution{$=-3x+7$}
		\item $x^2+2x-3 - (3x^2+4x+1)$ \solution{$=-2x^2-2x-4$}
		\item $x^2+2x-3 - (-3x^2-4x-1)$ \solution{$=4x^2+6x-2$}
		\item $y^3+y^2+y - (3y^3-4y^2+y-1)$ \solution{$=-2y^3+5y^2+1$}
	\end{enumerate}
	{\em Harder addition}:
	\begin{enumerate}
		\item $3x+1+2(4x-1)$ \solution{$=11x-1$}
		\item $(3x+1)+3(4x-1)$ \solution{$=15x-2$}
		\item $-2x-1+2(x^2-3x+2)$ \solution{$=2x^2-8x+3$}
		\item $x^2+3x+1+4(2x^2-4x-5)$ \solution{$=9x^2-13x-19$}
		\item $-3y^4+y^3-2y^2+1+ 3(y^5-y^4+y^2-3)$ \solution{$=3y^5-6y^4+y^3+y^2-8$}
	\end{enumerate}
\end{myProof} 

\section{Multiplying Polynomials}
\textref{6.2}{350}%
There are 3 basic rules for exponents that we have discussed so far in this module that will 
helps us to multiply polynomial
\begin{enumerate}
	\item The Product Rule- how to multiply exponential expression
	\item The Power Rule- how to raise exponential expression to a power (power to a power)
	\item The Products to Powers Rule- how to raise exponential products to a power.
\end{enumerate}
We will demonstrate the techniques with a series of examples
\begin{enumerate}
	\item Multiply monomial
	\item Multiply a monomial and a polynomial with more than 1 term
	\item Multiply polynomial with more than 1 term
\end{enumerate}

\subsection{Multiplying Monomials}
To multiply monomial, we need to make use of the commutative and associative property of 
multiplication. 

We multiply two monomial (which are just polynomial with one term), by multiplying the coefficients
together, and then the algebraic terms. For example
\begin{align*}
	(-3x)(5x^2) & =		(-3)(5)x\cdot x^2 \\
	            & =		-15 x^3           
\end{align*}
Notice here that we have used the commutative property of multiplication here to switch the 
order of multiplication.

Notice also that this is clearly very different from the addition
\[
	(-3x)+(5x^2)=-3x+5x^2
\] 
which are not like terms and \emph{can not} be simplified.

\begin{myDefinition}
	To multiply monomial, multiply the coefficients and then multiply the variables. Use the product
	rule for exponents to multiply the variables.
\end{myDefinition}

\subsection{Multiplying a Monomial and a Polynomial with more than 1 term}
We use the distributive property to multiply a monomial and a polynomial that has more than one term.

\begin{myexample}
Multiply
\[
	{\color{red}3x^2}(2x^3+5x)
\]
\end{myexample}
\begin{myProof}
	Notice that the 1st factor is a monomial (a polynomial with one term), while the 2nd
	term is a binomial (a polynomial with two terms)
	\begin{align*}
		{\color{red}3x^2}(2x^3+5x) & =		{\color{red}3x^2}\cdot 2x^3+{\color{red}3x^2}\cdot 5x                                             \\
		                           & =		{\color{red}3}\cdot 2 \cdot {\color{red}x^2}x^3 + {\color{red}3}\cdot 5 \cdot {\color{red}x^2}x^1 \\
		                           & =		6x^5+15x^3                                                                                        
	\end{align*}
\end{myProof}

\subsection{Multiplying Polynomials which both have more than 1 term}
Multiplication over a group of terms introduces another level of difficulty because we need to 
remember to distribute.

\begin{myexample}
Multiply the following polynomial
\[
	(2x+3)(x^2+4x-5)
\]
\end{myexample}
\begin{myProof}
	We have a group of terms multiplying a group of terms. There are several ways to do this problem
	\begin{align*}
		(2x+3)(x^2+4x-5) & =		(2x+3)x^2+(2x+3)(4x)+(2x+3)(-5)        \\
		                 & =		(2x)x^2+3x^2+2x(4x)+3(4x)+2x(-5)+3(-5) \\
		                 & =		2x^3+3x^2+8x^2+12x-10x-15              \\
		                 & =		2x^3+11x^2+2x-15                       
	\end{align*}
	That's it. All the terms have different exponents, and we have written it with the highest
	power of $x$ at the front.
		
	We could also have multiplied the $2x$ throughout the trinomial, and then the $3$
	\[
		2x(x^2+4x-5)+3(x^2+4x-5)
	\]
	From here we can distribute each term, and collect like terms (exercise).
\end{myProof}

\begin{myexample}
Simplify the following and consider the differences between them
\begin{enumerate}
	\item $3x^3y-(4x^3y)$
	\item $3x^3y(-4x^3y)$
	\item $3x^3y(-4x^3+y)$
	\item $3x^3+y(-4x^3+y)$
	\item $(3x^3+y)(-4x^3+y)$
\end{enumerate}
\end{myexample}
\begin{myProof}
	\begin{itemize}
		\item The essential difference between a and b is the placement of the parenthesis. The first
		one is a subtraction problem, while the 2nd problem is a multiplication problem.
		\item The 3rd is a multiplication problem, and it is monomial times a binomial. We will need to distribute the monomial to the 2 terms inside the parenthesis.
		\item The 4th is an addition problem. In the 2nd term, the $y$ can be distributed to the
		2 terms inside the parenthesis. 
		\item The 5th problem is a multiplication problem. it is a binomial times a binomial. As we will soon see, the first
		group needs to be distributed to each term in the second group. Before we can correctly 
		simplify each of them, we need to be able to discern the differences.
	\end{itemize}
		
	\begin{enumerate}
		\item $3x^3y-(4x^3y) = -1x^3y = -x^3y$
		\item $3x^3y(-4x^3y) = -12x^6y^2$
		\item $3x^3y(-4x^3+y) = -12x^6y+3x^3y^2$
		\item $3x^3+y(-4x^3+y) = 3x^3-4x^3y+y^2$
		\item \label{item:distribute}
		$\begin{aligned}[t]
			{\color{red}(3x^3+y)}(-4x^3+y) & =	{\color{red}(3x^3+y)}(-4x^3)+{\color{red}(3x^3+y)}(y) \\
			                               & =	-12x^6-4x^3y+3x^3y+y^2                                \\
			                               & =	-12x^6-x^3y+y^2                                       
		\end{aligned}$
	\end{enumerate}
	Note that in \ref{item:distribute} we had to apply the distributive property twice. Each term in the first factor
	had to be multiplied by each term in the second factor. Or, as we saw above we could 
	have distributed the $3x^3$ to both terms, then the $y$ to both terms.
	
\end{myProof}

\begin{myexample}
Simplify the following.
\drillandskill
	
monomial with a monomial:
\begin{enumerate}
	\item $(x)(5x)$ \solution{$=5x^2$}
	\item $(2x)(6x)$ \solution{$=12x^2$}
	\item $(2x)(-6x^3)$ \solution{$=-12x^4$}
	\item $(-3x^{34})(x)$ \solution{$=-3x^{35}$}
\end{enumerate}
	
monomial with a binomial:
\begin{enumerate}
	\item $3x(x+1)$ \solution{$=3x^2+3x$}
	\item $-4x(2x^2+5x)$ \solution{$=-8x^3-20x^2$}
	\item $-4x(2x^2-5x)$ \solution{$=-8x^3+20x^2$}
	\item $x^2(-5x^3-2x)$ \solution{$=-5x^5-2x^3$}
\end{enumerate}
	
monomial with a trinomial:
\begin{enumerate}
	\item $4x(x^2+3x-1)$ \solution{$=4x^3+12x^2-4x$}
	\item $-6x(2x^3-x^2+1)$ \solution{$=-12x^4+6x^3-6x$}
\end{enumerate}
	
Binomial with a binomial:
\begin{enumerate}
	\item $(x+1)(x-2)$ \solution{$=x^2-x-2$}
	\item $(x-1)(x+3)$ \solution{$=x^2+2x-3$}
	\item $(x-4)(x+5)$ \solution{$=x^2+x-20$}
	\item $(2x-6)(x+2)$\solution{$=2x^2-2x-12$}
\end{enumerate}	
\end{myexample}



% Mainfile:
%%%***++ mainfile.tex
% !arara: pdflatex: { files: [ mainfile.tex ] }
% arara: makechapters: { files:[mainfile], items: [module3], makeChapGlossaries: no}
% !arara: indent: { overwrite: on, trace: yes}
\chapter{Multiplying Polynomials}
\minitoc
\section{Special Binomial Products}
\textref{6.3}{360}%
In the previous module we examined some elementary \gls{polynomial} multiplication. We begin
this module by considering the following polynomial multiplications, all of which will
be done using the \gls{FOIL} method. In this section, we will:
\begin{enumerate}
	\item Multiply the sum and difference of 2 terms
	\item Find the square of a \gls{binomial} sum.
	\item Find the square of a binomial difference
\end{enumerate} 

\subsection{The FOIL method}
We will often be finding the product of 2 binomials, such as
\[
	(x-2)(x+3)
\]
so it will be useful to develop a solid place from which to start. There is a well known
acronym that can help us, which is {\bfseries FOIL}. We describe it as follows
\begin{description}
	\item Multiply the  {\color{red}F}irst terms from each binomial, then 
	\item	Multiply the  {\color{green}O}utside terms, then                
	\item	Multiply the  {\color{brown}I}nside terms, and finally          
	\item	Multiply the  {\color{blue}L}ast terms                          
\end{description}
The FOIL method distributes the 1st term, and then distributes the 2nd. In the above example the 
components of the FOIL are
\begin{itemize}
	\item {\color{red}F}: the {\em first} terms multiplied together are $x\cdot x = x^2$
	\item {\color{green}O}: the {\em outside} terms multiplied together are $x\cdot 3 = 3x$
	\item {\color{brown}I}: the {\em inside} terms are $-2\cdot x = -2x$
	\item {\color{blue}L}: the {\em last} terms are $-2(3)=-6$
\end{itemize} 
We combine all of this information as follows
\begin{align*}
	(x-2)(x+3) & =		x^2+3x-2x-6 \\
	           & =		x^2+x-6     
\end{align*} 	

\begin{myexample}
Use the FOIL method to expand the following polynomial 
\[
	(3x+2)(4x+5)
\]
\end{myexample}
\begin{myProof}
	We colour code the polynomial as follows
	\begin{tightcenter}
		\begin{tabular}{lll}
			$({\color{red}3x}+2)({\color{red}4x}+2)$     & First terms {\color{red}F}   & ${\color{red}(3x)(4x)=12x^2}$ \\
			$({\color{green}3x}+2)(4x+{\color{green}5})$ & Outer terms {\color{green}O} & ${\color{green}(3x)(5)=15x}$  \\	
			$(3x+{\color{brown}2})({\color{brown}4x}+5)$ & Inner terms {\color{brown}I} & ${\color{brown}(2)(4x)=8x}$   \\	
			$(3x+{\color{blue}2})(4x+{\color{blue}{5}})$ & Last terms {\color{blue}L}   & ${\color{blue}(2)(5)=10}$     
		\end{tabular} 
	\end{tightcenter}
	More simply, we can write
	\begin{align*}
		(3x+2)(4x+5) & =		{\color{red}12x^2}+{\color{green}15x}+{\color{brown}{8x}}+{\color{blue}{10}} \\
		             & =		12x^2+23x+10                                                                 
	\end{align*} 
\end{myProof} 

\begin{myexample}
\Gls{simplify} the following.
\drillandskill
\end{myexample}
\begin{enumerate}
	\item $(x+1)(x+2)$\solution{$=x^2+3x+2$}
	\item $(x-1)(x-4)$\solution{$=x^2-5x+4$}
	\item $(2x+1)(x-5)$\solution{$=2x^2-9x-5$}
	\item $(3x-4)(x+6)$\solution{$=3x^2+14x-24$}
\end{enumerate}

\subsection{Multiplying the Sum and Difference of 2 terms}
When using the FOIL method, there are a number of special products that arise, 
for example
\[
	(a-b)(a+b)
\]
When using the FOIL method on a product of this form, which is the product of the sum of two terms with the product
of the difference of two terms, there is a useful simplification
\begin{align*}
	(a-b)(a+b) & =		a^2+ab-ba-b^2 \\
	           & =		a^2-b^2       
\end{align*} 
Notice here that the `middle' terms (the {\color{green}O}utside and the {\color{brown}I}nside terms) cancel each other out. This will 
be particularly useful in later modules when we are asked to \gls{factor} a polynomial, which involves `going the other way' from FOILing. 

\begin{myexample}\label{ex:sumdiff}
Use the FOIL method to expand the following \gls{expression}
\[
	(x-3)(x+3)
\]
\end{myexample}
\begin{myProof}
	We note first of all that this is the product of the sum and difference of two terms, so the middle terms will cancel. We will show the
	working for demonstration
	\begin{align*}
		(x-3)(x+3) & =		x^2+3x-3x-9 \\
		           & =		x^2-9       
	\end{align*} 
\end{myProof} 

\begin{myexample}
Simplify the following.
\drillandskill
\end{myexample}
\begin{enumerate}
	\item $(x+10)(x-10)$\solution{$=x^2-100$}
	\item $(x+3)(x-3)$\solution{$=x^2-9$}
	\item $(x-4)(x+4)$\solution{$=x^2-16$}
	\item $(x^4-7)(x^4+7)$\solution{$=x^8-49$}
	\item $(3x+1)(3x-1)$\solution{$=9x^2-1$}
	\item $(2x+4)(2x-4)$\solution{$=4x^2-16$}
\end{enumerate}

\subsection{The Square of a Binomial Sum}\label{sec:binsum}
Another special case of a binomial \footnote{ Remember that a {\em binomial} is simply another way of saying a polynomial with 2 terms.} 
multiplication is the square of a binomial sum, for example
\[
	(x+y)^2
\] 
Using the FOIL method on this expression, we achieve
\begin{align*}
	(x+y)^2 & =		x^2+xy+xy+y^2 \\
	        & =		x^2+2xy+y^2   
\end{align*} 
In words: The square of a binomial sum is the first term squared, plus two times the product of the terms, plus the last term squared.

\begin{myexample}
Use the FOIL method to expand the following polynomial 
\[
	(2x+5)^2
\]
\end{myexample}
\begin{myProof}
	We apply the FOIL method and obtain
	\begin{align*}
		(2x+5)^2 & =		(2x)^2+(2x)(5)+(2x)(5)+5^2 \\
		         & =		4x^2+10x+10x+25            \\
		         & =		4x^2+20x+25                
	\end{align*} 
\end{myProof} 

\begin{myexample}
Simplify the following.
\drillandskill
\end{myexample}
\begin{enumerate}
	\item $(x+1)^2$\solution{$=x^2+2x+1$} 
	\item $(x+6)^2$\solution{$=x^2+12x+36$}
	\item $(x+10)^2$\solution{$=x^2+20x+100$}
	\item $(x+7)^2$\solution{$=x^2+14x+49$}
	\item $(x^2+9)^2$\solution{$=x^4+18x^2+81$}
	\item $(2x+4)^2$\solution{$=4x^2+16x+16$}
\end{enumerate}

\subsection{The Square of a Binomial Difference}\label{sec:bindiff}
The final special case of a binomial multiplication is the square of a binomial difference, for example
\[
	(x-2)^2
\]
Using the FOIL method gives
\begin{align*}
	(x-2)^2 & =		x^2-2x-2x+4 \\
	        & =		x^2-4x+4    
\end{align*} 
In general
\[
	(x-y)^2 = x^2-2xy+y^2
\]
\begin{myexample}
Use the FOIL method to expand the following polynomial 
\[
	(4x-6)^2
\]
{}
\end{myexample}
\begin{myProof}
	On applying the FOIL procedure, we obtain
	\begin{align*}
		(4x-6)^2 & =		(4x)^2+(4x)(-6)+(4x)(-6)+(-6)^2 \\
		         & =		16x^2 - 24x - 24x +36           \\
		         & =		16x^2-48x+36                    
	\end{align*} 
\end{myProof}

\begin{myexample}
Simplify the following.
\drillandskill
\end{myexample}
\begin{enumerate}
	\item $(x-1)^2$\solution{$=x^2-2x+1$}
	\item $(x-6)^2$\solution{$=x^2-12x+36$}
	\item $(x-10)^2$\solution{$=x^2-20x+100$}
	\item $(x-7)^2$\solution{$=x^2-14x+49$}
	\item $(x^2-9)^2$\solution{$=x^4-18x^2+81$}
	\item $(2x-4)^2$\solution{$=4x^2-16x+16$}
\end{enumerate}

\subsection{Miscellaneous FOIL examples}
\begin{myexample}
Find an expression for each of the following shaded areas (diagrams not to scale).
\end{myexample}
\begin{minipage}[t]{.5\textwidth}
	\centering
	\begin{tikzpicture}
		\draw[fill=red] (0,0) rectangle (5,3);
		\draw (-.5,1.5) node[rotate=90] {$4x+1$};
		\draw (2.5,-.5) node {$x-5$};
	\end{tikzpicture}
	\begin{align*}
		area & =  (4x+1)(x-5) \\
		     & = 4x^2-19x-5   
	\end{align*} 
\end{minipage}
\begin{minipage}[t]{.5\textwidth}
	\centering
	\begin{tikzpicture}
		\draw[fill=blue] (0,0) rectangle (5,3);
		\draw (-.5,1.5) node[rotate=90] {$4x+1$};
		\draw (2.5,-.5) node {$x-5$};
		\filldraw [white,draw=black] (3,1.5) circle (1);
		\draw (3,1.5)--(4,1.5) node (myradius){};
		\node[node distance=0cm,above left = of myradius] {$x$};
	\end{tikzpicture}
	\begin{align*}
		area & =  (4x+1)(x-5)-\pi x^2 \\
		     & = 4x^2-19x-5 - \pi x^2 \\
		     & = (4-\pi)x^2-19x-5     
	\end{align*} 
\end{minipage}

\begin{minipage}[t]{.5\textwidth}
	\centering
	\begin{tikzpicture}
		\draw[fill=yellow] (0,0) rectangle (5,3);
		\draw[fill=white] (1,1) rectangle (2,2);
		\draw (-.5,1.5) node[rotate=90] {$4x+1$};
		\draw (2.5,-.5) node {$x-5$};
		\draw (1.5,.75) node {$x+2$};
		\draw (.75,1.5) node[rotate=90] {$x+2$};
	\end{tikzpicture}
	\begin{align*}
		area & =  (4x+1)(x-5)-(x+2)(x+2) \\
		     & = 4x^2-19x-5 - (x^2+4x+4) \\
		     & = 3x^2-23x-9              
	\end{align*} 
\end{minipage}
\begin{minipage}[t]{.5\textwidth}
	\centering
	\begin{tikzpicture}
		\draw[fill=green] (0,0) rectangle (5,3);
		\draw (-.5,1.5) node[rotate=90] {$4x+1$};
		\draw (2.5,-.5) node {$x-5$};
		\draw[fill=white] (0,1)--(3,1)--(1.5,3)--cycle;
		\draw[dashed] (1.5,1)--(1.5,3);
		\draw (1.25,1.75) node[rotate=90] {$2x$};
		\draw (1.5,0.75) node {$x-3$};
	\end{tikzpicture}
	\begin{align*}
		area & =  (4x+1)(x-5)- \frac{1}{2}(2x)(x-3) \\
		     & = 4x^2-19x-5 - (x^2-3x)              \\
		     & = 3x^2-16x-5                         
	\end{align*} 
\end{minipage}

\begin{myexample}
Here are some miscellaneous short cut FOIL examples.
\drillandskill
\end{myexample}
\begin{myProof}
	Can you spot  which short cut to use? If not, don't worry, just FOIL!
				
	\begin{multicols}{2}
		\begin{enumerate}
			\item $(x+1)^2$\solution{$=x^2+2x+1$}
			\item $(x+2)^2$\solution{$=x^2+4x+4$}
			\item $(x+5)^2$\solution{$=x^2+10x+25$}
			\item $(x-1)^2$\solution{$=x^2-2x+1$}
			\item $(2x+1)^2$\solution{$=4x^2+4x+1$}
			\item $(2x-1)^2$\solution{$=4x^2-4x+1$}
			\item $(x-3)^2$\solution{$=x^2-6x+9$}
			\item $(2x+4)^2$\solution{$=4x^2+16x+16$}
			\item $(3x+2)^2$\solution{$=9x^2+12x+4$}
			\item $(2-x)^2$\solution{$=x^2-4x+4$}
			\item $(3-x)^2$\solution{$=x^2-6x+9$}
			\item $(2+x)^2$\solution{$=x^2+4x+4$}
			\item $(2-3x)^2$\solution{$=9x^2-12x+4$}
			\item $(2x+5)^2$\solution{$=4x^2+20x+25$}
			\item $(x-1)(x+1)$\solution{$=x^2-1$}
			\item $(x-2)(x+2)$\solution{$=x^2-4$}
			\item $(x+3)(x-3)$\solution{$=x^2-9$}
			\item $(2x-1)(2x+1)$\solution{$=4x^2-1$}
			\item $(1-x)(1+x)$\solution{$=1-x^2$}
			\item $(x+2)(x-2)$\solution{$=x^2-4$}
			\item $(x-5)(x+5)$\solution{$=x^2-25$}
			\item $(3x+3)(3x-3)$\solution{$=9x^2-9$}
		\end{enumerate}
	\end{multicols}
\end{myProof} 

\subsection{The cube of a binomial sum}
We can find the \emph{cube} of a binomial sum by using the FOIL technique combined 
with polynomial multiplication.
\begin{myexample}
Expand the following polynomial by using polynomial multiplication
\[
	(x+y)^3
\]
\end{myexample}
\begin{myProof}
$
\begin{aligned}[t]
(x+y)^3 & = (x+y)(x+y)^2\\
& = (x+y)(x^2+2xy+y^2)\\
& = x^3+2x^2y+xy^2 + yx^2 + 2xy^2 + y^3\\
& = x^3 + 3x^2y + 3 xy^2 + y^3
\end{aligned}
$
\end{myProof}
\begin{myexample}
Here are some miscellaneous short-cut cube examples.
\drillandskill
\end{myexample}
\begin{myProof}				
		\begin{enumerate}
			\item $(x+1)^3$\solution{$=x^3+3x^2+3x+1$}
			\item $(x+2)^3$\solution{$=x^3+6x^2+12x+8$}
			\item $(2x+3)^3$\solution{$=8x^3+36x^2+54x+27$}
			\item $(3x+2y)^3$\solution{$=27x^3+54x^2y+36x*y^2+8y^3$}
		\end{enumerate}
\end{myProof} 

\subsection{The cube of a binomial difference}
We can find the \emph{cube} of a binomial difference by using the FOIL technique combined 
with polynomial multiplication- this is very similar to the technique used for the 
cube of binomial sum.
\begin{myexample}
Expand the following polynomial by using polynomial multiplication
\[
	(x-y)^3
\]
\end{myexample}
\begin{myProof}
$
\begin{aligned}[t]
(x-y)^3 & = (x-y)(x-y)^2\\
& = (x-y)(x^2-2xy+y^2)\\
& = x^3-2x^2y+xy^2 - yx^2 + 2xy^2 - y^3\\
& = x^3 -3x^2y + 3 xy^2 - y^3
\end{aligned}
$
\end{myProof}
\begin{myexample}
Here are some miscellaneous short-cut cube examples.
\drillandskill
\end{myexample}
\begin{myProof}				
		\begin{enumerate}
			\item $(x-1)^3$\solution{$=x^3-3x^2+3x-1$}
			\item $(x+2)^3$\solution{$=x^3-6x^2+12x-8$}
			\item $(2x+3)^3$\solution{$=8x^3-36x^2+54x-27$}
			\item $(3x+2y)^3$\solution{$=27x^3-54x^2y+36x*y^2-8y^3$}
		\end{enumerate}
\end{myProof} 




\subsection{An application}
The following is an application of our work in polynomials.
\begin{myexample}\label{ex:fallingobject}
Consider the \gls{equation} of motion for an object tossed straight up into the air (projected vertically), 
or dropped (falling)
\begin{equation}\label{eq:objectfalling}
	s = -\frac{1}{2}gt^2+v_0t+s_0
\end{equation}
Note the following:
\begin{itemize}
	\item $s$ is vertical position (height above ground) of the tossed object in feet (ft)
	\item $g$ is the acceleration due to gravity (and we will assume $g=32$)
	\item $t$ is time in seconds that the object has been in motion (seconds)
	\item $v_0$ is the initial speed (or velocity) of the object in ft/sec
	\item $s_0$ is the initial position at time $t=0$
\end{itemize} 
Find the height of ball after
\begin{multicols}{3}
	\begin{enumerate}
		\item $2$ seconds
		\item $4$ seconds
		\item $6$ seconds
	\end{enumerate} 
\end{multicols} 
\end{myexample}
\begin{myProof}
	Before we begin this example, we first need to rewrite \cref{eq:objectfalling} using
	the given information. It becomes
	\[
		s = -16t^2+80t+96
	\]
	To find the height of the object at each of the specified times, we will simply replace $t$ by 2, 4, and 6 seconds in turn.
	\begin{enumerate}
		\item After 2 seconds, the height of the object (in feet) is
		\[
			-16(2^2)+80(-2)+96 = 192
		\]
		\item After 4 seconds, the height of the object (in feet) is
		\[
			-16(4^2)+80(4)+96=160
		\]
		\item After 6 seconds, the height of the object (in feet) is
		\[
			-16(6^2)+80(6)+96=0
		\]
	\end{enumerate} 
	We clearly see that as time increase, the height of the object is decreasing. In fact, after
	6 seconds the ball is at a height of 0ft above the ground- in other words it has hit the ground
	at this time.
				
	\Cref{fig:fallingobject2} shows a visual representation of the height of the object thrown in 
	the example considered so far.
				
	\begin{figure}[!h]
		\centering
		\begin{tikzpicture}
			\begin{axis}[
					framed,
					axis line style={->},
					xmin=-1,xmax=10,
					ymin=-20,ymax=200,
					xlabel={$t$},
					ylabel={$s$},
					xtick={0,2,...,8},
					ytick={0,20,...,180},
					grid=major,
					scatter/classes={a={mark=*,draw=violet,fill=violet,scale=1},%
						b={mark=*,draw=red,fill=red,scale=1},
						c={mark=*,draw=black,fill=black,scale=1},
						d={scale=0}}
				]
				\addplot[red]expression[domain=0:6,samples=100]{-16*x^2+80*x+96};
			\end{axis}
		\end{tikzpicture}
		\caption{A falling object}
		\label{fig:fallingobject2}
	\end{figure}
	\FloatBarrier
				
	The horizontal axis represents the object's time in seconds. The vertical axis represents
	the object's height above the ground.
	\begin{itemize}
		\item During which time period is the object increasing in height? \\
		We see that the height is increasing from $t=0$ to approximately
		$t=2.5$ seconds.
		\item During which time period is the object decreasing in height?\\
		We see that the height is decreasing from approximately $t=2.5$ seconds
		to $t=6$ seconds.
		\item After how much time does the object strike the ground?\\
		The object strikes the ground when the graph cuts the horizontal
		axis, which happens when $t=6$.
		\item After how many seconds does the object reach its maximum height? Give an
		approximation of this height.\\
		The maximum value of the height (which we will later discuss as the {\em \gls{vertex}} of
		the graph) occurs when $t \approx 2.5$, and looking at the vertical scale, we see
		that the maximum height is approximately 195ft.
	\end{itemize} 
\end{myProof}

\section{Polynomials in several variables}
\textref{6.4}{368}%
We have so far seen some elementary polynomial operations, using the FOIL 
method to expand special products involving binomials. In this section we will build 
on this, and discuss polynomials of more than one \gls{variable}.

\subsection{Evaluate polynomials in several variables}
We will introduce this in terms of a real world example.

\begin{myexample}\label{ex:storageshed}
The storage shed given below has a volume of
\[
	V= 2x^2y+ \frac{1}{2}\pi x^2 y
\]
\begin{figure}[!h]
	\centering
	\begin{tikzpicture}
		% foreground stuff
		\draw (0,1)--(0,0)--(2,0)--(4,1);
		\draw (2,1)--(4,2);
		\draw (2,0)--(2,1);
		\draw (4,1)--(4,2);
		\draw (2,1) arc (0:180:1);
		\draw (4,2) arc (0:90:1);
		% background stuff
		\draw[dashed] (2,2) arc (180:90:1);
		\draw[dashed] (0,1)--(2,2)--(4,2);
		\draw[dashed] (0,0)--(2,1)--(4,1);
		\draw[dashed] (2,1)--(2,2);
		\draw[dashed] (1,2)--(3,3);
		\draw[dashed] (0,1)--(2,1);
		\draw[dotted] (1,2)--(1,0);
		% labels
		\node at (1,-.5) {$2x$};
		\node at (-.5,.5) {$x$};
		\node at (3,.25) {$y$};
	\end{tikzpicture}
	\caption{A storage shed}
\end{figure}
\FloatBarrier
			
The small business that owns this storage shed tells you that the variables have the following 
values
\[
	x = 13, y = 27
\]
where both measurements are given in ft. 
			
Find the volume of the storage shed that has these
dimensions.
\end{myexample}
\begin{myProof}
	To find the required volume, we use our original expression for $V$, and replace each occurrence of $x$
	with 13, and each occurrence of $y$ with 27. This gives
	\begin{align*}
		V & =		2(13)^2\cdot 27 + \frac{1}{2}\pi (13)^2	\cdot 27 \\
		  & =		2(169)(27)+\frac{1}{2}\pi(169)(27)               \\
		  & =		9126+2281.5\pi                                   
	\end{align*} Note that this answer is {\em exact}, as we have not had to approximate any of the values
	involved. For practical purposes, we may need an approximation, so we use the $\pi$ key on our calculators,
	and obtain
	\[
		V \approx 16293.54 ft^3
	\]
	{\bfseries Important note}: Notice how we have used the $\approx$ symbol to represent
	`approximately equal to'; it is very important to distinguish between this and the regular $=$ symbol.
				
\end{myProof} 

The steps that we followed in \cref{ex:storageshed} can be summarized as follows:
\begin{itemize}
	\item Substitute the given value for each variable
	\item Perform the resulting computation, being careful to use the order of operations - notice how we evaluated
	the term with the exponent before we did any further multiplication.
\end{itemize} 

\subsection{The vocabulary of polynomials in two variables}
This carries across very naturally from our work in one variable polynomials. In general, a polynomial in two variables
$x$ and $y$

\begin{itemize}
	\item contains the sum of one or more monomials which have the form
	\[
		a x^n y^m
	\]
	\item here $a$ is the \gls{coefficient}
	\item and the \gls{degree} of this term is $n+m$
\end{itemize} 

\begin{myexample}
Find the degree of the following polynomials
\begin{multicols}{2}
	\begin{enumerate}
		\item $x^2y^3$
		\item $xy^2z^3$
	\end{enumerate} 
\end{multicols}
\end{myexample}
\begin{myProof}
	\begin{enumerate}
		\item The power of $x$ is 2, and power of $y$ is 3, so the degree of this term is $2+3=5$
		\item The power of $x$ is 1, the power of $y$ is 2, and the power of $z$ is 3, so the degree of this term
		is $1+2+3=6$
	\end{enumerate}	
\end{myProof} 


\subsection{Adding and subtract polynomials in several variables}
Adding and subtracting polynomials in several variables applies the principles that we learnt for adding and subtracting those
with just one variable
\begin{itemize}
	\item {\em Addition}: combine like terms
	\item {\em Subtraction}: distribute subtraction signs appropriately, and then add
\end{itemize} 

\subsection{Multiplying monomials (in $x$ and $y$)}
To multiply monomials with more than one variable we multiply the coefficients, and add the exponents on variables that have the same base.

\begin{myexample}
Multiply the following
\[
	(7x^2y)(5x^3y^2)
\]
\end{myexample}
\begin{myProof}
	As described above, we multiply the 7 and the 5, and combine the $x$ and $y$ terms using the properties
	of exponents discussed previously.
	\begin{align*}
		(7x^2y)(5x^3y^2) & =		35x^2x^3yy^2 \\
		                 & =		35x^5y^3     
	\end{align*} 
\end{myProof} 

\subsection{Multiplying a \gls{monomial} with a polynomial}
This follows the same method as that of the one variable case: we distribute the monomial through the polynomial, 
and then multiply as detailed in the previous example.

\begin{myexample}
Simplify the following expression
\[
	3x^2y(2x^3+5x+1)
\]
\end{myexample}
\begin{myProof}
	We follow the procedure outlined above
	\begin{align*}
		3x^2y(2x^3+5x+1) & =		3x^2y(2x^3)+3x^2y(5x)+3x^2y(1) \\
		                 & =		3(2)x^2x^3y+3(5)x^2xy+3x^2y    \\
		                 & =		6x^5y+15x^3y+3x^2y             
	\end{align*} 
\end{myProof}

\begin{myexample}
Simplify the following.
\drillandskill
\end{myexample}

\begin{enumerate}
	\item $2y^2(x^2y^3+xy+x)$\solution{$=2x^2y^5+2xy^3+2xy^2$}
	\item $-3xy(x^2+y^2+xy)$\solution{$=-3x^3y-3xy^3-3x^2y^2$}
	\item $-4x(x+y)(x-y)$\solution{$=-4x^3+4xy^2$}
	\item $x^2y^5(x^2+y^2)$\solution{$=x^4y^5+x^2y^7$}
\end{enumerate}


\subsection{Multiplying binomials with binomials }
We use the same FOIL procedure that we have used previously.

\begin{myexample}
Multiply the following
\begin{multicols}{2}
	\begin{enumerate}
		\item $(5x-3y)(5x+3y)$
		\item $(6x+4y)^2$
		\item $(9x-3y)^2$
		\item $(a^2+3b)(5a^2-4b)$
	\end{enumerate} 
\end{multicols}
\end{myexample}

\begin{myProof}
	\begin{enumerate}
		\item Notice that this is the multiplication of the sum and difference of 2 terms (see \cref{ex:sumdiff})
		\begin{align*}
			(5x-3y)(5x+3y) & =		(5x)^2-(3y)^2 \\
			               & =		25x^2-9y^2    
		\end{align*}		 
		\item Notice that this is the square of a binomial sum (see \cref{sec:binsum})
		\begin{align*}
			(6x+4y)^2 & =		(6x)^2+2(6)(4)xy+(4y)^2 \\
			          & =		36x^2+48xy+16y^2        
		\end{align*} 
		\item Notice that this is the square of a binomial difference (see \cref{sec:bindiff})
		\begin{align*}
			(9x-3y)^2 & =		(9x)^2-2(9)(3)xy+(3y)^2 \\
			          & =		81x^2-54xy+9y^2         
		\end{align*} 
		\item This does not fall into one of the special cases that we have discussed, so we simply FOIL it
		\begin{align*}
			(a^2+3b)(5a^2-4b) & =		5a^4-4a^2b+15ba^2-12b^2 \\
			                  & =		5a^4+11a^2b-12b^2       
		\end{align*} 
	\end{enumerate} 
\end{myProof} 

\begin{myexample}
Simplify the following.
\drillandskill
\end{myexample}
\begin{enumerate}
	\item $(x+y)^2$\solution{$=x^2+2xy+y^2$}
	\item $(x-y)^2$\solution{$=x^2-2xy+y^2$}
	\item $(x-y)(x+y)$\solution{$=x^2-y^2$}
	\item $(x+y)(x-y)$\solution{$=x^2-y^2$}
\end{enumerate}

\begin{myexample}
Find an expression for each of the following shaded areas (diagrams not to scale).
\end{myexample}

\begin{minipage}[t]{.5\textwidth}
	\centering
	\begin{tikzpicture}
		\draw[fill=red] (0,0) rectangle (5,3);
		\draw (-.5,1.5) node[rotate=90] {$4x+1$};
		\draw (2.5,-.5) node {$x-5$};
	\end{tikzpicture}
	{$area=4x^2-19x-5$}
\end{minipage}
\begin{minipage}[t]{.5\textwidth}
	\centering
	\begin{tikzpicture}
		\draw[fill=blue] (0,0) rectangle (5,3);
		\draw (-.5,1.5) node[rotate=90] {$4x+1$};
		\draw (2.5,-.5) node {$x-5$};
		\filldraw [white,draw=black] (3,1.5) circle (1);
		\draw (3,1.5)--(4,1.5) node (myradius){};
		\node[node distance=0cm,above left = of myradius] {$y$};
	\end{tikzpicture}
	{$area=4x^2-\pi y^2-19x-5$}
\end{minipage}

\begin{minipage}[t]{.5\textwidth}
	\centering
	\begin{tikzpicture}
		\draw[fill=yellow] (0,0) rectangle (5,3);
		\draw[fill=white] (1,1) rectangle (2,2);
		\draw (-.5,1.5) node[rotate=90] {$4y+1$};
		\draw (2.5,-.5) node {$y-5$};
		\draw (1.5,.75) node {$x+2$};
		\draw (.75,1.5) node[rotate=90] {$x+2$};
	\end{tikzpicture}
	{$area=4y^2-x^2-19y-4x-9$}
\end{minipage}
\begin{minipage}[t]{.5\textwidth}
	\centering
	\begin{tikzpicture}
		\draw[fill=green] (0,0) rectangle (5,3);
		\draw (-.5,1.5) node[rotate=90] {$4x+1$};
		\draw (2.5,-.5) node {$x-5$};
		\draw[fill=white] (0,1)--(3,1)--(1.5,3)--cycle;
		\draw[dashed] (1.5,1)--(1.5,3);
		\draw (1.25,1.75) node[rotate=90] {$2y$};
		\draw (1.5,0.75) node {$y-3$};
	\end{tikzpicture}
	{$area=4x^2-y^2-19x+3y-5$}
\end{minipage}

\begin{minipage}[t]{.5\textwidth}
	\centering
	\begin{tikzpicture}
		\draw[fill=brown] (0,0) rectangle (5,3);
		\draw (-.5,1.5) node[rotate=90] {$4x+1$};
		\draw (2.5,-.5) node {$x-5$};
	\end{tikzpicture}
	{$area=3xy-15y+x-5$}
\end{minipage}
\begin{minipage}[t]{.5\textwidth}
	\centering
	\begin{tikzpicture}
		\draw[fill=purple] (0,0) rectangle (5,3);
		\draw (-.5,1.5) node[rotate=90] {$4x+1$};
		\draw (2.5,-.5) node {$x-5$};
		\filldraw [white,draw=black] (3,1.5) circle (1);
		\draw (3,1.5)--(4,1.5) node (myradius){};
		\node[node distance=0cm,above left = of myradius] {$y$};
	\end{tikzpicture}
	{$area=4xy-\pi x^2-20x+y-5$}
\end{minipage}

%+*** mainfile.tex
% !arara: pdflatex: { files: [ mainfile.tex ] }
% arara: makechapters: { files:[mainfile], items: [module4], makeChapGlossaries: no}
% !arara: indent: { overwrite: on, trace: yes, localSettings: on}

\chapter{Quotient rules}
\minitoc
We have previously considered \gls{polynomial} multiplication, and evaluation. In particular,
we have studied the \gls{FOIL} method for multiplying two binomials, and we used it when 
dealing with polynomials in one and more than \gls{variable}. In this module we will study
how to divide polynomials.


\section{The quotient rule for exponents}
\textref{6.5}{376}%
Consider the quotient of two exponential expressions, such as 
\begin{align*}
	\frac{2^7}{2^3} & =	 \frac{2\cdot 2\cdot 2\cdot 2\cdot 2\cdot 2\cdot 2 }{2\cdot 2\cdot  2} \\
	                & =	 2 \cdot 2\cdot 2 \cdot 2                                              \\
	                & =	 2^4                                                                   
\end{align*} 
We have therefore shown that
\[
	\frac{2^7}{2^3} = 2^4
\]
We can find the exponent, $4$, on the quotient by {\em subtracting} the original exponents. Remember
that when multiplying two exponential expressions with the same base, we added the exponents.

\begin{myDefinition}
	Given any real number $b\ne 0$, and any integers $m$, $n$, we can state the quotient rule as
	\[
		\frac{b^m}{b^n} = b^{m-n}
	\]
	When dividing exponential expressions with the same nonzero base, subtract the
	exponent in the denominator from the exponent in the numerator. Use this
	difference as the exponent of the common base.
\end{myDefinition}

\begin{myexample}
Divide each \gls{expression} using the quotient rule, expressing any numerical answers in exponential form.
\begin{multicols}{3}
	\begin{enumerate}
		\item $\dd\frac{5^{20}}{5^{15}}$
		\item $\dd\frac{y^{15}}{y^3}$
		\item $\dd\frac{x^{200}y^{40}}{x^{25}y^{10}}$
	\end{enumerate} 
\end{multicols}
\end{myexample}
\begin{myProof}
	\begin{enumerate}
		\item 
		$\begin{aligned}[t]
			\frac{5^{20}}{5^{15}} & =		5^{20-15} \\
			                      & =		5^5       
		\end{aligned} $
		\item 
		$\begin{aligned}[t]
			\frac{y^{15}}{y^3} & =		y^{15-3} \\
			                   & =		y^{12}   
		\end{aligned}$ 
		\item Remember to analyze which terms can be simplified with one another
		\begin{align*}
			\frac{x^{200}y^{40}}{x^{25}y^{10}} & =		x^{200-25}y^{40-10} \\
			                                   & =		x^{175}y^{30}       
		\end{align*} 
	\end{enumerate} 
\end{myProof} 

\begin{myexample}
\Gls{simplify} the following.
\drillandskill
\end{myexample}
\begin{multicols}{4}
	\begin{enumerate}
		\item $\dd\frac{2^8}{2^3}$\solution{$=2^5$}
		\item $\dd\frac{x^4}{x^2}$\solution{$=x^2$}
		\item $\dd\frac{x^9}{x^3}$\solution{$=x^6$}
		\item $\dd\frac{y^{10}}{y^{2}}$\solution{$=y^8$}
		\item $\dd \frac{4x^5}{2x^3}$\solution{$=2x^2$}
		\item $\dd \frac{10y^7}{2y^2}$\solution{$=5y^5$}
		\item $\dd -\frac{40x^9}{10x^2}$\solution{$=-4x^7$}
		\item $\dd \frac{50x^2}{5x}$\solution{$=10x$}
	\end{enumerate}
\end{multicols}

Note that we can only cancel {\em factors} of the numerator and denominator. Consider the following example
\begin{align*}
	\frac{2^3+2^4}{2^3} & =	 \frac{2\cdot 2\cdot 2 + 2\cdot2\cdot2\cdot2}{2\cdot 2\cdot 2} \\
	                    & \ne	 2\cdot 2\cdot 2\cdot 2                                      
\end{align*} 
We will discuss later how to divide expressions that involve more than one term in either the numerator
or denominator. For the moment, our expressions will involve only factors.

\section{The zero exponent}
What is the meaning of zero as an exponent? We certainly know that the following is true
\[
	\frac{b^4}{b^4} = 1
\]
We also know that from the quotient rule
\begin{align*}
	\frac{b^4}{b^4} & =		b^{4-4} \\
	                & =		b^0     
\end{align*} 
These two expressions {\em must} be equivalent, and we therefore conclude
that
\[\
	b^0=1
\]
for any real number $b$.
\begin{myexample}
Simplify the following
\drillandskill
\end{myexample}
\begin{multicols}{4}
	\begin{enumerate}
		\item $x^0$\solution{$=1$}
		\item $73^0$\solution{$=1$}
		\item $-(2^2)^0$\solution{$=-1$}
		\item $(-2)^0$\solution{$=1$}
		\item $-x^0$\solution{$=-1$}
		\item $-3(2^0)$\solution{$=-3$}
		\item $-4(-2^0)$\solution{$=4$}
		\item $-5(-2^2)^0$\solution{$=-5$}
	\end{enumerate}
\end{multicols}

\section{The Quotients to powers rule}
Recall that when a (grouped) product is raised to a power, we raise every \gls{factor} in the product to the power
\[
	(ab)^n = a^n b^n
\]
Note that without the parenthesis the product $ab^n$  means something different, as only $b$ is raised to the
$n$th power.

We can apply this principle to quotients raised to a power. If $a$ and $b$ are real numbers
and $b$ is non zero, then
\[
	\left( \frac{a}{b}\right)^n = \frac{a^n}{b^n}
\]
When a (grouped) quotient is raised to a power, we raise the numerator to the power and 
divide by the denominator raised to the same power.

\begin{myexample}
Simplify the following
\begin{multicols}{2}
	\begin{enumerate}
		\item $\dd\left(\frac{1}{2}\right)^3$
		\item $\dd\left(\frac{x}{y}\right)^5$
	\end{enumerate} 
\end{multicols}
\end{myexample}
\begin{myProof}
	\begin{enumerate}
		\item 
		$\begin{aligned}[t]
			\left(\frac{1}{2}\right)^3 & =		\frac{1^3}{2^3} \\
			                           & =		\frac{1}{2^3}   \\
			                           & =		\frac{1}{8}     
		\end{aligned}	$
		\item 
		$\begin{aligned}[t]
			\left(\frac{x}{y}\right)^5 & =		\frac{x^5}{y^5} \\
		\end{aligned}$
	\end{enumerate} 
\end{myProof} 
Note that without the parenthesis, the quotient $\dd\frac{a^n}{b}$ means something different to $\dd\left(\frac{a}{b}\right)^n$. In
$\dd\frac{a^n}{b}$, it is only $a$ that is raised to the $n$th power.

\begin{myexample}
Simplify the following
\drillandskill
\end{myexample}
\begin{multicols}{2}
	\begin{enumerate}
		\item $\dd\left(\frac{x}{2}\right)^3$\solution{$\dd=\frac{x^3}{8}$}
		\item $\dd\left(\frac{x^3}{2}\right)^3$\solution{$\dd=\frac{x^9}{8}$}
		\item $\dd\left(-\frac{x}{5}\right)^2$\solution{$\dd=\frac{x^2}{25}$}
		\item $\dd\left(-\frac{2x}{3}\right)^2$\solution{$\dd=\frac{4x^2}{9}$}
		\item $\dd\left(-\frac{2x}{3}\right)^3$\solution{$\dd=-\frac{8x^3}{27}$}
	\end{enumerate}
\end{multicols}

\section{Dividing by monomials}
Understanding and remembering how to divide polynomials heavily depends on your ability to understand
the difference between adding and multiplying fractions.

For example, consider the following; the first is a fraction multiplication, 
and the second is a fraction addition
\begin{multicols}{2}
	\begin{itemize}
		\item $\dd\frac{2}{5}\cdot \frac{7}{5}= \frac{2\cdot 7}{5\cdot 5}$
		\item $\dd\frac{2}{5}+\frac{7}{5} = \frac{2+7}{5}$
	\end{itemize} 
\end{multicols}
Comparing these with the following examples, we see why we must distribute $4x^2$ in the 2nd example, but
not the 1st:
\begin{itemize}
	\item $\dd\frac{-12x^8}{4x^2}$ is really $\dd\frac{-12}{4}\cdot \frac{x^8}{x^2}$
	\item $\dd\frac{-12+x^8}{4x^2}$ is really $\dd\frac{-12}{4x^2}+\frac{x^8}{4x^2}$
\end{itemize} 

\begin{myexample}
Simplify the following
\begin{multicols}{2}
	\begin{enumerate}
		\item $\dd\frac{-12x^8}{4x^2}$
		\item $\dd\frac{-12+x^8}{4x^2}$
	\end{enumerate} 
\end{multicols}
\end{myexample}
\begin{myProof}
	\begin{enumerate}
		\item  
		$\begin{aligned}[t]
			\frac{-12x^8}{4x^2} & =		\frac{-12}{4}\cdot\frac{x^8}{x^2} \\
			                    & =		-3x^{8-2}                         \\
			                    & =		-3x^6                             
		\end{aligned}$
		\item 
		$\begin{aligned}[t]
			\frac{-12+x^8}{4x^2} & =		\frac{-12}{4x^2}+\frac{x^8}{4x^2}                               \\
			                     & =		\frac{-12}{4}\cdot\frac{1}{x^2}+\frac{1}{4}\cdot\frac{x^8}{x^2} \\
			                     & =		-3 \frac{1}{x^2} + \frac{1}{4}x^{8-2}                           \\
			                     & =		\frac{-3}{1}\cdot\frac{1}{x^2}+\frac{1}{4}\cdot\frac{x^6}{1}    \\
			                     & =		-\frac{3}{x^2}+\frac{x^6}{4}                                    
		\end{aligned}$
	\end{enumerate} 
\end{myProof} 

To summarize: 
\begin{itemize}
	\item To divide monomials, divide the coefficients and then divide the variables. Use the quotient
	rule for exponents to divide the variables and subtract the exponents
	\item To divide a polynomial by a \gls{monomial}, divide each term of the polynomial by the monomial
\end{itemize} 

\begin{myexample}
Simplify the following
\drillandskill
\end{myexample}

\begin{multicols}{2}
	\begin{enumerate}
		\item $\dd \frac{x^7+x^2}{x}$\solution{$=x^6+x$}
		\item $\dd \frac{3x^8+x^{10}}{x^4}$\solution{$=3x^4+x^6$}
		\item $\dd \frac{-3x^8+x^6+x^2}{x^2}$\solution{$=-3x^6+x^4+1$}
		\item $\dd \frac{x^9+x^2+3x}{x}$\solution{$=x^8+x+3$}
	\end{enumerate}
\end{multicols}

\section{Negative exponents and scientific notation}
\textref{6.7}{393}%
In this section we will look
at {\em negative exponents} and {\em scientific notation}.

\subsection{Negative exponents}
We define negative exponents as
\[
	a^{-n} = \frac{1}{a^n}
\]	
where (for this class) $n$ is an integer.

\begin{myexample}\label{ex:negexponnumeric}
Write the following without any negative exponents
\begin{multicols}{5}
	\begin{enumerate}
		\item $2^{-1}$
		\item $2^{-2}$
		\item $3^{-3}$
		\item $3^{-4}$
		\item $\left(\frac{1}{3}\right)^{-2}$
	\end{enumerate} 
\end{multicols}
\end{myexample}
\begin{myProof}
	We use the above definition to re-write these expressions
	\begin{enumerate}
		\item 
		$\begin{aligned}[t]
			2^{-1} & =  \frac{1}{2^1} \\
			       & =  \frac{1}{2}   
		\end{aligned}$
		\item 
		$\begin{aligned}[t]
			2^{-2} & =  \frac{1}{2^2} \\
			       & =  \frac{1}{4}   
		\end{aligned}$
		\item 
		$\begin{aligned}[t]
			3^{-3} & =  \frac{1}{3^3} \\
			       & =  \frac{1}{27}  
		\end{aligned}$
		\item 
		$\begin{aligned}[t]
			3^{-4} & =  \frac{1}{3^4} \\
			       & =  \frac{1}{81}  
		\end{aligned}$ 
		\item 
		$\begin{aligned}[t]
			\left(\frac{1}{3}\right)^{-2} & =  \frac{1}{\left(\frac{1}{3}\right)^2} \\
			                              & =  \frac{1}{\frac{1}{9}}                \\
			                              & =  9                                    \\
			                              & =  \left(\frac{3}{1}\right)^2           
		\end{aligned}$ 
	\end{enumerate} 
	Notice in the last part that we {\em flip the fraction} when raising a fraction to a negative exponent.
\end{myProof}

\begin{myexample}\label{ex:negexponnumericds}
Write each of the following using only positive exponents.
\drillandskill
		
\end{myexample}

\begin{multicols}{2}
	\begin{enumerate}
		\item $4^{-1}$ \solution{$\dd=\frac{1}{4}$}
		\item $6^{-2}$\solution{$\dd=\frac{1}{36}$}
		\item $7^{-4}$\solution{$\dd=\frac{1}{7^4}$}
		\item $9^{-15}$\solution{$\dd=\frac{1}{9^{15}}$}\\
		\item $\dd\left(\frac{1}{4}\right)^{-2}$\solution{$\dd=16$}
		\item $\dd\left(\frac{2}{3}\right)^{-2}$\solution{$\dd=\frac{9}{4}$}
		\item $\dd\left(\frac{2}{3}\right)^{-3}$\solution{$\dd=\frac{27}{8}$}
		\item $\dd\left(\frac{4}{5}\right)^{-2}$\solution{$\dd=\frac{25}{16}$}
	\end{enumerate}
\end{multicols}

{\em More advanced}
\begin{multicols}{2}
	\begin{enumerate}
		\item $\left(-\frac{1}{4}\right)^{-2}$\solution{$\dd=16$}
		\item $\left(-\frac{2}{3}\right)^{-2}$\solution{$\dd=\frac{9}{4}$}
		\item $\left(-\frac{2}{3}\right)^{-3}$\solution{$\dd=-\frac{27}{8}$}
		\item $\left(-\frac{4}{5}\right)^{-2}$\solution{$\dd=\frac{25}{16}$}
		\item $(x^3)^{-5}$\solution{$\dd=\frac{1}{x^{15}}$}
		\item $(x^{-5})^3$\solution{$\dd=\frac{1}{x^{15}}$}
		\item $(-x^2)^{-4}$\solution{$\dd=\frac{1}{x^8}$}
		\item $(-x^2)^{-3}$\solution{$\dd=-\frac{1}{x^6}$}
	\end{enumerate}
\end{multicols}

\begin{myexample}
Use the techniques from \cref{ex:negexponnumeric,ex:negexponnumericds} to simplify the following expressions.
\begin{multicols}{2}
\begin{enumerate}
	\item $\left(\dfrac{2}{y^3}\right)^{-4}$
    \item $\left( \dfrac{s^3}{t^5} \right)^{-6}$
\end{enumerate}
\end{multicols}
\end{myexample}
\begin{myProof}
  \begin{enumerate}
    \item 
      $\begin{aligned}[t]
	\left(\dfrac{2}{y^3}\right)^{-4} & = \left(\dfrac{y^3}{2}\right)^{4} \\
    & = \dfrac{y^{3\cdot 4}}{2^4}\\
    &= \dfrac{y^{12}}{16}
      \end{aligned}$

      Alternatively, you might have preferred to write
      \begin{align*}
        \left(\frac{2}{y^3}\right)^{-4} & = \left(\frac{2^{-4}}{y^{3\cdot -4}}\right) \\
        & = \frac{\frac{1}{16}}{y^{-12}}\\
    &= \frac{y^{12}}{16}
      \end{align*}
    \item
      $
      \begin{aligned}[t]
    \left( \dfrac{s^3}{t^5} \right)^{-6} & =  \left( \dfrac{t^5}{s^3} \right)^{6}\\
    & = \dfrac{t^{5\cdot 6}}{s^{3\cdot 6}} \\
    & = \dfrac{t^{30}}{s^{18}}
      \end{aligned}
      $
  \end{enumerate}
\end{myProof}

\begin{myexample}
  Simplify each of the given expressions as much as possible.
\drillandskill
\begin{multicols}{2}
	\begin{enumerate}
		\item $\dd\left( \frac{x^2}{y^3}\right)^{-2}$\solution{$\dd=\frac{y^6}{x^4}$}
		\item $\dd\left( \frac{y^2}{y^3}\right)^{-2}$\solution{$\dd=y^2$}
		\item $\dd\left( \frac{x^2z^7}{y^3}\right)^{-2}$\solution{$\dd=\frac{y^6}{x^4z^{14}}$}
		\item $\dd\left( \frac{-4x^3}{y^3}\right)^{-2}$\solution{$\dd=\frac{y^6}{16x^6}$}
	\end{enumerate}
\end{multicols}
\end{myexample}

\subsection{Scientific notation for numbers with absolute value $>10$}
We will now use our knowledge of exponents, both positive and negative, to write numbers that are very small or
very large in a neat and useful way. 

Consider, for example, the distance between the Sun and Pluto, which is
5,906,000,000km. This is a huge distance \footnote{{Of course, astronomically speaking the Sun is considered quite close to us.}}, and if 
we wanted to reference it regularly then it would be very cumbersome to write 5,906,000,000km every single time. Scientific notation is a short
hand way of writing powers of 10, and we will use it to simplify large numbers such as the one just considered.

We begin with the following 
\begin{align*}
	10    & =  1 \times 10   \\
	100   & =  1 \times 10^2 \\
	1000  & =  1 \times 10^3 \\
	10000 & =  1 \times 10^4 \\
\end{align*}
Notice here that the number of 0s {\em after} the number 10 on the left hand side is equal to the power of 10 on 
the right hand side.

The following guidelines may help you in converting numbers into scientific notation for numbers whose absolute
values is {\em greater than or equal to 10}.
\begin{itemize}
	\item To write a number whose absolute value is 10 or greater in scientific notation, write a product of
	a numerical factor and an exponential expression.
	\item The numerical factor is determined by moving the decimal \gls{point} to the left so that the resulting
	number is between 1 and 10, including 1.
	\item The exponential expression consists of the base 10 and a positive exponent that is the number of places the
	decimal was moved.
\end{itemize} 

\begin{myexample}
Write the following numbers in scientific notation
\begin{multicols}{2}
	\begin{enumerate}
		\item 5,906,000,000
		\item 32,100,000
	\end{enumerate} 
\end{multicols}
\end{myexample}
\begin{myProof}
	\begin{enumerate}
		\item The absolute value of this number is clearly greater than 10. We write this in scientific notation by
		moving the decimal point from the right until we have a number between 1 and 10 (in this case 5). We 
		then count the number of places that we had to move the decimal, and this will be the power of 10. So, 
		\[
			5,906,000,000 = 5.906 \times 10^9
		\]
		When you try this type of calculation in your calculator, you may find that it displays the result as
		\[
			5.906 E 9
		\]	
		You can read the `E' as `times 10 to the power of'. 
		\item Proceeding as in part a), 
		\begin{align*}
			32,100,000 & =  3.21\times 10^7 \\
			           & =  3.21 E7         
		\end{align*} 
	\end{enumerate} 
	{}
\end{myProof}

\subsection{Scientific notation for numbers with absolute value $<10$}
In a similar way, we have
\begin{align*}
	0.1    & =  1 \times 10^{-1} \\
	0.01   & =  1 \times 10^{-2} \\
	0.001  & =  1 \times 10^{-3} \\
	0.0001 & =  1 \times 10^{-4} \\
\end{align*} 
Notice here that the power of 10 decreases as the number decreases in magnitude. 

\begin{itemize}
	\item To write a number whose absolute value is less than 1 in scientific notation, write a product of 
	a numerical factor and an exponential expression.
	\item The numerical factor is determined by moving the decimal point to the right so that the resulting
	number is between 1 and 10, including 1.
	\item The exponential expression consists of the base 10 and a negative exponent that has an absolute
	value of the number of places the decimal was moved.
\end{itemize} 

\begin{myexample}
Write the following numbers in scientific notation
\begin{multicols}{2}
	\begin{enumerate}
		\item 0.00012
		\item -0.005
	\end{enumerate} 
\end{multicols}
\end{myexample}
\begin{myProof}
	\begin{enumerate}
		\item
          $\begin{aligned}[t]
			0.00012 & =  1.2\times 10^{-4} \\
			        & =  1.2 E -4          
                  \end{aligned}$
		\item
          $\begin{aligned}[t]
			-0.005 & =  -5 \times 10^{-3} \\
			       & =  -5E-3             
                 \end{aligned}$
	\end{enumerate} 
	{}
\end{myProof} 

\begin{myexample}
Simplify the following
\drillandskill
\end{myexample}

\begin{multicols}{2}
	\begin{enumerate}
		\item $32,400$\solution{$=3.24 \times 10^{4}$}
		\item $713$\solution{$=7.13\times 10^2$}
		\item $9876$\solution{$=9.876\times 10^3$}
		\item $12000$\solution{$=1.2\times 10^4$}
		\item $0.001$\solution{$=1\times 10^{-3}$}
		\item $0.0005$\solution{$=5\times 10^{-4}$}
		\item $-0.09$\solution{$=-9\times 10^{-2}$}
		\item $0.000012$\solution{$=1.2\times 10^{-5}$}
	\end{enumerate}
\end{multicols}

\subsection{Converting from scientific notation to standard notation}
Given a number in scientific notation, we first need to determine if the
exponent is negative or positive:
\begin{itemize}
	\item if it is positive then move the decimal point in the numerical factor to the right by the number of 
	places denoted by the exponent. For example
	\[
		3.00\times 10^8 = 300,000,000
	\]
	\item if is negative then move the decimal point in the numerical factor to the 
	left the number of places denoted by the exponent. For example
	\[
		7.4\times 10^{-5} = 0.000074
	\]
\end{itemize} 


% Mainfile:
%%%***++ mainfile.tex
% arara: makechapters: { files: [ mainfile], items: [module5] }
% !arara: indent: { overwrite: on, trace: yes}
\chapter{Scientific notation and factoring}
\minitoc

\section{Introduction to factoring}
\textref{7.1}{414}%
So far in this class we have studied polynomials; in particular we have looked at how to multiply
polynomials. Our goal for the remainder of this module will be to study how we can `go the other way'. The question we ask ourselves is: given a
\gls{polynomial}, can we find an equivalent \gls{expression} but in factored form? There are a few techniques that
will help us in this study, and we demonstrate by example.

\begin{myexample}\label{ex:motivationfacto}
Perform the following multiplication.
\begin{multicols}{2}
	\begin{enumerate}
		\item $3(x+1)$
		\item $x^2(x^3-4x+5)$
	\end{enumerate} 
\end{multicols}
\end{myexample}
\begin{myProof}
	This example is designed as a warm up- our goal in the content after 
	this example will be to `go the other way'.
	\begin{enumerate}
		\item $3(x+1) = 3x+3$
		\item $x^2(x^3-4x+5) = x^5-4x^3+5x^2$
	\end{enumerate} 
\end{myProof} 

\Cref{ex:motivationfacto} reminds us of the distributive property
of multiplication. Now consider the expression
\[
	4x+2
\]
The question we ask is: are there any common factors to both terms? The
answer is yes: $2$. This means that we can reverse the multiplication procedure
and write the expression as follows
\[
	4x+2 = 2(2x+1)
\]

The process that we have just demonstrated is a very simple example of factoring. The process
is further demonstrated in the next example.

\begin{myexample}
\Gls{factor} each of the given expressions.
\drillandskill
\end{myexample}

{\em Introductory}
\begin{multicols}{2}
	\begin{enumerate}
		\item $4x+16$\solution{$=4(x+4)$}
		\item $5x+30$\solution{$=5(x+6)$}
		\item $-2x+10$\solution{$=-2(x-5)$}
		\item $-2x-10$\solution{$=-2(x+5)$}
	\end{enumerate}
\end{multicols}
{\em More advanced}
\begin{multicols}{2}
	\begin{enumerate}
		\item $7x^2+49x$\solution{$=7x(x+7)$}
		\item $-7x^2+49x$\solution{$=-7x(x-7)$}
		\item $-7x^2-49x$\solution{$=-7x(x+7)$}
		\item $3x^5 + 15x^3$\solution{$=3x^3(x^2+5)$}
		\item $14w^8 - 7w^9$\solution{$=7w^8(2w-1)$}
		\item $-32y^8 - 64y^2$\solution{$=-32y^2(y^6+2)$}
	\end{enumerate}
\end{multicols}
{\em Further advanced}
\begin{multicols}{2}
	\begin{enumerate}
		\item $2x^2+4x+2$\solution{$=2(x^2+2x+1)$}
		\item $2x^2+4x-2$\solution{$=2(x^2+2x-1)$}
		\item $4x^3+4x+16$\solution{$=4(x^3+4x+4)$}
		\item $4x^3+4x^2+16x$\solution{$=4x(x^2+x+4)$}
		\item $5w^4 - 25w^2 - 5w$\solution{$=5w(w^3-5w-1)$}
		\item $5w^4 - 25w^2 + 50w$\solution{$=5w(w^3-5w+10)$}
		\item $6w^2 - 30w - 3$\solution{$=3(2w^2-10w-1)$}
		\item $-3x^2-9x+6$\solution{$=-3(x^2+3x-2)$}
	\end{enumerate}
\end{multicols}

{\em Examples with 2 variables}
\begin{multicols}{2}
	\begin{enumerate}
		\item $4xy-2x$\solution{$=2x(2y-1)$}
		\item $4xy-2y$\solution{$=2y(2x-1)$}
		\item $8x^2y-2x$\solution{$=2x(4xy-1)$}
		\item $8x^2y-2xy$\solution{$=2xy(4x-1)$}
		\item $16x^2y^2-2xy$\solution{$=2xy(8xy-1)$}
		\item $32x^2y^2-4x^2y$\solution{$=4x^2y(8y-1)$}
		\item $x^5y^2-2x^2y$\solution{$=x^2y(x^3y-2)$}
		\item $9xy^2-2x^3y$\solution{$=xy(9y-2x^2)$}
	\end{enumerate}
\end{multicols}

\section{Factoring by grouping}\label{sec:facbygroup}

We will now consider examples such as
\[
	x(x+7)+10(x+7)
\]
There are many approaches to factoring expressions like these, but we will generally keep the following guidelines in mind.
Put parentheses around factors with at least one common factor, usually pairs:
\begin{enumerate}
	\item Factor out the common factor from the first pair and rewrite
	\item Factor out the common factor from the next pair and rewrite
	\item At this stage we hope to see the same factors left in both parentheses- if not, then
	we might try to rearrange the original terms.
	\item The final step is to factor out the parenthetical terms (we will demonstrate these in our
	first example), leaving the other parenthetical term.
\end{enumerate} 

\begin{myexample}
Factor
\[
	x(x+7)+10(x+7)
\]
{}
\end{myexample}
\begin{myProof}
	In this case, the problem is at step 4. Do you see that the \gls{binomial} in each parentheses is the same? 
	So this gets `factored' out $(x+7)$ leaving $x()+10()$.
			
	Therefore the other binomial is $x+10$. The final answer is 
	\[
		(x+7)(x+10)
	\]
	or
	\[
		(x+10)(x+7)
	\]
	This can be checked by multiplying the two binomials using the method you prefer from
	the previous modules. 
\end{myProof} 

\begin{myexample}
Factor
\[
	x(y+9)-11(y+9)
\]
{}
\end{myexample}
\begin{myProof}
	Again this is at step 4, but with a different twist, and we will follow the same procedure as in 
	the previous example. A good first question is
	\begin{tightcenter}{\em Are the terms in the parentheses the same?}\end{tightcenter}
	The answer to this question is `yes', so we begin by factoring out $y+9$ from both terms leaving
	\[
		x()-11()
	\] 
	This makes the next factor $(x-11)$, and the answer is
	\[
		(y+9)(x-11)
	\]
	or
	\[
		(x-11)(y+9)
	\]
	because of the commutative property.
			
	Note that this can be checked by FOILing this expression, and making sure that it agrees with the original. Note that this can be checked by FOILing this expression, and making sure that it agrees with the original.
\end{myProof} 

\begin{myexample}
Factor
\[
	x^2+3x+5x+15
\]
\end{myexample}
\begin{myProof}
	This is the first example we have encountered where we need to do some manipulation before we can recognize that
	this can be broken into two factored terms. A natural question is: how do we decide which two terms to group together? A good first attempt (and it may often take more than one!)
	is to group the first two, and the second two. This therefore gives
	\[
		(x^2+3x)+(5x+15)
	\]
	Note that this would need considerably more care if the third `+' sign (between the $3x$ and the $5x$) was a `-'
	sign. 
			
	From here, we
	\begin{itemize}
		\item Factor the first group $x(x+3)$
		\item Factor the second group $5(x+3)$
	\end{itemize} 
	The terms in the parentheses are the same, so we can rewrite our expression as
	\[
		x(x+3)+5(x+3)
	\]
	from which we factor out the $(x+3)$ leaving $x()+5()$. So now the other binomial is
	$(x+5)$. The final answer is
	\[
		(x+3)(x+5)
	\]
\end{myProof} 

\begin{myexample}
Factor
\[
	x^3-3x^2+4x-12
\]
{}
\end{myexample}
\begin{myProof}
	We begin by inserting parentheses around the first 2 terms, and the last 2 terms (note that this may
	not always work- we might need to try another combination)
	\[
		(x^3-3x^2)+(4x-12)
	\]
	From here we
	\begin{itemize}
		\item Factor the first term: $x^2(x-3)$
		\item Factor the second term $4(x-3)$
	\end{itemize} 
	This terms in the parentheses are the same, so we can rewrite our expression as
	\[
		x^2(x-3)+4(x-3)
	\]
	As before, the parenthetical groups are the same, so we can factor out the $(x-3)$ term, 
	and our final answer is
	\[
		(x-3)(x^2+4)
	\]
\end{myProof} 

\begin{myexample}
Factor
\[
	x^3+6x^2-2x-12
\]
{}
\end{myexample}
\begin{myProof}
	The first differences we notice in this example as compared to those previous, are the
	`-' signs in front of the $-2x$ and $-12$. We can still proceed as before, but we must
	be careful to account for these. Before we start the factorization process, let us review 
	distributing a negative number.
			
	Consider
	\[
		-2(x+6)
	\]
	We distribute the $-2$ through the parenthesis and obtain
	\[
		-2x-12
	\]
	We can use this result with the expression in the above. We remove a factor of $x^2$ from the
	first two terms, and a factor of $-2$ from the second $2$ to give
	\[
		x^2(x+6)-2(x+6)
	\]
	Now we can factor as before, and our final result is
	\[
		(x^2-2)(x+6)
	\]
\end{myProof} 

\section{Factoring trinomials when $a=1$}
\textref{7.2}{422}%
In the previous section we saw how we could factor certain polynomials by grouping. In this section, we look at
trinomials that have a leading \gls{coefficient} of $1$. Remember that in our previous modules we used the \gls{FOIL} method
to expand binomial products. We considered examples such as
\begin{align*}
	(x+1)(x+3) & =		x^2+3x+x+3 \\	
	           & =		x^2+4x+3   
\end{align*}
The goal of this section is to develop the skills to go from the right hand side in the above (a \gls{trinomial}), to the left hand side-
in other words, to factor the trinomial.

There are a number of important observations that will help us to factor trinomials. We 
begin with a review of FOILing. 

\subsection{FOILing review}
We begin with a review of multiplying binomials using the FOIL method. Once we review
how the trinomial was put together it may be easier to see how to take it apart
\begin{center}
	\begin{tabular}{cccrrSc}
		\toprule
        Example & Binomials    & F     & O     & I     & {L}     & Trinomial   \\
		\midrule
		A       & $(x+2)(x+5)$ & $x^2$ & $5x$  & $2x$  & 10  & $x^2+7x+10$ \\
		B       & $(a-3)(a-2)$ & $a^2$ & $-2a$ & $-3a$ & 6   & $x^2-5a+6$  \\
		C       & $(c+4)(c-6)$ & $c^2$ & $-6c$ & $4c$  & -24 & $c^2-2c-24$ \\
		D       & $(m-3)(m+4)$ & $m^2$ & $4m$  & $-3m$ & -12 & $m^2+m-1$   \\
		\bottomrule
	\end{tabular}
\end{center}

Notice that the product of the first terms always yielded a squared term. Notice that the last
term of the trinomial is always the product of the two number. The middle term is found
by adding the O and I columns. 

Another helpful hint is to look at the signs of the terms for the trinomial compared to the signs
for the binomials. When the signs (connectors) between the terms is the same (as in examples A and B), 
the middle terms of the trinomial matches this same sign. In fact adding the middle terms always gives
the numerical value for the middle.

When the signs (connectors) between the terms are different, as in examples C and D, the middle
term could end up either sign. Looking at example C, we see that the middle term is negative
because the larges in absolute value $6$ was negative. The numerical value is the
difference of the two numbers. In example D, we see that the middle term was positive because
$4>3$, and 4 is a positive number. Using these sign rules will help you eliminate
many of the guesses you will need to make.

\begin{myDefinition} 
	\begin{itemize}
		\item if the numbers are the same sign then the middle term is the sum, and same sign as the last term
		\item if the numbers are different signs, then the middle term is the difference, and the sign is the same
		largest absolute value
	\end{itemize} 
\end{myDefinition}

\subsection{Products and sums of numbers}
We begin with an exercise in sums and products.  Consider the following table
\begin{center}
	\begin{tabular}{cSS}
		\toprule
        Two numbers & {Their product} & {Their sum} \\
		\midrule
		            & 8           & 6       \\
		            & 8           & 9       \\
		            & 8           & -9      \\
		\bottomrule
	\end{tabular}
\end{center}

In the left hand column, we wish to find two numbers that when multiplied together, give the middle column, and
when added together, give the third column. So we can fill in the first entry by noting that 
\[
	4+2=6,	
\]
and 
\[
	4\cdot 2=8
\]
So the numbers that we need to put into the first column are 
\begin{center}
	\begin{tabular}{cSS}
		\toprule
        Two numbers        & {Their product} & {Their sum} \\
		\midrule
		{\color{blue}4, 2} & 8           & 6       \\
		                   & 8           & 9       \\
		                   & 8           & -9      \\
		\bottomrule
	\end{tabular}
\end{center}



Note: choosing $4$ and $2$ was a very good first guess. In fact there are other factors of $8$ such as
$8$ and $1$, but these clearly give $8+1=9$, which is not $6$ (the required sum).

Similarly, in the second row we want two numbers that when multiplied together give $8$, and when 
added together give 9. We see that the required numbers are $8$ and $1$, since
\[
	8+1=9, \qquad 8*1=8
\]
We can therefore fill in the next row of our table
\begin{center}
	\begin{tabular}{cSS}
		\toprule
        Two numbers        & {Their product} & {Their sum} \\
		\midrule
		{\color{blue}4, 2} & 8           & 6       \\
		{\color{blue}8, 1} & 8           & 9       \\
		                   & 8           & -9      \\
		\bottomrule
	\end{tabular}
\end{center}

The last row is interesting, as the product is {\em negative}, and yet their product is still {\em positive}. Remember
from elementary arithmetic, that a negative number times a negative number is a positive. For example
\[
	(-2)(-5)=10.
\]
We use this result in the last row of the table. The factors of 8 are 
\[
	1, 8, \qquad 2,4,	\qquad -2, -4, \qquad -8, -1.
\]

The only pair that adds up to -9 is -8 and -1. So we complete the final row of the table with
\begin{center}
	\begin{tabular}{cSS}
		\toprule
        Two numbers           & {Their product} & {Their sum} \\
		\midrule
		{\color{blue}4, 2}    & 8           & 6       \\
		{\color{blue}8, 1}    & 8           & 9       \\
		{\color{blue}-8, -1 } & 8           & -9      \\
		\bottomrule
	\end{tabular}
\end{center}

{\bfseries\itshape Before reading the rest of this section, complete the following table. Answers
	are shown in the footnote \footnote{5,4; 10,2; -10,-2; 7,2; -13,-1; -3, 1; -7,1; }}
\begin{center}
	\begin{tabular}{cSS}
		\toprule
		Two numbers & {Their product} & {Their sum} \\
		\midrule
		            & 20              & 9           \\
		            & 20              & 12          \\	
		            & 20              & -12         \\	
		            & 14              & 9           \\
		            & 13              & -14         \\	
		            & -3              & -2          \\	
		            & -7              & -6          \\	
		            & -7              & 6           \\	
		            & 60              & -23         \\	
		            & -15             & -2          \\	
		            & -17             & 16          \\	
		\bottomrule
	\end{tabular}
\end{center}

\begin{myexample}
Factor
\[
	x^2+7x+10
\]
{}
\end{myexample}
\begin{myProof}
	\begin{itemize}
		\item First we determine the signs on the connectors of each binomial. Since
		the middle term and the last term are positive, we will have
		\[
			(\phantom{x}+\phantom{x})(\phantom{x}+\phantom{x})
		\]
		\item Next we determine where the first term (product) could have come from. 
		The choices are only $x$ and $x$. 
		\[
			(x+\phantom{x})(x+\phantom{x})
		\]
		\item To determine the numbers, we need to look at the last term. Since this will
		be the product of the factors, we need to write the number as the product of possible
		factors
		\[
			10 = 10(1)	\qquad 10 = 5 \cdot 2	
		\]
		\item Since the middle term is $7$ we know that we need the sum of the factors to be 7. 
		So we therefore try
		\[
			(x+5)(x+2)
		\]
		\item We should check our answer using the FOIL technique (exercise).
		\item Because of the commutative property of multiplication, the answer can also be written as
		\[
			(x+2)(x+5)
		\]
	\end{itemize}
\end{myProof} 

\begin{myexample}
Factor
\[
	x^2-14x+45
\]
{}
\end{myexample}
\begin{myProof}
	\begin{itemize}
		\item First we determine the signs on the connectors of each binomial. Since
		the middle term and the last term are negative, we will have
		\[
			(\phantom{x}-\phantom{x})(\phantom{x}-\phantom{x})
		\]
		\item Next we determine where the first term (product) could have come from. 
		The choices are only $x$ and $x$. 
		\[
			(x-\phantom{x})(x-\phantom{x})
		\]
		\item To determine the numbers, we need to look at the last term. Since this will
		be the product of the factors, we need to write the number as the product of possible
		factors
		\[
			45=45(1)\qquad 45=3(15)\qquad 5(9)
		\]
		\item Since the middle term is $-14$ we know that we need the sum of the factors to be -14. 
		So we therefore try
		\[
			(x-5)(x-9)
		\]
		\item We should check our answer using the FOIL technique (exercise).
		\item Because of the commutative property of multiplication, the answer can also be written as
		\[
			(x-9)(x-5)
		\]
	\end{itemize}
\end{myProof} 

\begin{myexample}\label{ex:factoring1}
Factor
\[
	y^2+5y-24
\]
{}
\end{myexample}
\begin{myProof}
	\begin{itemize}
		\item First we determine the signs on the connectors of each binomial. Since
		the middle term is positive, and the last term is negative, we will have
		\[
			(\phantom{x}-\phantom{x})(\phantom{x}+\phantom{x})
		\]
		\item Next we determine where the first term (product) could have come from. 
		The choices are only $y$ and $y$. 
		\[
			(y-\phantom{y})(y+\phantom{y})
		\]
		\item To determine the numbers, we need to look at the last term. Since this will
		be the product of the factors, we need to write the number as the product of possible
		factors
		\begin{tightcenter}
			$-24 = -1(24)\qquad -24=-24(1) \qquad -24=-12(2) \qquad -24=-2(12)$ \\
			$-24 = -8(3) \qquad -24=-3(8)\qquad -24=-6(4) \qquad -24 = -4(6)$
		\end{tightcenter}
		\item Since the middle term is $5$ we know that we need the sum of the factors to be $5$.
		So we therefore try
		\[
			(y-3)(y+8)
		\]
		\item We should check our answer using the FOIL technique (exercise).
		\item Because of the commutative property of multiplication, the answer can also be written as
		\[
			(y+8)(y-3)
		\]
	\end{itemize}
	Note: if we had tried
	\[
		(y+3)(y-8)
	\]
	then this would not have worked, as the O and I terms from FOIL do not add together to give $5$.
\end{myProof} 

\begin{myexample}
Factor
\[
	w^2+12w-64
\]
{}
\end{myexample}
\begin{myProof}
	\begin{itemize}
		\item First we determine the signs on the connectors of each binomial. Since
		the middle term is positive, and the last term is negative, we will have
		\[
			(\phantom{x}-\phantom{x})(\phantom{x}+\phantom{x})
		\]
		\item Next we determine where the first term (product) could have come from. 
		The choices are only $x$ and $x$. 
		\[
			(x-\phantom{x})(x+\phantom{x})
		\]
		\item To determine the numbers, we need to look at the last term. Since this will
		be the product of the factors, we need to write the number as the product of possible
		factors
		\begin{tightcenter}
			$-64=-64(1)\qquad -64=-1(64)\qquad -64=-32(2) \qquad -64 = -2(32) $\\
			$-64=-16(4)\qquad -64 = -4(16)\qquad -64 = -8(8)$
		\end{tightcenter}
		\item Since the middle term is $12$ we know that we need the sum of the factors to be $12$.
		So we therefore try
		\[
			(x-4)(x+16)
		\]
		\item We should check our answer using the FOIL technique (exercise).
		\item Because of the commutative property of multiplication, the answer can also be written as
		\[
			(x+16)(x-4)
		\]
	\end{itemize}
	Note: in this example there were quite a few different options to choose from. This is typical, and you 
	should expect to do some trial error for each exercise. Experience and practise will help to guide you.
\end{myProof}

\begin{myexample}
Factor
\[
	2x^2+10x-48
\]
{}
\end{myexample}
\begin{myProof}
	At first glance, it may appear that this example should not be in this section, as $a\ne 1$; however, we note
	that as a first step we may remove a factor of 2 from each term
	\[
		2x^2+10x-48 = 2(x^2+5x-24)
	\] 
	We recognize the expression within the () as \cref{ex:factoring1} (except with $x$ instead of $y$),
	so we can complete this example as follows
	\[
		2(x^2+5x-24)	=	2(x+8)(x-3)	
	\]
	As in the previous examples, we should check our answer by FOILing (exercise).
\end{myProof} 

\begin{myexample}
Factor the following.
\drillandskill
\end{myexample}
\begin{multicols}{2}
	\begin{enumerate}
		\item $x^2 + 4x+3 = \solution{(x+{3})}\solution{(x+{1})}$
		\item $x^2+5x+4 = \solution{(x+{4})}\solution{(x+{1})}$
		\item $x^2-7x+12=\solution{(x{-4})}\solution{(x{-3})}$
		\item $x^2+16x-17=\solution{(x{+17})}\solution{(x{-1})}$
		\item $x^2+13x+30=\solution{(x{+10})}\solution{(x{+3})}$
		\item $x^2-13x+30=\solution{(x{-10})}\solution{(x{-3})}$
		\item $x^2+12x+27=\solution{(x{+9})}\solution{(x{+3})}$
		\item $x^2+6x-27=\solution{(x{+9})}\solution{(x{-3})}$
		\item $x^2-6x-27=\solution{(x{-9})}\solution{(x{+3})}$
		\item $x^2+10x+21=\solution{(x{+7})}\solution{(x{+3})}$
		\item $x^2+13x+22=\solution{(x{+11})}\solution{(x{+2})}$
		\item $x^2-24x+80=\solution{(x{-20})}\solution{(x{-4})}$
		\item $x^2+18x+81=\solution{(x{+9})}\solution{(x{+9})}$
		\item $x^2+29x+100=\solution{(x{+25})}\solution{(x{+4})}$
	\end{enumerate} 
\end{multicols}

\begin{myexample}
Factor the following more advanced problems
\end{myexample}
\begin{multicols}{2}
	\begin{enumerate}
		\item $
		\begin{aligned}[t]
			2x^2-4x-30 & =  \solution{2({x^2-2x-15})}  \\
			           & =  \solution{2(x{-5})(x{+3})} 
		\end{aligned}
		$
		\item $w^4+3w^2+2 = \solution{(w^2+2)(w^2+1)}$
		\item $
		\begin{aligned}[t]
			3w^4+9w^2+6 & = \solution{3(w^4+3w^2+2)}    \\
			            & =  \solution{3(w^2+2)(w^2+1)} 
		\end{aligned}
		$
		\item $
		\begin{aligned}[t]
			4x^2+8x+4 & =  \solution{4(x^2+2x+1)} \\
			          & =  \solution{4(x+1)(x+1)} 
		\end{aligned}
		$
		\item $
		\begin{aligned}[t]
			-x^2-3x-2 & =  \solution{-(x^2+3x+2)} \\
			          & =  \solution{-(x+2)(x+1)} 
		\end{aligned}
		$
		\item $
		\begin{aligned}[t]
			-4x^2-20x-16 & =  \solution{-4(x^2+5x+4)} \\
			             & =  \solution{-4(x+4)(x+1)} 
		\end{aligned}
		$
		\item $
		\begin{aligned}[t]
			x^3+3x^2+2x & =  \solution{x(x^2+3x+2)} \\
			            & =  \solution{x(x+2)(x+1)} 
		\end{aligned}
		$
		\item $
		\begin{aligned}[t]
			x^4+3x^3+2x^2 & =  \solution{x^2(x^2+3x+2)} \\
			              & =  \solution{x^2(x+2)(x+1)} 
		\end{aligned}
		$
		\item $x^2+5xy+4y^2=\solution{(x+4y)(x+y)}$
		\item $x^2-5xy+4y^2=\solution{(x-4y)(x-y)}$
		\item $a^2+8ab+7b^2=\solution{(a+7b)(a+b))}$
		\item $x^2+9xy+14y^2=\solution{(x+7y)(x+2y)}$
	\end{enumerate}
\end{multicols}

\section{Factoring trinomials when $a\ne 1$}
\textref{7.3}{430}%
We have so far looked at two forms of factoring
\begin{itemize}
	\item factoring by grouping
	\item factoring by reversing the FOIL procedure
\end{itemize} 
In all of our examples so far, the leading coefficient of the polynomial has been $a=1$,
for example
\[
	x^2+4x+3
\]
In this section we will study a slightly more complicated set of polynomials that have leading
coefficient {\em other than} 1. We will restrict ourselves to trinomials of \gls{degree} 2 (which we will
later term {\em Quadratics}), which have the form
\[
	ax^2+bx+c
\]
We will attempt to factor these using the `$ac$ method', which relates to using the $a$ and the $c$
in the above. This method is particularly effective when $a\ne 1$.

Note that factoring trinomials when $a\ne 1$ requires patience, so do not be disappointed if it
takes some time to sharpen the skill.

\subsection{The $ac$ method}
\begin{myexample}
Factor
\[
	8x^2-14x-15
\]
{}
\end{myexample}
\begin{myProof}
	The first observation we make about this trinomial is that the leading term has a coefficient
	of $8$, which significantly hinders our previous approach to factoring.
				
	We will organize our information in the following table
	\begin{itemize}
		\item the first column displays $a$, the second displays $c$, and the third displays $ac$
		\item the fourth column displays every combination of (integer) factors of the product $ac$
		\item we sum the factors in the fifth column, and compare them to the middle term in the
		original trinomial
	\end{itemize} 
				
	\begin{list}{}{%
			\setlength{\leftmargin}{-1.5cm}
			\setlength{\rightmargin}{1.5cm}}%
		\item[]%
		\begin{tabular}{SSScSSp{4.5cm}}
			\toprule
			&		&		&			&	\multicolumn{2}{c}{Need these 2 columns to match}	&	\\
			\midrule
			{$a$} & {$c$} & {$ac$} & Factors & {Sum of factors} & {middle term} & Match?                                          \\
			8     & -15   & -120   & -1,120  & 119              & -14           & No. Also note that the middle term is negative, 
			so when we list the factors let's make the factor
			with the largest absolute value negative so the sum will be negative\\
			8     & -15   & -120   & 1,120   & -119             & -14           & No- but at least it's the same sign!            \\
			8     & -15   & -120   & 2,-60   & -58              & -14           & No- but getting closer!                         \\
			8     & -15   & -120   & 3,-40   & -37              & -14           & No- but getting closer!                         \\
			8     & -15   & -120   & 4,-30   & -26              & -14           & No- but getting closer!                         \\	
			8     & -15   & -120   & 5,-24   & -19              & -14           & No- but getting closer!                         \\
			8     & -15   & -120   & 6,-20   & -14              & -14           & Bingo!                                          \\
			\bottomrule
		\end{tabular}
	\end{list} 
				
	We still have some work to do, as the factors 6 and -20 will not appear in our final answer. 
				
	Recall that in all the binomial multiplication that we did, the middle term came from adding the Inner and the Outer terms. We now need
	to find those 2 terms, which is where the 6 and -20 come in. Notice that
	\[
		6x-20x = -14x
	\]
	which is the middle term of our trinomial. 
				
	This means that we can rewrite our trinomial as
	\[
		8x^2 - 14x-15 = 8x^2 -20x+6x-15
	\]
	We can now use the skills we learnt in factoring by grouping (see \vref{sec:facbygroup}) and proceed as follows
	\begin{align*}
		8x^2 - 14x-15 & =		8x^2-20x+6x-15                             \\
		              & =		(8x^2-20x)+(6x-15)                         \\
		              & =		4x{\color{red}(2x-5)}+3{\color{red}(2x-5)} \\
		              & =		(4x+3){\color{red}(2x-5)}                  
	\end{align*} 
	What if we had written the middle term as $6x-20x$ instead? We would still end up at the same answer, but the
	working would be as follows
	\begin{align*}
		8x^2-14x-15 & =		8x^2+6x-20x-15                               \\
		            & =		(8x^2+6x)+(-20x-15)                          \\
		            & =		2x{\color{blue}(4x+3)}-5{\color{blue}(4x+3)} \\
		            & =		(2x-5){\color{blue}(4x+3)}{}                 
	\end{align*} 
	which is obviously the same as before.
\end{myProof} 

\begin{myexample}
Factor
\[
	4w^2-24w-64
\]
{}
\end{myexample}
\begin{myProof}
	At first glance this seems like a trinomial with $a\ne 1$, but if we examine each term individually we notice
	that each of the coefficients is divisible by 4. We can therefore factor out 4 from each term and rewrite the 
	trinomial as
	\[
		4w^2-24w-64 = 4(w^2-6w-16)
	\]
	Now we need to see if we can factor $w^2-6w-16$, which is straight forward since $a=1$. 
	\begin{center}
		\begin{tabular}{SSScSSl}
			\toprule
			&		&		&			&	\multicolumn{2}{c}{Need these 2 columns to match}	&	\\
			\midrule
			{$a$} & {$c$} & {$ac$} & {Factors} & {Sum of factors} & {middle term} & Match? \\
			1     & -16   & -16    & 1,-16     & -16              & -6            & No     \\
			1     & -16   & -16    & -2,8      & 6                & -6            & No     \\
			1     & -16   & -16    & -4,4      & 0                & -6            & No     \\
			1     & -16   & -16    & 2,-8      & -6               & -6            & Yes    \\
			\bottomrule
		\end{tabular}
	\end{center} 
				
	We can use this table to write
	\[
		w^2-6w-16 = (w+2)(w-8)
	\]
	The answer to our original question is therefore
	\begin{align*}
		4w^2-24w-64 & =	 	4(w^2-6w-16) \\
		            & =		4(w-8)(w+2)   
	\end{align*} 
\end{myProof} 

\subsection{Trial and error `method'}
This can not really be described as a method, but another way to factor is by trial and error. Consider	again factoring
\[
	8x^2-14x-15
\]
We could try any one of the following factorizations until we achieve the correct result by FOILing
\begin{align*}
	8x^2-14x-15 & \stackrel{?}{=}		(8x-3)(x+5)  \\
	8x^2-14x-15 & \stackrel{?}{=}		(8x+3)(x-5)  \\
	8x^2-14x-15 & \stackrel{?}{=}		(8x-1)(x+15) \\
	8x^2-14x-15 & \stackrel{?}{=}		(8x-15)(x+1) \\
	8x^2-14x-15 & \stackrel{?}{=}		(4x-3)(2x+5) \\
	8x^2-14x-15 & =			(4x+3)(2x-5)              \\
\end{align*} 

\begin{myexample}
\drillandskill
Factor the following - notice that both connectors are $+$:
\begin{multicols}{2}
	\begin{enumerate}
		\item $2x^2+3x+1$\solution{$=(2x+1)(x+1)$}
		\item $2x^2+5x+2$\solution{$=(2x+1)(x+2)$}
		\item $2x^2+7x+6$\solution{$=(2x+3)(x+2)$}
		\item $2x^2+11x+12$\solution{$=(2x+3)(x+4)$}
	\end{enumerate}
\end{multicols}

Factor the following -notice that both connectors are $-$:
\begin{multicols}{2}
	\begin{enumerate}
		\item $2x^2-3x+1$\solution{$=(2x-1)(x-1)$}
		\item $2x^2-5x+2$\solution{$=(2x-1)(x-2)$}
		\item $2x^2-7x+6$\solution{$=(2x-3)(x-2)$}
		\item $2x^2-11x+12$\solution{$=(2x-3)(x-4)$}
	\end{enumerate}
\end{multicols}

Factor the following - notice that the connectors are one of each:
\begin{multicols}{2}
	\begin{enumerate}
		\item $3x^2-2x-1$\solution{$=(3x+1)(x-1)$}
		\item $3x^2-5x-2$\solution{$=(3x+1)(x-2)$}
		\item $3x^2-4x-4$\solution{$=(3x+2)(x-2)$}
		\item $3x^2-2x-8$\solution{$=(3x+4)(x-2)$}
	\end{enumerate}
\end{multicols}

Factor the following (miscellaneous)
\begin{multicols}{2}
	\begin{enumerate}
		\item $6x^2+5x+1$\solution{$=(3x+1)(2x+1)$}
		\item $6x^2-x-1$\solution{$=(3x+1)(2x-1)$}
		\item $10x^2+9x+2$\solution{$=(5x+2)(2x+1)$}
		\item $10x^2+x-2$\solution{$=(5x-2)(2x+1)$}
		\item $12x^2+41x+24$\solution{$=(4x+3)(3x+8)$}
	\end{enumerate}
\end{multicols}

Factor the following- look for a common factor first!
\begin{enumerate}
	\item $6x^4+13x^3+6x^2$\solution{$=x^2(3x+2)(2x+3)$}
	\item $6x^4+5x^3-6x^2$\solution{$=x^2(3x-2)(2x+3)$}
	\item $8x^5y+22x^4y+15x^3y$\solution{$=x^3y(4x+5)(2x+3)$}
	\item $16x^5y-60x^4y+50x^3y$\solution{$=2x^3y(4x-5)(2x-5)$}
\end{enumerate}

Factor the following:
\begin{multicols}{2}
	\begin{enumerate}
		\item $4x^2+4xy+y^2$\solution{$=(2x+y)(2x+y)$}
		\item $6x^2+5xy+y^2$\solution{$=(2x+y)(3x+y)$}
	\end{enumerate}
\end{multicols}
\end{myexample}


%+*** mainfile.tex
% arara: makechapters: { files: [ mainfile], items: [module6] }
% !arara: indent: { overwrite: on, trace: yes, localSettings: on}
\chapter{Factoring}
\minitoc

\section{Factoring special products}
\textref{7.4}{438}%
We have so far considered factoring by grouping, reversing
the \gls{FOIL} procedure, and using the $ac$ method. In this section we will consider factoring
polynomials that have arisen from the following special products:
\begin{itemize}
	\item the sum and difference of two terms
	\item the square of the sum of two terms
	\item the square of the difference of two terms
\end{itemize} 

\subsection{The sum and difference of two terms}
Remember that when multiplying two binomials, one of which is the sum of two
terms and the other is the difference of the same the two terms, the FOILing procedure
gives a result that has no {\color{blue}{O}}utside or {\color{blue}{I}}nside terms. Consider the following examples
to refresh your memory
\[
	\begin{array}{lclcl}
		(x+2)(x-2)   & = & x^2-2x+2x-4         & = & x^2-4   \\
		(y-3)(y+3)   & = & y^2-3y+3y-9         & = & y^2-9   \\
		(2x+5)(2x-5) & = & 4x^2 -10x + 10x -25 & = & 4x^2-25 \\
		(4a-1)(4a+1) & = & 16a^2+4a-4a -1      & = & 16a^2-1 
	\end{array}
\]

We therefore see that the only terms we need to consider when FOILing these types of products are
the  {\color{blue}{F}}irst and  {\color{blue}{L}}ast terms. 

Note: the connector between the terms in the end result will always be a `-' sign if it has arisen
from the product of the sum and difference of two terms. If the connector is positive then
the \gls{polynomial} can not be factored (in the real numbers) and is called `\gls{prime}'.

These observations will help us in factoring expressions. The technique to factoring a difference of squares can
be described as follows
\begin{steps}
	\item Are there any common factors?
	\item \Gls{factor} the first term (find the square root of this term)
	\item Factor the last term (find the square root of this term)
	\item Create binomials with parentheses
	\item Put the first terms in each \gls{binomial}
	\item Put the last term in each binomial
	\item Put an addition sign between one binomial and a subtraction sign in between the other binomial
	\item As always, CHECK using the FOIL method
\end{steps} 
\label{list:stepsforfactoring}
\begin{myexample}\label{ex:factordiffsquares}
Factor
\[
	x^2-16
\]
{}
\end{myexample}
\begin{myProof}
	We follow the steps listed above
	\begin{steps}
		\item Are there any common factors?\hfill No
		\item	Factor the first term \hfill               $(x)^2$
		\item	Factor the last term  \hfill               $(4)^2$
		\item	Create the parenthesis\hfill               $(\phantom{x+4})(\phantom{x-4})$
		\item	Put the first term in each binomial\hfill  $(x\phantom{+4})(x\phantom{-4})$
		\item	Put the last terms in the binomial  \hfill $(x\phantom{+}4)(x\phantom{-}4)$
		\item	Put in one of each sign              \hfill $(x+4)(x-4)$
		\item	Check using FOIL                     \hfill exercise                     
	\end{steps}
	We conclude that 
	\[
		x^2-16 = (x+4)(x-4)
	\]
	{}
\end{myProof} 

\begin{myexample}
Factor
\[
	9x^2-25
\]
{}
\end{myexample}
\begin{myProof}
	We proceed as in \cref{ex:factordiffsquares}
	\begin{steps}
		\item	Are there any common factors?      \hfill  No                             
		\item	Factor the first term              \hfill  $(3x)^2$
		\item	Factor the last term               \hfill  $(5)^2$
		\item	Create the parenthesis             \hfill  $(\phantom{x+4})(\phantom{x-4}) $  
		\item	Put the first term in each binomial\hfill  $(3x\phantom{+4})(3x\phantom{-4})$ 
		\item	Put the last terms in the binomial \hfill  $(3x\phantom{+}5)(3x\phantom{-}5)$ 
		\item	Put in one of each sign            \hfill  $(3x+5)(3x-5)                    $ 
		\item	Check using FOIL                   \hfill  exercise                       
	\end{steps}
	We conclude that
	\[
		9x^2-25 = (3x+5)(3x-5)
	\]
	{}
\end{myProof} 

In some cases there might be a common factor which should be taken care of first. For example
\[
	2x^2-8 = 2(x^2-4)
\]
from which we can clearly move on as demonstrated in the previous examples. 

In other cases one of the binomials might need to be factored again. The next two examples 
show this.

\begin{myexample}
Factor
\[
	x^4-1
\]
\end{myexample}
\begin{myProof}
	Let's follow the steps we've followed in the previous examples:
	\begin{steps}
		\item	Are there any common factors?                          \hfill No                               
		\item	Factor the first term                                  \hfill $(x^2)^2$
		\item	Factor the last term                                   \hfill $(1)^2$
		\item	Create the parenthesis                                 \hfill $(\phantom{x+4})(\phantom{x-4})$     
		\item	Put the first term in each binomial                    \hfill $(x^2\phantom{+4})(x^2\phantom{-4})$
		\item	Put the last terms in the binomial                     \hfill $(x^2\phantom{+}1)(x^2\phantom{-}1)$ 
		\item	Put in one of each sign                                \hfill $(x^2+1)(x^2-1)$                     
		\item	ADDITIONAL STEP: Can any of the new terms be factored? \hfill Yes                              
		\item	Factor the second binomial                             \hfill $(x^2+1)(x-1)(x+1)$
		\item	Check using FOIL                                       \hfill exercise                         
	\end{steps}
	We conclude that
	\[
		x^4-1 = (x^2+1)(x+1)(x-1)
	\]
	{}
\end{myProof} 

\begin{myexample}
Factor
\[
	2x^3-8x
\]
{}
\end{myexample}
\begin{myProof}
	Off we go with the steps from \cpageref{list:stepsforfactoring}.
	\begin{steps}
		\item Are there any common factors?                \hfill Yes, $2x$    
		\item Factor out the common factor                 \hfill $2x(x^2-4)$    
		\item Does this contain the difference of squares? \hfill Yes        
		\item Factor using the previous techniques         \hfill $2x(x+2)(x-2)$
		\item Check                                        \hfill exercise   
	\end{steps}
	We conclude that
	\[
		2x^3-8x = 2x(x+2)(x-2)
	\]
	{}
\end{myProof} 

\begin{myexample}
\drillandskill
Factor the following 
\end{myexample}
\begin{multicols}{2}
	\begin{enumerate}
		\item $x^2-1$  \solution{$=(x-1)(x+1)$}
		\item $x^2-25$ \solution{$=(x-5)(x+5)$}
		\item $x^2-36$ \solution{$=(x-6)(x+6)$}
		\item $x^2-49$ \solution{$=(x-7)(x+7)$}
		\item $
		\begin{aligned}[t]
			x^4-1 & =  \solution{(x^2-1)(x^2+1)}    \\
			      & =  \solution{(x-1)(x+1)(x^2+1)} 
		\end{aligned}
		$
		\item $x^4-16$ \solution{$=(x-2)(x+2)(x^2+4)$}
		\item $x^4-81$ \solution{$=(x-3)(x+3)(x^2+9)$}
		\item $x^4-256$\solution{$=(x-4)(x+4)(x^2+16)$}
	\end{enumerate}
\end{multicols}

Factor the following- remove a common factor first!
\begin{multicols}{2}
	\begin{enumerate}
		\item $
		\begin{aligned}[t]
			x^3-x & =  \solution{x (x^2-1)}   \\
			      & =  \solution{x(x-1)(x+1)} 
		\end{aligned}$
		\item $x^3-25x$ \solution{$=x(x-5)(x+5)$}
		\item $x^3-36x$ \solution{$=x(x-6)(x+6)$}
		\item $x^3-49x$ \solution{$=x(x-7)(x+7)$}
	\end{enumerate}
\end{multicols}

Factor the following:
\begin{enumerate}
	\item $4x^2-1$        \solution{$=(2x-1)(2x+1)$}
	\item $9x^2-4$        \solution{$=(3x-2)(3x+2)$}
	\item $4x^2-25y^2$    \solution{$=(2x-5y)(2x+5y)$}
	\item $16a^4 - 36b^4$ \solution{$=4(2a^2-3b^2)(2a^2+3b^2)$}
\end{enumerate}

\subsection{The square of the sum or difference of two terms}
Remember from our previous work in FOILing polynomials, there are two special products
of binomials, which result in a `perfect square \gls{trinomial}'. This means that a binomial was
squared to get the trinomial. 

For example, we considered polynomials such as 
\begin{itemize}
	\item $(x+3)^2 = x^2+6x+9$
	\item $(3x-4)^2 = 9x^2-24x+16$
	\item $(5a+1)^2 = 25a^2+10a+1$
\end{itemize} 
Notice in using the FOIL method that the {\color{blue}O} and {\color{blue}I} 
terms are the same in each case, which means that the middle terms gets {\em doubled}. 
What we double is actually the product of the two terms. 

To determine if a trinomial is a perfect square trinomial, the following questions may be useful:
\begin{steps}
	\item Is the first term a perfect square? What from?	
	\item Is the last term positive and a perfect square? What form?
	\item Is the middle term double the product of the `what forms' above?
\end{steps} 
If we can answer each of these questions, then we have every chance of factoring a trinomial
that has arisen from squaring the sum or difference of two terms. We will demonstrate the technique
with examples.

\begin{myexample}
Factor
\[
	x^2+10x+25
\]
{}
\end{myexample}
\begin{myProof}
	\begin{steps}
		\item Is the first term a perfect square?          \hfill Yes.  What form? $x^2$
		\item Is the last term added and a perfect square? \hfill Yes.  What form? $5^2$ 
		\item Is the middle term double the product? \hfill $2(x)(5)=10x$. Yes! 
		\item Now rewrite as a perfect square binomial     \hfill $(x+5)^2$ 
		\item Check:
		$\begin{aligned}[t]
			(x+5)^2 & =		x^2+5x+5x+25 \\
			        & =		x^2+10x+25   
		\end{aligned}$
	\end{steps}
	{}
\end{myProof} 

\begin{myexample}
Factor
\[
	x^2-10x+25
\]
\end{myexample}
\begin{myProof}
	\begin{steps}
		\item Is the first term a perfect square?          \hfill Yes.    What form?  $x^2$
		\item Is the last term added and a perfect square? \hfill Yes.    What form?  $5^2$  or  $(-5)^2$
		\item Is the middle term double the product?        \hfill 
		$\begin{aligned}[t]
			2(x)(5)=10x     & \text{ No}  \\
			2(x)(-5) = -10x & \text{ Yes} 
		\end{aligned}$
		\item Now rewrite as a perfect square binomial     \hfill $(x-5)^2$
		\item Check:
		$\begin{aligned}[t]
			(x-5)^2 & =		x^2-5x-5x+25 \\
			        & =		x^2-10x+25   
		\end{aligned}$
	\end{steps}
	{}
\end{myProof} 

\begin{myexample}
Factor
\[
	18c^2+48c + 32
\]
\end{myexample}
\begin{myProof}
	\begin{steps}
		\item Is there a common factor?                    \hfill Yes, $2$ is the common factor 
		\item Factor this term out:                        \hfill $2(9c^2+24c+16)$
		\item Is the first term a perfect square?          \hfill Yes:  $3c$
		\item Is the last term added and a perfect square? \hfill Yes:  $4$                      
		\item Is the middle term double the product?       \hfill Yes:  $2(3c)(4)=24c$
		\item Rewrite as a binomial                        \hfill $2(3c+4)^2$                     
		\item Check: 
		$\begin{aligned}[t]
			2(3c+4)^2 & = 2(9c^2+24c+16) \\
			          & = 18c^2+48c+32   
		\end{aligned}$ 
	\end{steps}
	{}
\end{myProof} 

\section{Factoring sums and differences of cubes}
In the previous module we considered factoring special products; in particular, this 
included
\begin{itemize}
	\item the product of the sum and difference of two terms, e.g $(x-y)(x+y)=x^2-y^2$
	\item the square of the sum of two terms e.g $(x+y)^2=x^2+2xy+y^2$
	\item the square of the difference of two terms e.g $(x-y)^2=x^2-2xy+y^2$
\end{itemize} 
We begin this module by considering factoring the sum or difference of two cubes of the form
\[
	x^3+y^3, \qquad  x^3-y^3
\]

\subsection{Factoring the sum of two cubes}
We begin this discussion with an example in multiplying two polynomials- this will hopefully
make the process of factoring more transparent.

\begin{myexample}\label{ex:sumcubes}
Multiply 
\[
	(x+y)(x^2-xy+y^2)
\]
\end{myexample}
\begin{myProof}
	We use vertical format for this problem, and write like terms underneath one another:
	\begin{alignat*}{3}
		(x+y)(x^2-xy+y^2) & =  x^3 & -x^2y & +xy^2 &      \\
		                  &        & +yx^2 & -xy^2 & +y^3 \\
	\end{alignat*}
	Notice here that the two `mixed' terms consisting of both $x$ and $y$ cancel one another, 
	and so the result that we are left with is
	\[
		(x+y)(x^2-xy+y^2) = x^3+y^3
	\]  
	This will clearly be useful to us when factoring. We now demonstrate this technique with some examples.
\end{myProof} 

\begin{myexample}
Factor
\[
	x^3+8
\]
\end{myexample}
\begin{myProof}
	In this example it is crucial to note that $8=2^3$. This means that we can write our \gls{expression}
	as
	\[
		x^3 +8 = x^3 + 2^3
	\]
	which is the sum of two cubes, and means that the work we described in \cref{ex:sumcubes}
	will be very useful to us. 
				
	Therefore
	\begin{align*}
		x^3+2^3 & =  (x+2)(x^2-2x+2^2) \\
		        & =  (x+2)(x^2-2x+4)   
	\end{align*} 
	As always, it is vital to check that this works by multiplying the result (exercise).
\end{myProof} 

\begin{myexample}\label{ex:anothersumcubes}
Factor
\[
	8x^3+27
\]
\end{myexample}
\begin{myProof}
	In this example we first need to realize that, by the properties of exponents, $8x^3 = (2x)^3$, and
	that $27=3^3$. For our examples, there will usually be a `nice' factorization. We can use this
	information and put it together with our theory so far
	\begin{align*}
		8x^3+27 & =  (2x)^3+3^3                 \\
		        & =  (2x+3)((2x)^2-(2x)(3)+3^2) \\
		        & =  (2x+3)(4x^2-6x +9)         
	\end{align*}{} 
	As in the previous example, we must always check our answer (exercise).
\end{myProof} 

\begin{myexample}
\drillandskill
Factor the following:
\end{myexample}

\begin{multicols}{2}
	\begin{enumerate}
		\item $x^3+1$    \solution{$=(x+1)(x^2-x+1)$}
		\item $x^3+27$   \solution{$=(x+3)(x^2-3x+9)$}
		\item $x^3+64$   \solution{$=(x+4)(x^2-4x+16)$}
		\item $x^3+125$  \solution{$=(x+5)(x^2-5x+25)$}
		\item $8x^3+1$   \solution{$=(2x+1)(4x^2-2x+1)$}
		\item $8x^3+27$  \solution{$=(2x+3)(4x^2-6x+9)$}
		\item $8x^3+27y^3$\solution{$=(2x+3y)(4x^2-6xy+9y^2)$}
		\item $27x^6+64$  \solution{=$(3x^2+4)(9x^2-12x^2+16)$}
	\end{enumerate}
\end{multicols}

\subsection{Factoring the difference of two cubes}
This discussion parallels the technique used to factor the sum of two cubes. We begin by
demonstrating an example involving polynomial multiplication, and then move onto factorization.

\begin{myexample}
Multiply
\[
	(x-y)(x^2+xy+y^2)
\]
\end{myexample}
\begin{myProof}
	We use the vertical format
	\begin{alignat*}{3}
		(x-y)(x^2+xy+y^2) & =  x^3 & +x^2y & +x y^2 &      \\
		                  &        & -yx^2 & -xy^2  & -y^3 
	\end{alignat*}
	We notice that the mixed terms cancel, and that we are left with
	\[
		(x-y)(x^2+xy+y^3) = x^3-y^3
	\]
	We will use this when factoring the difference of two cubes.
\end{myProof} 

\begin{myexample}
Factor
\[
	y^3-64
\]
\end{myexample}
\begin{myProof}
	The crucial observation that we must make here is that $64=4^3$. This therefore means
	that
	\begin{align*}
		y^3-64 & =  y^3 - 4^3             \\
		       & =  (y-4)(y^2+(y)(4)+4^2) \\
		       & =  (y-4)(y^2+4y+16)      
	\end{align*}
	{}
\end{myProof} 

\begin{myexample}
Use \cref{ex:anothersumcubes} to help you factor
\[
	8x^3-27
\]
\end{myexample}
\begin{myProof}
	As we observed in \cref{ex:anothersumcubes}, $8x^3 = (2x)^3$, and $27=3^3$. This allows
	us to write
	\begin{align*}
		8x^3-27 & =  (2x)^3 - 3^3                 \\
		        & =  (2x-3)((2x)^2 + (2x)(3)+3^2) \\
		        & =  (2x-3)(4x^2 + 6x + 9)        
	\end{align*} 
\end{myProof} 

\begin{myexample}
\drillandskill
Factor the following:
\begin{multicols}{2}
	\begin{enumerate}
		\item $x^3-1$       \solution{$=(x-1)(x^2+x+1)$}
		\item $x^3-27$      \solution{$=(x-3)(x^2+3x+9)$}
		\item $x^3-64$      \solution{$=(x-4)(x^2+4x+16)$}
		\item $x^3-125$     \solution{$=(x-5)(x^2+5x+25)$}
		\item $a^3-b^3$     \solution{$=(a-b)(a^2+ab+b^2)$}
		\item $27a^3-b^3$   \solution{$=(3a-b)(9a^2+3ab+b^2)$}
		\item $64a^3-b^3$   \solution{$=(4a-b)(16a^2+4ab+b^2)$}
		\item $64a^3-b^3c^3$\solution{$=(4a-bc)(16a^2+4abc+b^2c^2)$}
	\end{enumerate}
\end{multicols}

\end{myexample}

\begin{myexample}
\drillandskill
Factor the following miscellaneous expressions:
\end{myexample}
\begin{multicols}{2}
	\begin{enumerate}
		\item $x^4+x$    \solution{$=x(x+1)(x^2-x+1)$}
		\item $x^4+27x$  \solution{$=x(x+3)(x^2-3x+9)$}
		\item $x^4+64x$  \solution{$=x(x+4)(x^2-4x+16)$}
		\item $x^4+125x$ \solution{$=x(x+5)(x^2-5x+25)$}
		\item $x^4-x$    \solution{$=x(x-1)(x^2+x+1)$}
		\item $x^4-27x$  \solution{$=x(x-3)(x^2+3x+9)$}
		\item $x^4-64x$  \solution{$=x(x-4)(x^2+4x+16)$}
		\item $x^4-125x$ \solution{$=x(x-5)(x^2+5x+25)$}
	\end{enumerate}
\end{multicols}

\section{Solving quadratic equations by factoring}
\textref{7.6}{454}%
\reformatstepslist{Q} % the steps list should be P1, P2, \ldots
In this class we have so far considered factoring polynomials and trinomials in a number of different
ways. One of the many applications of this skill is in solving {\em \gls{quadratic} equations}. A quadratic
\gls{equation} has the form
\[
	ax^2+bx+c=0
\]
where $a$, $b$, and $c$ are real numbers, and we assume that $a\ne 0$. We have actually been working
with quadratic expressions a lot so far in our work, but we have been calling them 2nd \gls{degree} polynomials, or
trinomials. Each description is equivalent.

In this section we will discuss the zero product rule, and demonstrate how to use the zero product rule together
with factorization in order to \gls{solve} quadratic equations. 

\begin{myDefinition}\label{sec:zeroprodprin}
	%\subsection{Zero product rule}\label{sec:zeroprodprin}
	{Zero product principle}: 
	This rule says that if the product of two numbers is 0, then either the first one equals 0, or the second one
	equals 0, or they both equal 0. 
				
	Mathematically, we say that if
	\[
		AB=0
	\]
	then either $A=0$ or $B=0$.
\end{myDefinition}

This property will allow us to solve quadratic equations by first factoring the quadratic expression, then set
each of the factors equal to 0, then solve each equation.

We demonstrate this with an example.

\begin{myexample}
Solve the quadratic equation
\[
	x^2+3x+2=0
\]
\end{myexample}
\begin{myProof}
	If we are to use the zero product principle, then we must first factor the quadratic. Remember that we 
	are looking for factors of 2 that add up to 3. We therefore see that this equation can be written as
	\[
		(x+2)(x+1)=0
	\]
	The zero product rule tells us that either
	\[
		x+2=0\qquad or \qquad x+1 = 0
	\]
	Solving these two equations should pose no problem to us, and we see that these equations give
	\[
		x=-2 \qquad or \qquad x=-1
	\]
	It remains to check our answers, which we do by plugging them back into the original equation
	\[
		(-2)^2 + 3(-2)+2 = 0 \qquad {\mathrm and}\qquad (-1)^2+3(-1)+2=0
	\]
	{}
\end{myProof} 

This method relies on being able to factor. This may take time to develop so be patient. We can
summarize the procedure for solving a quadratic equation by factoring as follows
\begin{steps}
	\item Put it in standard form with all terms involving $x$ on one side of the = sign
	\item Factor our any common factors
	\item Factor what is left
	\item Set each factor with a \gls{variable} in it to 0
	\item Solve each individual equation
	\item Place all answers in set notation
	\item Check in the original equation to see if they do indeed make the equation true
\end{steps} 

If there are no solutions to the given equation then the overall \gls{solution} is the empty set which
can be written as either $\{\}$ or $\emptyset$. In the event that every number will work for this equation
then the solution is All Real Numbers.

\begin{myexample}
Use factoring to solve the quadratic equation
\[
	x^2+x-42 = 0
\]
\end{myexample}
\begin{myProof}
	\begin{steps}
		\item Is it in standard form and equal to 0?     \hfill Yes               
		\item	Are there any common factors?              \hfill No                
		\item	Factor                                     \hfill $(x+7)(x-6)=0$        
		\item	Set each factor with a variable equal to 0 \hfill $x+7=0$  and  $x-6=0$ 
		\item	Solve each equation                        \hfill $x=-7$  and  $x=6$    
		\item	Put your answers in set notation           \hfill $\{-7,6\}$
		\item	Finally we check our answers
		\[
			(-7)^2+(-7)-42 = 0 \qquad (6)^2+6-42 = 0
		\]
		both of which are true. We conclude that the solutions to the given equation are
		\[
			\{-7,6\}
		\]
	\end{steps}
	{}
\end{myProof}

The next example is different from most that we have considered previously, as it is not a trinomial
(a polynomial with three terms). The process that we use to solve this type of equation follows the 
same methodology, as we will demonstrate.

\begin{myexample}
Solve the quadratic equation by factoring
\[
	x^2=8x
\]
\end{myexample}
\begin{myProof}
	\begin{steps}
		\item Is it in standard form and equal to 0? \hfill No  $x^2-8x=0$
		\item	Are there any common factors?          \hfill No              
		\item	Factor                                 \hfill $x(x-8)=0$          
		\item	Set up equations:                      \hfill $x=0$  and  $x-8=0$ 
		\item	Solve                                  \hfill $x=0$  and  $x=8$   
		\item	Put answers in set notation            \hfill $\{0,8\}$           
		\item	Finally we check our answers
		\[
			0^2 = 8(0) \qquad 8^2 = 8(8)
		\]
		both of which are true, and we conclude that the solutions to the given equation are
		\[
			\{0,8\}
		\]
	\end{steps}
	{}
\end{myProof} 

In our examples so far there have been two distinct (unequal) answers. This may not always be the case, 
as we will see in the next example. 

\begin{myexample}
Solve the quadratic equation by factoring
\[
	9x^2-30x = -25
\]
\end{myexample}
\begin{myProof}
	\begin{steps}
		\item Put in standard form and equal to 0 \hfill $9x^2-30x+25 = 0$                     
		\item	Are there any common factors?       \hfill No                                
		\item	Factor                              \hfill $(3x-5)(3x-5)=0$                      
		\item	Set up equations:                   \hfill $3x-5=0$  and  $3x-5=0$               
		\item	Solve                               \hfill $x=\frac{5}{3}$  and  $x=\frac{5}{3}$ 
		\item	Put answers in set notation         \hfill $\left\{\frac{5}{3}\right\}$
		\item Check \hfill exercise
	\end{steps}
	Note that there is no need to write the the same number twice. After checking (exercise), we conclude that
	the solution set to the given equation is
	\[
		\left\{\frac{5}{3}\right\}
	\]
	It seems like there is only one answer here, but in fact we say that there are two and that the solution is
	repeated. 
\end{myProof}

We have so far seen examples that demonstrate that it is possible to have {\em two} real solutions, or {\em one} real
solution. The only other case is when there are {\em no} real solutions; we will study this case in a later module. For
the moment we continue with some further examples. 

\begin{myexample}
Solve the following quadratic equation by factoring
\[
	25x^2=49
\]
\end{myexample}
\begin{myProof}
	\begin{steps}
		\item Put in standard form and equal to 0 \hfill $25x^2-49=0$                           
		\item	Are there any common factors?       \hfill No                                 
		\item	Factor                              \hfill $(5x-7)(5x+7)=0$                       
		\item	Set up equations:                   \hfill $5x-7=0$  and  $5x+7=0$                
		\item	Solve                               \hfill $x=\frac{7}{5}$  and  $x=-\frac{7}{5}$ 
		\item	Put answers in set notation         \hfill $\left\{\frac{7}{5}, -\frac{7}{5}\right\}$
		\item Check   \hfill exercise
	\end{steps}
	Once we have checked (exercise), we can conclude that the solution to the given equation is
	\[
		\left\{\frac{7}{5}, -\frac{7}{5}\right\}
	\] 
	When we have two solutions that are the same but with opposite sign, we can use the following abbreviation
	\[
		x = \pm \frac{7}{5}
	\]
	We read this as `$x$ equals plus or minus 7 over 5'.
\end{myProof} 

\begin{myexample}
Solve the following equations by factoring.
\drillandskill
\end{myexample}

{\em Straight forward}
\begin{multicols}{2}
	\begin{enumerate}
		\item $x^2+5x+4=0$  \solution{$\left\{-1,-4\right\}$}
		\item $x^2+14x+13=0$ \solution{$\left\{-1,-13\right\}$}
		\item $x^2-4x-5=0$ \solution{$\left\{5,-1\right\}$}
		\item $x^2-5x-6=0$ \solution{$\left\{6,-1\right\}$}
		\item $x^2+12x-64=0$ \solution{$\left\{4,-16\right\}$}
		\item $x^2-22x+72=0$ \solution{$\left\{18,4\right\}$}
		\item $x^2-x-30=0$ \solution{$\left\{6,-5\right\}$}
		\item $x^2-2x-35=0$ \solution{$\left\{7,-5\right\}$}
	\end{enumerate}
\end{multicols}
{\em Slightly harder}
\begin{multicols}{2}
	\begin{enumerate}
		\item $x^2+5x=-4$  \solution{$\left\{-1,-4\right\}$}
		\item $x^2-18x=-80$ \solution{$\left\{10,8\right\}$}
		\item $x^2=-6x-8$ \solution{$\left\{-2,-4\right\}$}
		\item $x^2=3x+18$ \solution{$\left\{6,-3\right\}$}
		\item $10x-48 = -2x^2$ \solution{$\left\{3,-8\right\}$}
		\item $2x^2+24 = 14x$ \solution{$\left\{4,3\right\}$}
		\item $-3x-10 = -x^2$ \solution{$\left\{5,-2\right\}$}
		\item $45x = -15x^2+60$ \solution{$\left\{1,-4\right\}$}
	\end{enumerate}
\end{multicols}

{$a\ne 1$}
\begin{multicols}{2}
	\begin{enumerate}
		\item $2x^2+3x+1=0$ \solution{$\left\{-\frac{1}{2}, -1\right\}$}
		\item $2x^2+19x+35=0$ \solution{$\left\{-\frac{5}{2}, -7\right\}$}
		\item $3x^2+14x-5=0$ \solution{$\left\{\frac{1}{3}, -5\right\}$}
		\item $6w^2-11w+4=0$ \solution{$\left\{\frac{4}{3}, \frac{1}{2}\right\}$}
	\end{enumerate}
\end{multicols}

{\em Difference of squares}
\begin{multicols}{2}
	\begin{enumerate}
		\item $x^2-25=0$ \solution{$\left\{\pm 5\right\}$}
		\item $x^2 = 4$ \solution{$\left\{\pm 2 \right\}$}
		\item $x^2 = 16$ \solution{$\left\{\pm 4\right\}$}
		\item $x^2 = 64$ \solution{$\left\{\pm 8\right\}$}
		\item $4x^2 = 16$ \solution{$\left\{\pm 2\right\}$}
		\item $-x^2 = -9$ \solution{$\left\{\pm 3\right\}$}
		\item $3x^2 = 75$ \solution{$\left\{\pm 5\right\}$}
		\item $81x^2 = 25$ \solution{$\left\{\pm \frac{5}{9}\right\}$}
	\end{enumerate}
\end{multicols}

{$(a+b)^2$ or $(a-b)^2$}
\begin{multicols}{2}
	\begin{enumerate}
		\item $x^2+4x+4 = 0$ \solution{$\left\{-2\right\}$}
		\item $x^2-4x+4 = 0$ \solution{$\left\{2\right\}$}
		\item $x^2+8x+16 = 0$ \solution{$\left\{-4\right\}$}
		\item $x^2+16 = 8x$ \solution{$\left\{4\right\}$}
		\item $2x^2+20x+50 = 0$ \solution{$\left\{-5\right\}$}
		\item $3x^2+12x+12=0$ \solution{$\left\{-2\right\}$}
	\end{enumerate}
\end{multicols}

\begin{multicols}{2}
	\begin{enumerate}
		\item $x^2+2x = -4x-8$ \solution{$\left\{-2,-4\right\}$}
		\item $x^2+3x = -5x-15$ \solution{$\left\{-3,-5\right\}$}
		\item $x^2+7x = 4x+28$ \solution{$\left\{4,-7\right\}$}
		\item $x^2+3x = 5x+15$ \solution{$\left\{5,-3\right\}$}
	\end{enumerate}
\end{multicols}


\section{Solving quadratic equations and problem solving}\label{sec:solvequadfac}
In the previous section we introduced quadratic equations, and they have the form
\[
	ax^2+bx+c=0
\]
where $a$, $b$, and $c$ are real numbers and we assume that $a\ne 0$. We learnt how we
can solve quadratic equations (i.e find the values of $x$ that satisfy the equation), by
factoring the equation and using the {\em Zero product principle} (see 
\cpageref{sec:zeroprodprin}). In this section we will apply 
our factoring skills and knowledge of quadratic equations to some word problems. We will work through
a series of examples to demonstrate approaches and techniques.

\begin{myexample}
The area of a square is 121 square units. Find the length of its sides.
\begin{center}
	\begin{tikzpicture}
		\draw (0,0)--(2,0)--(2,2)--(0,2)--cycle;
		\draw (-.5,1) node {$x$};
		\draw (1,-.5) node {$x$};
	\end{tikzpicture}
\end{center}

\end{myexample}
\begin{myProof}
	\begin{itemize}
		\item The first key piece of information that we need to remember about squares
		is that they have sides of equal length. This means that if we let one of
		the lengths be $x$, then the other sides must also have length $x$, as shown in the
		above diagram.
		\item The second piece of information that we need to remember about squares is that the area is
		the (length of one side)$^2$. In our example, this therefore translates to
		\[
			{\mathrm{area~of~square} = x^2}
		\]
		\item Combining this information with the fact that the area of this square is 121 square units gives
		the equation
		\[
			x^2=121
		\]	
	\end{itemize} 
	Remember from our work with factoring (see \cref{sec:solvequadfac}) that the first step is to write the equation in standard form, in
	other words with all of the terms on one side
	\[
		x^2-121=0
	\]
	On remembering that $121=11^2$, we realize that this is one of the special products discussed previously, so we can write it as
	\[
		(x-11)(x+11)=0
	\]
	We can now use the principle of zero products which means that we have to solve the two equations
	\[
		x-11=0, \qquad x+11=0
	\]
	At this stage we have to think back to what $x$ represents in this problem. Recall that $x$ is the length
	of one of the sides of the square, so the questions we must ask ourselves are:
	\begin{itemize}
		\item can a square have two different values for the length of its sides?
		\item can a square have a negative value for one of its lengths?
	\end{itemize}
	The answer to both of these questions is clearly {\em no}, so at this stage we must clearly reject the solution $x=-11$ as it is unrealistic.
				
	We therefore conclude that the length of side of a square that has an area of 121 square units is 11 units. 
				
	We check our answer by calculating $11^2=121$.
	{}
\end{myProof}

\begin{myexample}
An object is thrown upward from the top of an 80 foot building with an initial
velocity of 64 feet per second. The height of the object above the ground, $h$, after
$t$ seconds is given by
\[
	h = -16t^2+64t+80
\]
When will the object hit the ground? 
\end{myexample}
\begin{myProof}
	We have considered an example similar to this in \vref{ex:fallingobject}. In the current context
	we are concerned with the time when the object will hit the ground; since $h$ is the height of the {\em object
		above the ground}, when the object is on the ground this corresponds to $h=0$. So, we need to solve the equation
	\[
		0=-16t^2+64t+80
	\]
	Our approach will be
	\begin{itemize}
		\item factor the equation
		\item use the zero product rule to find the values of $t$ that satisfy the equations
		\item check our answer
	\end{itemize} 
				
	We begin by factoring the equation. At first glance, it seems intimidating since the coefficients are all fairly
	large numbers. However, note first that we can remove a factor of $16$ from each of the terms, so we can therefore
	write the equation as
	\[
		0=-t^2+4t+5
	\]
	Note: It is certainly possible to continue factoring at this stage, but the factor of $-1$ in front of the $t^2$ complicates
	matters somewhat; we can remove it by multiplying both sides of the equation by $-1$ to give
	\[
		0 = t^2-4t-5
	\]
	This equation is now significantly more friendly than the original, and using the techniques described in previous modules we see
	that we can factor is as
	\[
		0=(t-5)(t+1)
	\]
	and therefore by the zero product rule,
	\[
		t=5 {\mathrm{~or~}} t=-1
	\]
	In principal we have two values of $t$, but there are two questions that we must ask ourselves
	\begin{itemize}
		\item {\color{red}can there be two values of $t$ in this problem?} Theoretically speaking (in general), yes there could be. If the object
		were starting at a height of $0ft$ above the ground then $t=-1$ could represent this. However, we know that the object is starting
		at a height of 80ft above the ground, son in this particular example, the answer is NO.
		\item {\color{red}can $t$ be negative?} In this example $t$ represents time. For the duration of our class we will assume that $t$ can not 
		be negative, so the answer to this question is NO.
	\end{itemize} 
				
	We therefore conclude that $t=5$ is the time at which the object hits the ground. We must check our answer
	\[
		-16(5)^2+64(5)+80=0
	\]
	We can confirm this graphically by plotting values of $h$ against values of $t$ as shown below in 
	\cref{fig:thrownobject}
				
	\begin{figure}[!h]
		\centering
		\begin{tikzpicture}
			\begin{axis}[
					framed,
					xmin=-1,xmax=6,
					ymin=-20,ymax=150,
					xlabel={$t$},
					ylabel={$h$},
					xtick={0,...,5},
					ytick={0,20,...,200},
					grid=major
				]
				\addplot+[-]expression[domain=0:5,samples=100]{-16*x^2+64*x+80};
				\legend{$h(t)=-16t^t+64t+80$};
			\end{axis}
		\end{tikzpicture}
		\caption{A thrown object}
		\label{fig:thrownobject}
	\end{figure}
	\FloatBarrier
				
	The \gls{point} where the graph cuts the horizontal axis at $t=5$ is called the {\em horizontal \gls{intercept}} and in our example
	represents the value of $t$ at which the height of the object is $0$. 
				
	In general quadratics will have {\em at most} two horizontal intercepts.
\end{myProof}

%+*** mainfile.tex
% arara: pdflatex: { files: [ mainfile.tex ] }
% !arara: indent: { overwrite: on, trace: yes, localSettings: on}
\chapter{Radicals}
\minitoc
\section{Finding roots}
\textref{9.1}{558}%
In the previous module we considered factoring \gls{quadratic} {\em expressions} of the form
\[
	ax^2+bx+c
\]
and applied the skill to solving quadratic {\em equations} of the form
\[
	ax^2+bx+c=0
\]
This process works well for equations that can be factored, but we will soon encounter
equations that can not be factored simply. In this case, we will need another tool to help
us, and in order to develop this tool we need to discuss {\em square roots}. In fact
we will also briefly discuss cubic roots. We begin with some terminology.

\begin{myDefinition}
	Terminology:
	\begin{itemize}
		\item Square roots- if $b^2=a$ then $b$ is a square root of $a$
		\item Radical sign- this is the symbol to indicate that a root is required $\sqrt{}$
		\item Radicand - the terms inside the radical sign, for example in $\sqrt{9}$ the radicand is $9$
	\end{itemize} 
	{}
\end{myDefinition} 

\begin{myexample}
Find the following
\begin{multicols}{3}
	\begin{enumerate}
		\item $\sqrt{9}$
		\item $\sqrt{25}$
		\item $\sqrt{64}$
	\end{enumerate} 
\end{multicols}
\end{myexample}
\begin{myProof}
	\begin{enumerate}
		\item We are looking for a number that when squared gives us $9$. We see that $3^2=9$, and also that
		$(-3)^2=9$. The {\em principal square root} of $9$ is 3; if we wanted the negative square root, then
		we would write $-\sqrt(9)$. We therefore conclude that
		\[
			\sqrt{9}=3
		\]
		\item Using a similar argument to part a
		\[
			\sqrt{25} = 5
		\]
		\item And finally
		\[
			\sqrt{64} = 8
		\]
	\end{enumerate} 
\end{myProof} 

You might have examined finding square roots of numbers in previous classes. Perhaps you
are used to the symbol but not necessarily the terminology. The numbers we encountered
in this example worked out very nicely, in that the results were whole numbers (integers). When
a number results from squaring an integer, we call it a {\em perfect square}. 

\begin{myDefinition}
	Principle square root: If $a$ is a non-negative real number, the non-negative number $b$ such that $b^2=a$, denoted
	by $b=\sqrt{a}$, is the principal square root of $a$. 
				
	Perfect squares: When the square root of a number is an integer, that number is a perfect square. For example, 
	100 is a perfect square since $\sqrt{100}=10$, and $17$ is not a perfect square.
\end{myDefinition}

Note: for the purposes of this class we will only consider taking the square root of non-negative numbers; if 
ever we do find ourselves in the situation of taking the square root of a negative number, we will give the answer
as `not a real number'.

\begin{myexample}
Find the following
\begin{multicols}{3}
	\begin{enumerate}
		\item $\sqrt{36}$
		\item $-\sqrt{49}$
		\item $-\sqrt{81}$
	\end{enumerate} 
\end{multicols}
\end{myexample}
\begin{myProof}
	\begin{enumerate}
		\item $\sqrt{36}=6$
		\item $-\sqrt{49}=-7$
		\item $-\sqrt{81}=-9$
	\end{enumerate} 
\end{myProof}
Note that the - sign in front of the radical symbol indicates taking the negative square root of the number,

We have so far discussed square roots, and in fact there are an infinity of other roots that we can consider. For example
\begin{myDefinition}
	Cube root: If $b^3=a$ then $b$ is the cube root of $a$, and we write $b=\sqrt[3]{a}$
					
	Fourth root: If $b^4=a$ then $b$ is a fourth root of $a$, and we write $b=\sqrt[4]{a}$
\end{myDefinition}


Notice the subtle use of language here; for the cube root we said that $b$ is {\em the} cube root of $a$, where
as for the fourth root, we said that $b$ is  {\em a } fourth root of $a$.

The {\em index} of radical is the number found in the crook of the radical sign, and indicates which root is
desired. Although the index is not written for finding square roots, it is in fact understood to 
be 2. Technically the square root could look like $\sqrt[2]{}$, but since square roots are used so frequently, 
the index is dropped. 

If the index is even, the radicand must be non-negative for the root to be a real number.

\begin{myexample}
Find the following $n^{th}$ roots; if they do not exist as a real number then say so.
\begin{multicols}{4}
	\begin{enumerate}
		\item $\sqrt[3]{-27}$
		\item $\sqrt[4]{81}$
		\item $\sqrt[4]{-8}$
		\item $\sqrt[3]{8}$
	\end{enumerate}
\end{multicols}
{}
\end{myexample}
\begin{myProof}
	\begin{enumerate}
		\item $\sqrt[3]{-27} = -3$ because $(-3)^3 = -27$. Notice how we {\em do} need parenthesis here
		\item $\sqrt[4]{81} = 3$ because $3^4 = 81$. Notice how we do {\em not} need parenthesis here.
		\item $\sqrt[4]{-8}$ does not exist as a real number because there is no real value of $x$ such that $x^4 = -81$. In fact, provided the
		index of the radical symbol is even, we necessarily require the radicand to be non-negative.
		\item $\sqrt[3]{8} = 2$ because $2^3 = 8$.
	\end{enumerate}
	{}
\end{myProof}

\begin{myexample}
\drillandskill
\Gls{simplify} the following
\end{myexample}
{\em Exponent reminder: Evaluate the following}:
\begin{multicols}{4}
	\begin{enumerate}
		\item $3^2$  \solution{$=9$}
		\item $(-3)^2$ \solution{$=9$}
		\item $-3^2$ \solution{$=-9$}
		\item $4^2$ \solution{$=16$}
		\item $5^2$ \solution{$=25$}
		\item $6^2$ \solution{$=36$}
		\item $7^2$ \solution{$=49$}
		\item $8^2$ \solution{$=64$}
		\item $2^3$ \solution{$=8$}
		\item $3^3$ \solution{$=27$}
		\item $4^3$ \solution{$=64$}
		\item $5^3$ \solution{$=125$}
		\item $(-2)^3$ \solution{$=-8$}
		\item $(-3)^3$ \solution{$=-27$}
		\item $(-4)^3$ \solution{$=-64$}
		\item $(-5)^3$ \solution{$=-125$}
		\item $1^4$ \solution{$=1$}
		\item $2^4$ \solution{$=16$}
		\item $3^4$ \solution{$=81$}
		\item $4^4$ \solution{$=256$}
	\end{enumerate}
\end{multicols}
{\em Find the following square roots}
\begin{multicols}{4}
	\begin{enumerate}
		\item $\sqrt{4}$ \solution{$=2$}
		\item $\sqrt{9}$ \solution{$=3$}
		\item $\sqrt{25}$ \solution{$=5$}
		\item $-\sqrt{16}$ \solution{$=-4$}
		\item $-\sqrt{36}$ \solution{$=-6$}
		\item $\sqrt{49}$ \solution{$=7$}
		\item $\sqrt{100}$ \solution{$=10$}
		\item $\sqrt{144}$ \solution{$=12$}
	\end{enumerate}
\end{multicols}

\section{Simplifying radical expressions}
\textref{9.2}{558}%
In the previous section we introduced radicals. In this section we will see examples of how to manipulate and
simplify some elementary radical expressions (this will be built upon significantly in Math 95). 

\subsection{Multiplying radical expressions}
\begin{myexample}
Find the following 
\begin{multicols}{2}
	\begin{enumerate}
		\item $\sqrt{9}\sqrt{16}$
		\item $\sqrt{9}\sqrt{25}$
	\end{enumerate} 
\end{multicols}
\end{myexample}
\begin{myProof}
	\begin{enumerate}
		\item 
		$\begin{aligned}[t]
			\sqrt{9}\sqrt{16} & =  3(4) \\
			                  & = 12    
		\end{aligned}$
		\item 
		$\begin{aligned}[t]
			\sqrt{4}\sqrt{25} & =  2(5) \\
			                  & = 10    
		\end{aligned}$
	\end{enumerate} 
\end{myProof} 
Now let's try some experimentation with some related examples. Consider
\begin{align*}
	\sqrt{9(16)} & =		\sqrt{144} \\
	             & =		12         
\end{align*} 
Compare this to part a). What do you notice? Next consider
\begin{align*}
	\sqrt{4(25)} & =		\sqrt{100} \\
	             & =		10         
\end{align*} 
Compare this to part b). What do you notice?

Hopefully you can see that
\[
	\sqrt{9(16)} = \sqrt{9}\cdot \sqrt{16}
\]
and that
\[
	\sqrt{4(25)} = \sqrt{4}\cdot\sqrt{25}
\]

In other words, when multiplying inside the radical sign we can distribute the radical
symbol to each \gls{factor}. We can use this in simplifying radical expressions that are not
perfect squares.

\begin{myexample}\label{ex:multradicals}
Simplify the following
\begin{multicols}{3}
	\begin{enumerate}
		\item $\sqrt{50}$
		\item $\sqrt{90}$
		\item $\sqrt{2x^6}$
	\end{enumerate} 
\end{multicols}
\end{myexample}
\begin{myProof}
	\begin{enumerate}
		\item 
		$\begin{aligned}[t]
			\sqrt{50} & =  \sqrt{2(25)}      \\
			          & =  \sqrt{2}\sqrt{25} \\
			          & =  5\sqrt{2}         
		\end{aligned}$
						
		Note: the reason we chose to write 50 as $50=2(25)$ is that 25 is a perfect square.
		\item 
		$\begin{aligned}[t]
			\sqrt{90} & =  \sqrt{9(10)}      \\
			          & =  \sqrt{9}\sqrt{10} \\
			          & =  3\sqrt{10}        
		\end{aligned}$
						
		Note:  we chose to write 90 as $90=9(10)$ as 9 is a perfect square. We can not simplify this any
		further as 10 can not be factored into a product involving perfect squares.
		\item 
		$\begin{aligned}[t]
			\sqrt{2x^6} & =  \sqrt{2}\sqrt{x^6} \\
			            & =  \sqrt{2}x^3        
		\end{aligned}$
						
		Note: the reason that $\sqrt{x^6}=x^3$is that $(x^3)^2 = x^6$ (by the properties of exponents).
	\end{enumerate} 
\end{myProof} 

\begin{myDefinition}
	In the above examples we have developed and used the property that
	\begin{equation}\label{eq:multradicals}
		\sqrt{ab} = \sqrt{a}\sqrt{b}
	\end{equation}
	where $a, b\geq 0$. In fact, this can be applied to $n^{th}$ roots as well.
\end{myDefinition} 

\begin{myexample}\label{ex:multradicalsdrillskill}
Simplify the following:
\drillandskill
\end{myexample}
\begin{multicols}{4}
	\begin{enumerate}
		\item $\sqrt{75}$ \solution{$=5\sqrt{3}$}
		\item $\sqrt{32}$ \solution{$=4\sqrt{2}$}
		\item $\sqrt{44}$ \solution{$=2\sqrt{11}$}
		\item $\sqrt{50}$ \solution{$=5\sqrt{2}$}
		\item $\sqrt{60}$ \solution{$=2\sqrt{3}\sqrt{5}$}
		\item $\sqrt{63}$ \solution{$=3\sqrt{7}$}
		\item $\sqrt{48}$ \solution{$=2\sqrt{3}$}
		\item $\sqrt{24}$ \solution{$=2\sqrt{2}\sqrt{3}$}
	\end{enumerate}
\end{multicols}

\subsection{Dividing radical expressions}
We know that
\[
	5^2 = 25, \quad {\mathrm{and}} \quad \sqrt{25}=5
\]
and that
\[
	6^2 = 36, \quad {\mathrm{and}} \quad \sqrt{36}=6
\]
Also, we know that when multiplying fractions
\[
	\frac{5}{6}\cdot \frac{5}{6} = \frac{25}{36}
\]
We can use this to help us simplify radical expressions involving division. For example, consider
simplifying the following
\begin{align*}
	\sqrt{\frac{25}{36}} & = \frac{\sqrt{25}}{\sqrt{36}} \\
	                     & = \frac{5}{6}                 
\end{align*} 
\begin{myDefinition}
	The equivalent result to that shown in \cref{eq:multradicals} is
	\[
		\sqrt{\frac{a}{b}} = \frac{\sqrt{a}}{\sqrt{b}}
	\]
	We put the restriction that $b\ne 0$. We now demonstrate this with examples.
\end{myDefinition} 

\begin{myexample}
Simplify the following
\begin{multicols}{2}
	\begin{enumerate}
		\item $\dd\sqrt{\frac{100}{49}}$
		\item $\dd\sqrt{\frac{30}{49}}$
	\end{enumerate} 
\end{multicols}
\end{myexample}
\begin{myProof}
	\begin{enumerate}
		\item 
		$\begin{aligned}[t]
			\sqrt{\frac{100}{49}} & =  \frac{\sqrt{100}}{\sqrt{49}} \\
			                      & = \frac{10}{7}                  
		\end{aligned}$
		\item 
		$\begin{aligned}[t]
			\sqrt{\frac{30}{49}} & =  \frac{\sqrt{30}}{\sqrt{49}} \\
			                     & =  \frac{\sqrt{30}}{7}         
		\end{aligned}$
	\end{enumerate} 
			
	Note that we can not simplify $\sqrt{30}$ any further, so we leave it as is.
				
\end{myProof} 

\begin{myexample}
Simplify the following
\begin{multicols}{2}
	\begin{enumerate}
		\item $\dd\sqrt[3]{\frac{8}{27}}$
		\item $\dd\sqrt[4]{\frac{64}{81}}$
	\end{enumerate}
\end{multicols}
{}
\end{myexample}
\begin{myProof}
	\begin{enumerate}
		\item 
		$\begin{aligned}[t]
			\sqrt[3]{\frac{8}{27}} & =  \frac{\sqrt[3]{8}}{\sqrt[3]{27}} \\
			                       & =  \frac{2}{3}                      
		\end{aligned}$
		\item 
		$\begin{aligned}[t]
			\sqrt[4]{\frac{64}{81}} & =  \frac{\sqrt[4]{64}}{\sqrt[4]{81}} \\
			                        & =  \frac{4}{3}                       
		\end{aligned}$
	\end{enumerate} 
\end{myProof}

\begin{myexample}
Simplify the following
\drillandskill
\end{myexample}
\begin{multicols}{2}
	\begin{enumerate}
		\item $\dd\sqrt{\frac{25}{7}}$ \solution{$\dd=\frac{5}{\sqrt{7}}$}
		\item $\dd\frac{\sqrt{25}}{\sqrt{7}}$ \solution{$\dd=\frac{5}{\sqrt{7}}$}
		\item $\dd\frac{\sqrt{21}}{\sqrt{3}}$ \solution{$\dd=\sqrt{7}$}
		\item $\dd\frac{\sqrt{27}}{\sqrt{3}}$ \solution{$\dd=3$}
		\item $\dd\frac{\sqrt{35}}{\sqrt{5}}$ \solution{$\dd=\sqrt{7}$}
		\item $\dd\frac{\sqrt{45}}{\sqrt{3}}$ \solution{$\dd=\frac{3\sqrt{5}}{\sqrt{3}}$}
		\item $\dd\frac{\sqrt{15}}{\sqrt{5}}$ \solution{$\dd=\sqrt{3}$}
		\item $\dd\frac{\sqrt{55}}{\sqrt{3}}$ \solution{$\dd=\frac{\sqrt{5}\sqrt{11}}{\sqrt{3}}$}
	\end{enumerate}
\end{multicols}

\subsection{Simplifying radical expressions by combining like terms}
Sometimes radical expressions may be simplified by adding or subtracting. We can combine like
terms just as we would do with $x$ or $x^2$ terms for example. It is very important to note
that we can only add and subtract like radicals- for example the \gls{expression}
\[
	\sqrt{x}+ \sqrt[3]{x}
\]
can not be simplified any further.

\begin{myexample}
Combine
\[
	\sqrt{3}+5\sqrt{3} -2\sqrt{3}
\]
\end{myexample}
\begin{myProof}
	Are they all like radicals? Yes, so we can combine the coefficients. Remember that the \gls{coefficient}
	of the first term is 1 even though it is not seen there.
	\begin{align*}
		\sqrt{3}+5\sqrt{3}-2\sqrt{3} & =  (1+5-2)\sqrt{3} \\
		                             & =  4\sqrt{3}       
	\end{align*} 
	Even though we do not usually write the parenthesis as we have done in the above, this helps to 
	visualize the thought process that goes into calculating the answer. 
\end{myProof} 


\begin{myexample}
Simplify
\[
	2\sqrt{5}+3\sqrt{2}-4\sqrt{5}+6\sqrt{2}
\]
\end{myexample}
\begin{myProof}
	Remember to combine the only like radicals. The way we do this relies on the commutative
	property of addition- in other words we can rearrange the terms into any order
	\[
		2\sqrt{5} - 4\sqrt{5}+3\sqrt{2}+6\sqrt{2} = -2\sqrt{5}+9\sqrt{2}
	\]
	{}
\end{myProof} 
In the next examples we will combine the techniques that we have discussed so far.

\begin{myexample}
Simplify 
\[
	2\sqrt{18}+ 4\sqrt{50}
\]
\end{myexample}
\begin{myProof}
	Notice first of all that the radicands (the numbers inside the radical symbol) are not the same. This
	does not necessarily mean that the expression can not be simplified. We factor each radicand into
	the product of perfect squares, and simplify as follows
	\begin{align*}
		2\sqrt{18}+ 4\sqrt{50} & =  2\sqrt{9\cdot 2}+ 4 \sqrt{25\cdot 2}             \\
		                       & =  2\sqrt{9}\sqrt{2}+4\sqrt{25}\sqrt{2}             \\
		                       & =  2\cdot 3 \cdot \sqrt{2} + 4\cdot 5\cdot \sqrt{2} \\
		                       & =  6\sqrt{2} + 20 \sqrt{2}                          \\
		                       & =  26\sqrt{2}                                       
	\end{align*} 
	A safe technique is to work with each term individually, and then to combine like terms at the end.
\end{myProof} 
We have so far concentrated on factoring numbers under the radical symbol by factoring into perfect
squares or perfect cubes, etc. Another technique is to factor the radicand into the product of primes. 
Either way is perfectly acceptable, and you should pick whichever suits you best. The next example demonstrates
factoring into the product of primes. 

\begin{myexample}
Combine
\[
	5\sqrt{75}-4\sqrt{12}+3\sqrt{8}
\]
\end{myexample}
\begin{myProof}
	\begin{align*}
		5\sqrt{75}-4\sqrt{12}+3\sqrt{8} & =  5\sqrt{5\cdot 5\cdot 3 }	-4\sqrt{2\cdot 2 \cdot 3  }+3\sqrt{2\cdot 2 \cdot 2} \\  
		                                & =  5\cdot 5\sqrt{3}-4\cdot 2\sqrt{3} + 3\cdot 2\sqrt{2}                          \\
		                                & =  25\sqrt{3} - 8 \sqrt{3} + 6\sqrt{2}                                           \\	
		                                & =  17\sqrt{3}+6\sqrt{2}                                                          
	\end{align*} 
	Notice that the last line in the above does not contain like radicals, and can not be simplified any further.
\end{myProof} 

\begin{myexample}
Simplify
\[
	4\sqrt[3]{54} - 2\sqrt[3]{64}
\]
\end{myexample}
\begin{myProof}
	We begin by noting that $54=27\cdot 2$, and that $\sqrt[3]{64}=4$, which helps
	us as follows
	\begin{align*}
		4\sqrt[3]{54} - 2\sqrt[3]{64} & =  4\sqrt[3]{27 \cdot 2} - 2(4)       \\
		                              & =  4\sqrt[3]{27}\cdot \sqrt[3]{2} - 8 \\
		                              & =  12 \sqrt[3]{2}-8                   
	\end{align*}
	We can not simplify this expression any further.
\end{myProof}

\begin{myexample}
Find the following square roots- if the answer is not a real number, then say so
\drillandskill
\end{myexample}
\begin{multicols}{2}
	\begin{enumerate}
		\item $\sqrt{\frac{1}{4}}$ \solution{$=\frac{1}{2}$}
		\item $\sqrt{\frac{1}{9}}$ \solution{$=\frac{1}{3}$}
		\item $\sqrt{\frac{25}{36}}$ \solution{$=\frac{5}{6}$}
		\item $\sqrt{-\frac{1}{4}}$ \solution{not a real number}
		\item $\sqrt{9}+\sqrt{16}$ \solution{$=7$}
		\item $\sqrt{9+16}$ \solution{$=5$}
		\item $\sqrt{-\frac{1}{2}}$ \solution{not a real number}
		\item $\sqrt{3-17}$ \solution{not a real number}
	\end{enumerate}
\end{multicols}

\section{More on multiplying radicals}
\textref{9.3}{567}%
We saw in \vref{eq:multradicals} that
\[
	\sqrt{a}\sqrt{b} = \sqrt{ab}
\]
where $a, b\geq 0$. We have used this property to help us simplify radical 
expressions by factoring numbers into the product of perfect squares (see 
\cref{ex:multradicals,ex:multradicalsdrillskill}).

We can deduce that
\begin{myDefinition}
	For any $a\geq 0$
	\[
		\sqrt{a}\sqrt{a} = a
	\]
\end{myDefinition} 

\begin{myexample}
\drillandskill
Simplify the following:
\end{myexample}
\begin{multicols}{2}
	\begin{enumerate}
		\item $\sqrt{5}\sqrt{5} = 5$
		\item $\sqrt{7}\sqrt{7} = 7$
		\item $\sqrt{13}\sqrt{13}=13$
		\item $\sqrt{1045}\sqrt{1045}=1045$
	\end{enumerate} 
\end{multicols}

\subsection{FOILing with radicals}
We can operate with radicals using the same algebraic techniques we have 
developed so far.

\begin{myexample}
Perform the following multiplication
\begin{multicols}{2}
	\begin{enumerate}
		\item $\sqrt{3}(5+\sqrt{7})$
		\item $(\sqrt{2}+3)(\sqrt{5}+7)$
	\end{enumerate} 
\end{multicols}
\end{myexample}
\begin{myProof}
	\begin{enumerate}
		\item $ \sqrt{3}(5+\sqrt{7})  =  5\sqrt{3}+\sqrt{3}\sqrt{7} $
		\item We use the \gls{FOIL} method
		\[ 
			(\sqrt{2}+3)(\sqrt{5}+7)  =  \sqrt{2}\sqrt{5}+7\sqrt{2}+3\sqrt{5}+21
		\]
	\end{enumerate} 
\end{myProof} 

\begin{myexample}\label{ex:conjugatemult}
Perform the following multiplication:
\begin{multicols}{3}
	\begin{enumerate}
		\item $(2+\sqrt{3})(2-\sqrt{3})$
		\item $(5-\sqrt{7})(5+\sqrt{7})$
		\item $(\sqrt{11}+\sqrt{5})(\sqrt{11}-\sqrt{5})$
	\end{enumerate} 
\end{multicols}
\end{myexample}
\begin{myProof}
	We will use the FOIL method.
	\begin{enumerate}
		\item $
		\begin{aligned}[t]
			(2+\sqrt{3})(2-\sqrt{3}) & =  4 - 2\sqrt{3} + 2\sqrt{3} -3 \\
			                         & =  4-3                          \\
			                         & =  1                            
		\end{aligned}
		$
		\item $
		\begin{aligned}[t]
			(5-\sqrt{7})(5+\sqrt{7}) & =  25 + 5\sqrt{7} - 5\sqrt{7} -7 \\
			                         & =  25 -7                         \\
			                         & =  18                            
		\end{aligned}
		$
		\item $
		\begin{aligned}[t]
			(\sqrt{11}+\sqrt{5})(\sqrt{11}-\sqrt{5}) & =  11 -\sqrt{5}\sqrt{11} +\sqrt{5}\sqrt{11} - 5 \\
			                                         & =  11-5                                         \\
			                                         & =  6                                            
		\end{aligned}
		$
	\end{enumerate} 
\end{myProof} 

What do you notice about all of the solutions in \cref{ex:conjugatemult}? We note that each \gls{solution} does 
not contain a radical. Notice also that each problem involved the multiplication of the sum and difference of 
two terms. This leads us to the definition of the {\em conjugate}.

\begin{myDefinition}\label{def:conjugate}
	The conjugate of
	\[
		\sqrt{a}-\sqrt{b}
	\]
	is 
	\[
		\sqrt{a}+\sqrt{b}
	\]
	The conjugate is useful because
	\[
		(\sqrt{a}+\sqrt{b})(\sqrt{a}-\sqrt{b}) = a-b
	\]
\end{myDefinition} 

\begin{myexample}
Find the conjugate of each of the following:
\begin{multicols}{4}
	\begin{enumerate}
		\item $2+\sqrt{5}$
		\item $3-\sqrt{2}$
		\item $9+\sqrt{17}$
		\item $\sqrt{19}+10$
	\end{enumerate} 
\end{multicols}
\end{myexample}
\begin{myProof} 
	This is a simple procedure: change the sign between the terms.
	\begin{multicols}{4}
		\begin{enumerate}
			\item $2-\sqrt{5}$
			\item $3+\sqrt{2}$
			\item $9-\sqrt{17}$
			\item $\sqrt{19}-10$
		\end{enumerate} 
	\end{multicols}
\end{myProof} 

\begin{myexample}
\drillandskill
Perform the following multiplication:
\begin{multicols}{2}
	\begin{enumerate}
		\item $\sqrt{2}(11+\sqrt{3})$\solution{$=11\sqrt{2}+\sqrt{6}$}	
		\item $\sqrt{5}(11-\sqrt{3})$\solution{$=11\sqrt{5}-\sqrt{3}\sqrt{5}$}	
		\item $-\sqrt{3}(4+\sqrt{7})$\solution{$=-4\sqrt{3}-\sqrt{3}\sqrt{7}$}	
		\item $\sqrt{7}(\sqrt{19}+\sqrt{7})$\solution{$=\sqrt{7}\sqrt{19}+7$}	
		\item $(\sqrt{5}+2)(\sqrt{7}-1)$\solution{$=\sqrt{35}-\sqrt{5}+2\sqrt{7}-2$}
		\item $(\sqrt{3}+\sqrt{2})(\sqrt{3}+\sqrt{2})$\solution{$=5+2\sqrt{6}$}
		\item $(\sqrt{3}-\sqrt{2})(\sqrt{3}-\sqrt{2})$\solution{$=5-2\sqrt{6}$}
		\item $(\sqrt{3}+\sqrt{2})(\sqrt{3}-\sqrt{2})$\solution{$=1$}
	\end{enumerate}
\end{multicols}
\end{myexample}

%+*** mainfile.tex
% arara: pdflatex: { files: [ mainfile.tex ] }
% !arara: indent: { overwrite: on, trace: yes, localSettings: on}
\chapter{Quadratic equations}
\minitoc
\section{Rationalizing the denominator}
\subsection{Simple case}
\textref{9.4}{574}%
We have so far dealt with radical expressions such as
\[
	\sqrt{2} + 3\sqrt{2}
\]
There may be situations that force us to deal with radical symbols in the denominator, for example
\begin{equation}\label{eq:rationalize}
	\frac{1}{\sqrt{3}} + \sqrt{3}
\end{equation}
It is hard for a contemporary student to imagine a time without calculators, but be assured that 
it is only relatively recently that we have had impressive computing power at our finger tips. 

Historically if a student was asked to perform the calculation in \cref{eq:rationalize}
(by hand!) then they would find it significantly more simple if they could remove the radical symbol from the denominator. They would do so
as follows
\begin{align}
	\frac{1}{\sqrt{3}} + \sqrt{3} & =  \frac{1}{\sqrt{3}}\cdot {\color{red}\frac{\sqrt{3}}{\sqrt{3}}} + \sqrt{3}\label{eq:keysteprationalize} \\
	                              & =  \frac{\sqrt{3}}{3}+\sqrt{3} 	\nonumber                                                                 \\	
	                              & =  \frac{4\sqrt{3}}{3}\nonumber                                                                           
\end{align}
Notice that in \cref{eq:keysteprationalize} we multiply the term that has the radical in the denominator by a \gls{factor} of 1 (shown in red). We have used
the property of radicals that
\[
	\sqrt{3}\sqrt{3} = 3
\]
Even though the advent of technology means that we no longer need to perform such computations by hand, we still rationalize
the denominator as a matter of convention.

\begin{myexample}
Rationalize the denominator in each of the following expressions, and \gls{simplify}.
\begin{multicols}{3}
	\begin{enumerate}
		\item $\dd\frac{1}{\sqrt{2}}$
		\item $\dd\frac{2}{\sqrt{5}}+3\sqrt{3}$
		\item $\dd -\frac{3}{\sqrt{7}}+\sqrt{9}$
	\end{enumerate}
\end{multicols}
\end{myexample}
\begin{myProof}
	\begin{enumerate}
		\item 
		$\begin{aligned}[t]
			\frac{1}{\sqrt{2}} & =  \frac{1}{\sqrt{2}}\cdot \frac{\sqrt{2}}{\sqrt{2}} \\
			                   & =  \frac{\sqrt{2}}{2}                                
		\end{aligned}$
		\item 
		$\begin{aligned}[t]
			\frac{2}{\sqrt{5}}+3\sqrt{3} & =  \frac{2}{\sqrt{5}}\cdot \frac{\sqrt{5}}{\sqrt{5}}	+3 \sqrt{3} \\
			                             & =  \frac{2\sqrt{5}}{5}+ 3 \sqrt{3}                               
		\end{aligned}$
		\item 
		$\begin{aligned}[t]
			-\frac{3}{\sqrt{7}}+\sqrt{9} & =  -\frac{3}{\sqrt{7}}\cdot \frac{\sqrt{7}}{\sqrt{7}}+3 \\
			                             & =  -\frac{3\sqrt{7}}{7} + 3                             
		\end{aligned}$
	\end{enumerate}
	{}
\end{myProof}

\begin{myexample}
\drillandskill
Rationalize the denominator in each of the following.
\end{myexample}
\begin{multicols}{4}
	\begin{enumerate}
		\item $\dd\frac{1}{\sqrt{19}}$\solution{$=\dd\frac{\sqrt{19}}{19}$}
		\item $\dd\frac{2}{\sqrt{3}}$\solution{$=\dd\frac{2\sqrt{3}}{3}$}
		\item $\dd\frac{5}{\sqrt{30}}$\solution{$=\dd\frac{\sqrt{30}}{6}$}
		\item $\dd\frac{36}{\sqrt{40}}$\solution{$=\dd\frac{9\sqrt{40}}{10}$}
	\end{enumerate}
\end{multicols}

\subsection{Harder case}
We may sometimes wish to rationalize the denominator in expressions such as 
\begin{equation}\label{equn:hardrad}
	\frac{2}{3+\sqrt{7}}
\end{equation}
Our goal is the same as before- to remove the radical from the denominator. We 
do so using the conjugate, which we first introduced on \cpageref{def:conjugate}, 
but we state again for clarity:
\begin{myDefinition}
	The conjugate of
	\[
		\sqrt{a}-\sqrt{b}
	\]
	is 
	\[
		\sqrt{a}+\sqrt{b}
	\]
	The conjugate is useful because
	\[
		(\sqrt{a}+\sqrt{b})(\sqrt{a}-\sqrt{b}) = a-b
	\]
\end{myDefinition} 

We can use the conjugate to help us rationalize the denominator in \cref{equn:hardrad} as follows
\begin{align*}
	\frac{2}{3+\sqrt{7}} & =  \frac{2}{3+\sqrt{7}}\cdot {\color{red}\frac{3-\sqrt{7}}{3-\sqrt{7}}} \\
	                     & =  \frac{6-2\sqrt{7}}{9-7}                                              \\
	                     & =  \frac{6-2\sqrt{7}}{2}                                                \\
	                     & =  3 - \sqrt{7}                                                         
\end{align*} 
Notice that this process has removed the radical symbol from the denominator.

\begin{myexample}
\drillandskill
Rationalize the denominator (using the conjugate) in each of the following 
examples.
\end{myexample}
\begin{multicols}{2}
	\begin{enumerate}
		\item $\dd\frac{1}{\sqrt{2}+1}$\solution{$=\dd\sqrt{2}-1$}
		\item $\dd\frac{3}{\sqrt{5}+3}$\solution{$=\dd\frac{9}{4}-\frac{3\sqrt{5}}{4}$}
		\item $\dd\frac{10}{\sqrt{8}-\sqrt{2}}$\solution{$=\dd 5\sqrt{2}$}
		\item $\dd\frac{5}{\sqrt{7}-\sqrt{2}}$\solution{$=\dd \sqrt{2}+\sqrt{7}$}
	\end{enumerate}
\end{multicols}

\section{Solving quadratic equations by factoring}
\textref{7.6}{454}%
In the previous modules we have considered solving \gls{quadratic} equations by factoring;
we have also manipulated radical expressions. In this module we will combine the two
principles to help us \gls{solve} quadratic equations that can not be factored. We begin
this module with a review of factoring: 
\begin{enumerate}
	\item Use the zero-product principle
	\item Solve quadratic equations by factoring
	\item Solve problems involving quadratic equations
\end{enumerate} 
As a reminder, remember that equations that can be written in the form
\[
	ax^2 + bx+c =0
\]
where $a\ne 0$, $b$, and $c$ are real numbers, are called {\em Quadratic Equations} in $x$. A 
quadratic \gls{equation} in $x$ is also called a second \gls{degree} \gls{polynomial} equation in $x$.

\begin{myexample}
Solve
\begin{equation}\label{equn:reviewquad}
	x^2-7x+10 = 0
\end{equation}
\end{myexample}
\begin{myProof}
	As we have previously mentioned, because of the $x^2$ term this is not a \gls{linear} equation. If we factor the left side
	of the equation, then we obtain
	\[
		(x-5)(x-2) = 0
	\]
	We now have a product that equals 0. If a quadratic equation has a zero on one side and a factored \gls{expression} (multiplication) on 
	the other, it can be solved by the Zero product principle which we state as a reminder:
		
	\begin{myDefinition}
		{Zero product principle}: 
		This rule says that if the product of two numbers is 0, then either the first one equals 0, or the second one
		equals 0, or they both equal 0. 
				
		Mathematically, we say that if
		\[
			AB=0
		\]
		then either $A=0$ or $B=0$.
	\end{myDefinition}
		
	According to the zero product principle, this product can be zero only if at least one of the factors is equal
	to 0. We set each individual factor equal to 0, and solve each resulting equation for $x$. 
	Therefore
	\[
		x-5 = 0 \qquad {\mathrm{or}}\qquad  x-2=0
	\]
	and so either
	\[
		x=5 \qquad {\mathrm{or}} \qquad x=2
	\]
	We have 2 proposed solutions: $x=5$ and $x=2$. We need to check that both of these values satisfy \cref{equn:reviewquad},
	which we leave as an exercise for the reader.
\end{myProof} 

\subsection{Summary of Solving Quadratic equations by factoring}
\begin{steps}
	\item If necessary, rewrite the equation in the form $ax^2+bx+c=0$ by moving all terms to one side, thereby obtaining
	zero on the other side.
	\item  Factor
	\item Apply the zero-product principle, setting each factor equal to zero.
	\item Solve the equations in step 3.
	\item Check the solutions in the original equation. 
\end{steps} 
Please see the work in our earlier modules for more review.

\section{Solving quadratic equations using the square root property}
\textref{10.1}{600}%
We have so far considered solving quadratic equations by factoring. We may encounter quadratic equations
that can not be factored, and so we will need some alternative approaches. In this section we will discuss the square root property
and Pythagoras` Theorem. We will include examples to help describe each.
\begin{myDefinition}
	The square root property: If $x^2=d$, where $d>0$ is some real number, then
	\[
		x=\sqrt{d} \qquad or \qquad x=-\sqrt{d}
	\]
	{}
	We will use the square root property to help us solve quadratic equations.
\end{myDefinition}

\begin{myexample}
Solve
\[
	x^2-9=0
\]
{}
\end{myexample}
\begin{myProof}
	Notice firstly that this is a quadratic equation, and the left hand side is the difference of 2 perfect squares.
	Therefore, we could factor the left hand side to obtain
	\[
		(x-3)(x+3) = 0
	\]
	and conclude, by the zero product principle, that $x=\pm 3$. However, our intention in this example is to use 
	the square root property. 
		
	We begin by isolating $x^2$ which we do by adding $9$ to both sides
	\[
		x^2=9
	\]
	and now by the square root property, 
	\[
		x = \sqrt{9} \qquad or \qquad x=-\sqrt{9}
	\]
	Since $9$ is a perfect square, we can conclude that
	\[
		x = 3 \qquad or \qquad x=-3
	\]
	{}
\end{myProof} 

\begin{myexample}
Solve 
\[
	x^2-5=0
\]
{}
\end{myexample}
\begin{myProof}
	We proceed on the same lines as in the previous example by first isolating the square term
	\[
		x^2=5
	\]
	We can now solve this equation by using the square root property to obtain
	\[
		x = \pm \sqrt{5}
	\]
	{}
\end{myProof} 

\begin{myexample}
Solve the following quadratic equation using the square root property
\[
	(x+3)^2 = 13
\]
{}
\end{myexample}
\begin{myProof}
	Notice that the squared term is already isolated, so we can use the square root property
	to obtain
	\[
		x+3 = \pm \sqrt{13}
	\]
	At this stage we can subtract 3 from both sides, and conclude that
	\[
		x = -3 \pm \sqrt{13}
	\]
	{}
\end{myProof} 
\begin{myexample}
Solve the following quadratic equation using the square root property
\[
	(2x-1)^2 - 60=0
\]
{}
\end{myexample}
\begin{myProof}
	We first need to isolate the squared term, which we do by adding 60 to both sides
	\[
		(2x-1)^2=60
	\]
	We can now use the square root property to obtain
	\begin{equation}\label{equn:squarerootprop}
		2x-1 = \pm \sqrt{60}
	\end{equation}
	and we need to simplify the right hand side- we note that we can write $\sqrt{60}$ as
	\begin{align*}
		\sqrt{60} & =  \sqrt{4\cdot 15} \\
		          & =  2\sqrt{15}       
	\end{align*} 
	We can now continue to solve \cref{equn:squarerootprop} for $x$ by first adding 1 to both sides
	\[
		2x = 1 \pm 2\sqrt{15}
	\]
	and finally dividing both sides by 2
	\begin{align*}
		x & =   \frac{1\pm 2\sqrt{15}}{2} \\
		  & =  \frac{1}{2}\pm \sqrt{15}   
	\end{align*} 
\end{myProof}

\begin{myexample}
\drillandskill
Solve the following quadratic equations, reducing radicals as much as possible.
\end{myexample}

{\em Straight forward}:
\begin{multicols}{4}
	\begin{enumerate}
		\item $x^2 = 49$  \solution{$\{\pm 7\}$}
		\item $x^2 = 81$  \solution{$\{\pm 9\}$}
		\item $x^2 = 121$ \solution{$\{\pm 11\}$}
		\item $x^2 = 144$ \solution{$\{\pm 12\}$}
	\end{enumerate}
\end{multicols}

{\em Slightly harder}:
\begin{multicols}{2}
	\begin{enumerate}
		\item $2x^2 = 34$ \solution{$\{\pm\sqrt{17}\}$}
		\item $9x^2 = 25$ \solution{$\left\{\pm\frac{5}{4}\right\}$}
		\item $7y^2 = 49$ \solution{$\{\pm\sqrt{7}\}$}
		\item $8w^2 = 36$ \solution{$\left\{\pm\frac{3\sqrt{2}}{2}\right\}$}
	\end{enumerate}
\end{multicols}

{\em More involved}:
\begin{multicols}{2}
	\begin{enumerate}
		\item $(3x-5)^2 = 45$\solution{$\left\{\frac{5}{3}\pm\sqrt{5}\right\}$}
		\item $(4x+1)^2 = 100$\solution{$\left\{\frac{9}{4},-\frac{11}{4}\right\}$}
		\item $(1-2w)^2 = 80$\solution{$\left\{\frac{1}{2}\pm\sqrt{5}\right\}$}
		\item $(1+3y)^2 = 75$\solution{$\left\{-\frac{1}{3}\pm \frac{5}{3}\sqrt{3}\right\}$}
	\end{enumerate}
\end{multicols}


\subsection{Pythagoras' Theorem}
Consider the following right angled triangle
\begin{center}
	\begin{tikzpicture}[inner sep=0mm,scale=3]
		\node (a) at (0,0) {};
		\node (b) at (1,0) {};
		\node (c) at (0,1) {};
		\draw rectangle (.15,.15);
		\draw (a)--(b) node [midway,below left] {$c$};
		\draw (b)--(c) node [midway,above right] {$a$};
		\draw (c)--(a) node [midway,left] {$b$};
	\end{tikzpicture}
\end{center}
Then the following relationship is true
\[
	a^2=b^2+c^2	
\]
The side $a$ opposite the right angle (illustrated with a square on the diagram) is called the 
{\em hypotenuse}. 

There are many proofs of this theorem, some using more advanced mathematics than others. It is a useful
theorem in many applications, and will usually require us to use the square root property.

\begin{myexample}
A baseball diamond is a square with 90 foot sides. Find the distance from home plate to second base.
\end{myexample}
\begin{myProof}
	A typical baseball diamond is shown below
	\begin{center}
		\begin{tikzpicture}
			\node (mycenter) at (0,0) {};
			\node[draw,rectangle,fill=red] (b1) [left = of mycenter] {};
			\node[draw,rectangle,fill=red] (b2) [below = of mycenter] {};
			\node[draw,rectangle,fill=red] (b3) [right = of mycenter] {};
			\node[draw,rectangle,fill=red] (b4) [above= of mycenter] {};
			%\draw (b1)--(b2)--(b3)--(b4)--(b1);
			\node [node distance=0cm] (b11) [left=of b1] {Home plate};
			\node [node distance=0cm] (b21) [below=of b2] {First base};
			\node [node distance=0cm] (b31) [right=of b3] {Second base};
			\node [node distance=0cm] (b41) [above=of b4] {Third base};
			\draw[<->] (b1)--(b2) node [midway,below left] {90ft};
			\draw[<->] (b2)--(b3) node [midway,below right] {90ft};
			\draw[<->] (b3)--(b4) node [midway,above right] {90ft};
			\draw[<->] (b4)--(b1) node [midway,above left] {90ft};
			\draw[dotted] (b1) -- (b3) node [midway] {$x$};
		\end{tikzpicture}
	\end{center}
		
	Let $x$ represent the distance, in feet, from home plate to 2nd base. Using Pythagoras' Theorem
	\[
		90^2+90^2 = x^2
	\]
	and therefore
	\[
		16200 = x^2
	\]
	Using the square root property, we see that
	\[
		x = \pm \sqrt{16200}
	\]
	Since $x$ represents distance, we reject the negative \gls{solution} and conclude that
	\begin{align*}
		x & =  \sqrt{16200} \\
		  & \approx  127.3  
	\end{align*} 
	The distance from home plate to 2nd base is approximately 127.3 feet.
\end{myProof} 

\begin{myexample}
An object is dropped from a building that is 100ft high, and its height $h$ (in feet) at time $t$ (in seconds)
is given by
\[
	h = -16t^2+100
\] 
At what time does the object hit the ground?
\end{myexample}
\begin{myProof}
	When the object hits the ground, the height of the object is 0. This means that we have to solve the equation
	\[
		-16t^2+100 = 0
	\]
	To solve this equation for $t$ we subtract 100 from both sides to give
	\[
		-16t^2=-100
	\]
	On dividing by $-16$, we achieve
	\[
		t^2 = \frac{25}{4}
	\]
	from which we can use the square root property to give
	\[
		t = \pm \frac{5}{2}
	\]
	When dealing with application problems such as this, we have to ask ourselves if there are two possible
	values for $t$. In this case, $t$ represents time, and since we can not have negative time we reject the negative
	solution. 
		
	Therefore, the object hits the ground when $t=\frac{5}{2}$ seconds. The check is left as an exercise for the reader.
\end{myProof}

%+*** mainfile.tex
% arara: pdflatex: { files: [ mainfile.tex ] }
% !arara: indent: { overwrite: on, trace: yes, localSettings: on}
\chapter{The quadratic formula and functions}
\minitoc
\section{The Quadratic Formula}
\textref{10.3}{614}%
We have so far solved some \gls{quadratic} equations by factoring, and using the square root property. There
are two other options: completing the square (which you will discuss in Math 95), and the quadratic
formula, which is the focus of this section.

\begin{myDefinition}
	Given a quadratic \gls{equation}
	\[
		ax^2 + bx + c = 0 \qquad (a\ne 0)
	\]
	then it can be shown that the two solutions are
	\[
		x = \frac{-b+\sqrt{b^2-4ac}}{2a} \qquad or \qquad x = \frac{-b-\sqrt{b^2-4ac}}{2a}
	\]
	These solutions can be written more efficiently as
	\[
		x = \frac{-b\pm \sqrt{b^2-4ac}}{2a}
	\]
	This is known as the {\em quadratic formula.}
\end{myDefinition} 

There will be at most two real solutions to a quadratic equation, and the quadratic formula will work for {\em any} quadratic
equation. We will demonstrate how to use it in this section, and give examples that have two real solutions, one real \gls{solution}, 
and no real solutions.

\begin{myexample}
\Gls{solve}
\[
	2x^2+9x-5=0
\]
using the quadratic formula.
\end{myexample}
\begin{myProof}
	We begin by identifying the coefficients $a$, $b$, and $c$. This is easily done by remembering that $a$ is
	the \gls{coefficient} of the $x^2$ term, $b$ is the coefficient of the $x$ term, and $c$ is the constant coefficient. 
	Therefore, in this example
	\[
		a = 2, \qquad b=9, \qquad c = -5
	\]
	Remember that $a$, $b$, and $c$ will always be {\em numbers}, and we are trying to find values of $x$ that satisfy the 
	given equation.
				
	Note that the directions ask us to solve the equation using the quadratic formula- it may help you to write out the 
	quadratic formula before starting each problem, and it is well worth memorizing for this class, and for all future Math classes. Take care
	to extend the fraction bar the entire length of the numerator
	\begin{align*}
		x & =  \frac{-b\pm \sqrt{b^2-4ac}}{2a}                                     \\
		  & =  \frac{-9\pm\sqrt{9^2-4(2)(-5)}}{2(2)}                               \\
		  & =  \frac{-9\pm \sqrt{81-(-40)}}{4}                                     \\
		  & =  \frac{-9\pm\sqrt{121}}{4}                                           \\
		  & =  \frac{-9+11}{4} = \frac{1}{2} \qquad or\qquad  \frac{-9-11}{4} = -5 
	\end{align*} 
	The solution set to the given quadratic equation is
	\[
		\left\{\frac{1}{2}, -5\right\}
	\]
	The check is left as an exercise.
				
	Notice in this example that the problem can also be solved by factoring
	\[
		(2x-1)(x+5)=0
	\]
	which gives the solution set
	\[
		\left\{\frac{1}{2}, -5\right\}
	\]
	as before.
\end{myProof} 

\begin{myexample}\label{ex:quadform1}
Solve the following quadratic equation
\[
	x^2=9 
\]
using the quadratic formula.
\end{myexample}
\begin{myProof}
	Note that we could solve this equation by factoring, or by using the square root property, however
	the problem asks us to use the quadratic formula. In order to do so we first need to set the equation
	equal to 0, which we do by subtracting 9 from both sides
	\[
		x^2-9 = 0
	\]
				
	We now need to identify the values of $a$, $b$, and $c$; in our example we see that
	\[
		a = 1, \qquad b=0, \qquad c = -9
	\] 
	Following along the same lines as in the previous example, we have
	\begin{align*}
		x & =  \frac{-b\pm \sqrt{b^2-4ac}}{2a}       \\
		  & =  \frac{-0\pm\sqrt{0^2-4(1)(-9)}}{2(1)} \\
		  & =  \frac{0\pm\sqrt{36}}{2}               \\
		  & = \pm\frac{6}{2}                         \\
		  & = \pm 3                                  
	\end{align*} 
	The solution set to the equation is
	\[
		{-3,3}
	\]
	As in the previous example, you should check these solutions in the original equation (one at a time). 
				
	Note that if you had solved this problem by factoring, then the first step would have been to write
	\[
		(x-3)(x+3)=0
	\]
	from which it follows that the solution set is as before.
				
\end{myProof} 
\begin{myDefinition}
	When using the square root property or the quadratic formula, do not forget that
	there are two solutions because of the $\pm$ sign.
\end{myDefinition} 

\begin{myexample}
Solve
\[
	x^2+2x+1=0
\]
using the quadratic formula.
\end{myexample}
\begin{myProof}
	Note that we have so far encountered only examples that have two real solutions. This example gives a slightly
	different result
	\begin{align*}
		x & =  \frac{-b\pm \sqrt{b^2-4ac}}{2a}      \\
		  & =  \frac{-2\pm\sqrt{2^2-4(1)(1)}}{2(1)} \\
		  & =  \frac{-2\pm\sqrt{0}}{2}              \\
		  & =  -1                                   
	\end{align*}
	This example is interesting, as we have only {\em one real solution}. In this case, we say that we have one real
	repeated solution.
				
	Note that if you had solved the equation by factoring the first step would have been to write the equation as
	\[
		(x+1)^2=0
	\]
	from which it follows that $x=-1$ twice, as before.
\end{myProof} 

\begin{myexample}
Solve
\[
	x^2=x+6
\]
using the quadratic formula.
\end{myexample}
\begin{myProof}
	We begin by setting the equation equal to 0,
	\[
		x^2-x-6=0
	\]
	which allows us to identify that
	\[
		a = 1, \qquad b = -1, \qquad c = -6
	\]
	We proceed as in the previous examples
	\begin{align*}
		x & =  \frac{-b\pm \sqrt{b^2-4ac}}{2a}        \\
		  & =  \frac{1\pm\sqrt{(-1)^2-4(1)(6)}}{2(1)} \\
		  & =  \frac{1\pm \sqrt{1-24}}{2}             \\
		  & =  \frac{1\pm\sqrt{-23}}{2}               
	\end{align*}
	{}
	Notice here that we have ended with an \gls{expression} containing the square root of a negative number. 
	The result of taking the square root of a negative number is {\em not } real; in this case, 
	we therefore say that the equation has {\em no real solutions}.
\end{myProof} 

\section{Relations and functions}
\textref{10.6}{642}%
The concept of {\em relations} and {\em functions} is fundamental to mathematics. In fact, we have
been dealing with functions throughout the class so far, it's just that we have not referred to them as this.
\begin{itemize}
	\item A {\em relation} is a rule or set of rules that maps an element in the 
	domain to an element in the range.
	\item A {\em function} is a relation that maps an element in the domain onto 
	a single element in the range.
\end{itemize} 

\begin{center}
	\begin{tikzpicture}
		\filldraw [yellow,draw=black] (0,0) circle (1) node (mydomain){};
		\node (d1) at (0,.5) {$1$};
		\node (d2) at (0,-.5){$4$};
		\draw  (3,0) circle (1);
		\node (r1) at (3,.75) {$1$};
		\node (r2) at (3,.25){$1$};
		\node (r3) at (3,-.25){$2$};
		\node (r4) at (3,-.75){$2$};
		\draw[->] (d1)--(r1);
		\draw[->] (d1)--(r2);
		\draw[->] (d2)--(r3);
		\draw[->] (d2)--(r4);
	\end{tikzpicture}
	\begin{tikzpicture}
		\filldraw [yellow,draw=black] (0,0) circle (1) node (mydomain){};
		\node (d1) at (0,.7) {$1$};
		\node (d2) at (0,0){$2$};
		\node (d3) at (0,-.7){$3$};
		\draw  (3,0) circle (1);
		\node (r1) at (3,.75) {$1$};
		\node (r2) at (3,.25){$1$};
		\node (r3) at (3,-.25){$2$};
		\node (r4) at (3,-.75){$2$};
		\draw[->] (d1)--(r1);
		\draw[->] (d2)--(r3);
		\draw[->] (d3)--(r3);
	\end{tikzpicture}
\end{center}

\begin{itemize}
	\item The domain of a function is the {\em input} set. In the above, the domain of the function $F$ is $\{1,2,3\}$.
	\item The range of a function is the {\em output} set. In the above, the range of the function $F$ is $\{1,-1,2,2\}$.
\end{itemize} 
Notice the distinction between a {\em relation} and a {\em function}. A function takes each element from the input set and
maps it into only one element in the output set- the relation $N$ is not a function as the number 1 goes to both 1 and -1 (notice
also that 4 goes to 2 and -2).

\subsection{Function notation}
We represent a function algebraically using the notation $f(x)$. For example, we can define the \gls{linear} function
\[
	f(x)=2x+3
\]
\begin{itemize}
	\item we pronounce $f(x)$ as `$f$ of $x$', and say that $f$ is a function of $x$
	\item $x$ is the input \gls{variable}, or more commonly known as the {\em independent variable}
\end{itemize} 
Let's say that we want to evaluate this function at $x=4$. To do this we write
\begin{align*}
	f(4) & =		2(4)-3 \\
	     & =		8-3    \\
	     & =		5      
\end{align*} 
Notice that we write the \gls{point} that we want to evaluate in the parenthesis on the left hand side, 
and replace each occurrence of $x$ with 4 on the right hand side.

We may also write $y=2x-3$, in which case we still say that $x$ is the independent variable, 
and that $y$ is the {\em dependent} variable. 

\begin{myexample}
Let $f$ be the function that has formula
\[
	f(x)=x^2+2
\]
Evaluate 
\begin{multicols}{3}
	\begin{enumerate}
		\item $f(2)$
		\item $f(0)$
		\item $f(-4)$
	\end{enumerate} 
\end{multicols}
\end{myexample}
\begin{myProof}
	\begin{enumerate}
		\item 
		$\begin{aligned}[t]
			f(2) & =		2^2+2 \\
			     & =		4+2   \\
			     & =		6     
		\end{aligned}$
		\item 
		$\begin{aligned}[t]
			f(0) & =		0^2+2 \\
			     & =		0+2   \\
			     & =		2     
		\end{aligned}$
		\item 
		$\begin{aligned}[t]
			f(-4) & =		(-4)^2+2 \\
			      & =		16+2     \\
			      & =		18       
		\end{aligned}$
	\end{enumerate} 
\end{myProof} 

\begin{myexample}
\drillandskill
Let
\[
	f(x) = 4x-3
\]
Find each of the following
\begin{multicols}{4}
	\begin{enumerate}
		\item $f(2)$ \solution{$=5$}
		\item $f(0)$\solution{$=-3$}
		\item $f(1)$\solution{$=1$}
		\item $f(3)$\solution{$=9$}
	\end{enumerate}
\end{multicols}
\end{myexample}

\begin{myexample}
\drillandskill
Let $g$ be the function that has formula
\[
	g(x) = x^2 + 2x +3 
\]
Find each of the following
\begin{multicols}{4}
	\begin{enumerate}
		\item $g(2)$\solution{$=11$}
		\item $g(0)$\solution{$=3$}
		\item $g(1)$\solution{$=6$}
		\item $g(3)$\solution{$=18$}
	\end{enumerate}
\end{multicols}
\end{myexample}

\begin{myexample}
Let's interpret each of the following functions verbally
\begin{description}
	\item $y=x+2$      $y$ is equal to $x$ plus 2          	
	\item $f(x)=x+2$   $f$ of $x$ is equal to $x$ plus 2   
	\item $u=3t-5$     $u$ is equal to $3t$ minus 5        
	\item $g(t)=3t-5$  $g$ of $t$ is equal to $3t$ minus 5 
\end{description}
\end{myexample}

\begin{myexample}
The law firm of BDD charges an account set up fee of \$1,000.00 and charges a consultation
rate of \$100.00 per hour. Find $B(t)$, the cost of dealing with this firm as a function of time, $t$
in hours. 

Evaluate $B(1)$, $B(2)$, and $B(3)$.
\end{myexample}
\begin{myProof}
	We recognize that this is an example of a linear function, and we can write
	\[
		B(t)=1000+100t
	\]
	where $t\geq 0$. We evaluate the function by using the above method of replacing the independent
	variable with whatever is in the parenthesis.
	\begin{align*}
		B(1) & =	 1000+100(1) & B(2) & =	 1000+100(2) & B(3) & =	 1000+100(3) \\
		     & =	 1000+100    &      & =	 1000+200    &      & =	 1000+300    \\
		     & =	 1100        &      & =	 1200        &      & =	 1300        
	\end{align*} 
	These represent the cost of setting up an account that takes 1 hour, 2 hours, and 3hours
	respectively.
\end{myProof} 


\begin{myexample}
Let $f$ be the function that has formula
\[
	f(x) = 4x-2
\]
which is shown in \cref{fig:linfunctioneval}. Find the following
\begin{multicols}{3}
	\begin{enumerate}
		\item $f(0)$
		\item $f(1)$
		\item $f(-3)$
	\end{enumerate} 
\end{multicols}
{}
\end{myexample}
\begin{figure}[!h]
	\centering
	\begin{tikzpicture}
		\begin{axis}[
				framed,
				xmin=-10,xmax=10,
				ymin=-10,ymax=10,
				xtick={-8,-6,...,8},
				minor xtick={-9,-7,...,9},
				ytick={-8,-6,...,8},
				minor ytick={-9,-7,...,9},
				grid=both,
			]
			\addplot expression[domain=-2:3,samples=100]{4*x-2};
			\legend{$f(x)=4x-2$};
		\end{axis}
	\end{tikzpicture}
	\caption{$f(x)=4x-2$}
	\label{fig:linfunctioneval}
\end{figure}
\begin{myProof}
	\begin{enumerate}
		\item To find $f(0)$ we are looking for the $y$ value when $x=0$; we go to 0 on the 
		horizontal axis, and then go down (in this case) until we hit the line. We conclude that
		\[
			f(0)=-2
		\]
		\item To find $f(1)$ we are looking for the $y$ value when $x=1$; we go to 1 on the 
		horizontal axis, and then go up (in this case) until we hit the line. We conclude that
		\[
			f(1) = 2
		\]  
		\item To find $f(-3)$ we proceed as in the previous parts of the problem. However, we see immediately
		that we will not be able to use \cref{fig:linfunctioneval}, because the $y$ value corresponding
		to $x=-3$ is not shown. We therefore find $f(-3)$ algebraically as follows
		\begin{align*}
			f(-3) & =  4(-3)-2 \\
			      & =  -12-2   \\
			      & =  -14     
		\end{align*} 
	\end{enumerate} 
	{}
\end{myProof} 
\FloatBarrier

\begin{myexample}\label{ex:quadraticeval}
The function $g$ has formula
\[
	g(x) = 5-x
\]
and is shown in \cref{fig:linfunctionevalfindx}. Find the values of $x$ for which
\begin{multicols}{2}
	\begin{enumerate}
		\item $g(x)=4$
		\item $g(x)=-1$
	\end{enumerate} 
\end{multicols}
{}
\end{myexample}
\begin{figure}[!h]
	\centering
	\begin{tikzpicture}
		\begin{axis}[
				framed,
				xmin=-10,xmax=10,
				ymin=-10,ymax=10,
				xtick={-8,-6,...,8},
				minor xtick={-9,-7,...,9},
				ytick={-8,-6,...,8},
				minor ytick={-9,-7,...,9},
				grid=both,
			]
			\addplot expression[domain=-5:10,samples=100]{5-x};
			\legend{$g(x)=5-x$};
		\end{axis}
	\end{tikzpicture}
	\caption{$g(x)=5-x$}
	\label{fig:linfunctionevalfindx}
\end{figure}
\begin{myProof}
	\begin{enumerate}
		\item We want to find the value of $x$ for which $g(x)=4$. We are given the $y$ value, and need to find the $x$ value, so we 
		proceed by going to 4 on the $y$ axis, and then go horizontally until we hit the line- we then go down (in this case) until
		we hit the horizontal axis. This gives that
		\[
			g(1)=4
		\]
		An alternative approach to this problem is to solve the equation
		\[
			4 = 5-x
		\]  
		which also gives that $x=1$.
		\item Again, we are given the $y$ value, and we want to find the $x$ value. As before, we go to $-1$ on the $y$ axis, and then
		go horizontally until we hit the line- we then go up (in this case) until we hit the horizontal axis. This gives
		\[
			g(6)=-1
		\]
		An algebraic approach to this problem is to solve the equation
		\[
			-1 = 5 - x
		\]
		which gives $x=6$.
	\end{enumerate} 
	{}
\end{myProof} 

\begin{myexample}
Let $h$ be the function that has formula
\[
	h(x) = (x-2)(x+3)
\]  
which is shown in \cref{fig:quadraticeval}. Find the $x$ values for which
\begin{multicols}{2}
	\begin{enumerate}
		\item $g(x)=0$
		\item $g(x)=6$
	\end{enumerate} 
\end{multicols}
\end{myexample}
\begin{figure}[!h]
	\centering
	\begin{tikzpicture}
		\begin{axis}[
				framed,
				xmin=-10,xmax=10,
				ymin=-10,ymax=10,
				xtick={-8,-6,...,8},
				minor xtick={-9,-7,...,9},
				ytick={-8,-6,...,8},
				minor ytick={-9,-7,...,9},
				grid=both,
			]
			\addplot expression[domain=-4.4:3.4,samples=100]{(x-2)*(x+3)};
			\legend{$h(x)=5-x$};
		\end{axis}
	\end{tikzpicture}
	\caption{$h(x)=(x-2)(x+3)$}
	\label{fig:quadraticeval}
\end{figure}
\begin{myProof}
	We use exactly the same approach that we used in \cref{ex:quadraticeval}. 
	\begin{enumerate}
		\item We begin by going to 0 on the $y$  axis, and then go horizontally until we hit the function. Notice that in 
		this case there are {\em 2} values of $x$, and we have that
		\[
			g(-3) = 0, \qquad g(2)=0
		\]
		We could approach this problem algebraically by solving the equation
		\[
			0 = (x-2)(x+3)
		\]
		which gives $x=2, -3$ as before.
		\item We go to 6 on the $y$ axis, and then go horizontally until we hit the function. Again, there are 2 values of $x$, 
		and we have that
		\[
			g(-4) = 6, \qquad g(3)=6
		\]
		Approaching this problem algebraically would require us to solve the equation
		\[
			6 = (x-2)(x+3)
		\]
		We would begin by FOILing the right hand side
		\[
			6 = x^2+x - 6
		\]  
		and therefore, on subtracting 6 from both sides, obtain
		\[
			0 = x^2 +x -12
		\]
		The right hand side can be factored
		\[
			0 = (x+4)(x-3)
		\]
		and therefore $x=-4,3$ as before.
	\end{enumerate} 
	{}
\end{myProof} 

\begin{myexample}
\drillandskill
Consider the function $f$ given below. Complete the given table.

\begin{minipage}{.5\textwidth}
	\centering
	\begin{tikzpicture}
		\begin{axis}[
				framed,
				xmin=-5,xmax=5,
				ymin=-5,ymax=5,
				grid=major,
				xtick={-4,...,4},
				ytick={-4,...,4},
				width=.9\textwidth,
			]
			\addplot expression[domain=-3:2,samples=100]{2*x+1};
		\end{axis}
	\end{tikzpicture}
\end{minipage}
\begin{minipage}{.5\textwidth}
	\begin{tabular}{ccl}
		\toprule
		$x$            & $f(x)=2x+1$     &                       \\
		\midrule
		$-3$           & \solution{$-5$} & $f(-3)=\solution{-5}$ \\
		$0 $           & \solution{$1$}  & $f(0) =\solution{1} $ \\
		\solution{$1$} & $3$             & $f(\solution{1})=3$   \\
		\solution{$2$} & $5$             & $f(\solution{2})=5$   \\
		\bottomrule
	\end{tabular}
\end{minipage}

\end{myexample}

\begin{myexample}
Consider the function $f$ given below. Complete the given table.

\begin{minipage}{.5\textwidth}
	\centering
	\begin{tikzpicture}
		\begin{axis}[
				framed,
				xmin=-5,xmax=5,
				ymin=-5,ymax=8,
				xtick={-4,...,4},
				ytick={-4,-2,...,6},
				minor ytick={-3,-1,...,7},
				grid=both,
				width=.9\textwidth,
			]
			\addplot expression[domain=-3.1:2.2,samples=100]{x*(x+1)};
		\end{axis}
	\end{tikzpicture}
\end{minipage}
\begin{minipage}{.5\textwidth}
	\begin{tabular}{ccl}
		\toprule
		$x$               & $f(x)=x(x+1)$ &                       \\
		\midrule
		$-3$              & \solution{6}  & $f(-3)=\solution{6}$  \\
		$0 $              & \solution{0}  & $f(0) = \solution{0}$ \\
		\solution{$-2,1$} & $2$           & $f(-2)=f(1)=2$        \\
		\solution{$-3,2$} & $6$           & $f(-3)=f(2)=6$        \\
		\bottomrule
	\end{tabular}
\end{minipage}

\end{myexample}


\subsection{Domain and range}
\begin{myexample}\label{ex:firstlinear}
Find the domain and range of the function $f$ that has formula
\[
	f(x) = 4x - 3
\]
{}
\end{myexample}
\begin{myProof}
	The function $f$ is shown in \cref{fig:linfunction}. 
	The domain is the set of all input values, and the range is the set of all output values.
	Graphically this means that the domain is the set of all $x$ values, and the range is the 
	set of all $y$ values.
				
	\begin{figure}[!h]
		\centering
		\begin{tikzpicture}
			\begin{axis}[
					framed,
					xmin=-5,xmax=5,
					ymin=-5,ymax=5,
					xtick={-4,...,4},
					ytick={-4,...,4},
					grid=major,
					legend pos = north west,
				]
				\addplot expression[domain=-.5:2,samples=100]{4*x-3};
				\legend{$f(x)=4x-3$};
			\end{axis}
		\end{tikzpicture}
		\caption{$f(x)=4x-3$}
		\label{fig:linfunction}
	\end{figure}
				
	This function is defined for all values of $x$, so we say that the domain is
	\[
		(-\infty, \infty )
	\]  
	Similarly, we can achieve all $y$ values, so we say that the range is
	\[
		(-\infty, \infty)
	\]
				
\end{myProof} 

\begin{myexample}
The observations that we have made for the linear function in \cref{ex:firstlinear} can be generalized 
to any linear function
\[
	f(x) = mx +b
\]
Find the domain and range of this function.
{}
\end{myexample}
\begin{myProof}
	The domain of any linear function is
	\[
		(-\infty, \infty )
	\]
	and the range is
	\[
		(-\infty, \infty )
	\]
	{}
\end{myProof} 

\begin{myexample}
Find the domain and range of the functions $f$ and $g$ that have formulas
\[
	f(x) = (x-2)^2 +3, \qquad g(x) = 5-(x-1)^2
\]  
which are shown in \cref{fig:quadraticrange}.
\end{myexample}
\begin{figure}[!h]
	\begin{subfigure}{.5\textwidth}
		\centering
		\begin{tikzpicture}
			\begin{axis}[
					framed,
					xmin=-5,xmax=5,
					ymin=-5,ymax=10,
					grid=both,
					xtick={-4,...,4},
					ytick={-4,-2,...,8},
					minor ytick={-3,-1,...,9},
					legend pos=north west,
					width=.9\textwidth,
				]
				\addplot expression[domain=-.5:4.5,samples=100]{(x-2)^2+3};
				\legend{$y=f(x)$}
			\end{axis}
		\end{tikzpicture}
		\caption{$f$}
	\end{subfigure}%
	\begin{subfigure}{.5\textwidth}
		\centering
		\begin{tikzpicture}
			\begin{axis}[
					framed,
					xmin=-5,xmax=7,
					ymin=-5,ymax=5,
					grid=major,
					xtick={-4,...,6},
					ytick={-4,...,4},
					width=.9\textwidth,
				]
				\addplot expression[domain=-2:4,samples=100]{5-(x-1)^2};
				\legend{$y=g(x)$}
			\end{axis}
		\end{tikzpicture}
		\caption{$g$}
	\end{subfigure}
	\caption{The functions $f$ and $g$.}
	\label{fig:quadraticrange}
\end{figure}

\begin{myProof}
	\begin{enumerate}
		\item The domain of $f$ is the set of all possible input values. We can input any $x$ values into $f$, so we say
		that the domain of $f$ is $(-\infty, \infty)$. The range of $f$ is the set of all possible output values; notice
		that the lowest $y$ value that $f$ has is $y=3$, and that it can take all $y$ values above this. We therefore say
		that the range of $f$ is $[3,\infty)$.
			\item The domain of $g$ is $(-\infty, \infty)$. To find the range, notice that the highest value of $g(x)$ is $y=5$, and
		that it can take all values below this. We therefore say that the domain of $g$ is $(-\infty, 5]$.
	\end{enumerate} 
	In general, the domain of any given quadratic function will be $(-\infty, \infty)$, and the range will depend on the shape
	of the parabola, and the position of the \gls{vertex}- we will study more on this in the next module.
	{}
\end{myProof} 

\subsection{Vertical line test}
If a vertical line can be drawn so that it intersects the graph of a relation at more
than one point, then the relation is not a function.

\begin{myexample}
Assume that $x$ is the independent variable and $y$ is the dependent variable. Use the vertical
line test to determine if the following relation is a function
\[
	y = x^2-2
\]
{}
\end{myexample}
\begin{myProof}
	The graph of this function is given in \cref{fig:verticallinetest}
	\begin{figure}[!ht]
		\centering
		\begin{tikzpicture}
			\begin{axis}[
					framed,
					xmin=-5,xmax=5,
					ymin=-5,ymax=5,
					xtick={-4,...,4},
					ytick={-4,...,4},
					grid=major,
					legend pos=south east,
				]
				\addplot expression[domain=-2.5:2.5,samples=100]{x^2-2};
				\legend{$y=x^2-2$};
			\end{axis}
		\end{tikzpicture}
		\caption{$f(x)=x^2-2$}
		\label{fig:verticallinetest}
	\end{figure}
	We notice that any vertical line will cut the graph only once. This relation is therefore a function.
\end{myProof} 

%+*** mainfile.tex
% arara: pdflatex: { files: [ mainfile.tex ] }
% !arara: indent: { overwrite: on, trace: yes, localSettings: on}
\chapter{Plotting quadratic functions}
\minitoc

\section{Graphing Quadratic Functions}
\textref{10.5}{630}%
In the previous section we studied the \gls{quadratic} formula
\[
	x = \frac{-b\pm \sqrt{b^2-4ac}}{2a}
\]
which tells us the solutions to the quadratic \gls{equation}
\[
	ax^2+bx+c=0
\]
In this section we will relate these solutions in helping us graph quadratic functions. Throughout 
our work in this module we will also use function notation.

\subsection{Horizontal intercepts}
The graph of a quadratic will {\em always} be a parabola. The parabola will open
\begin{itemize}
	\item upwards if $a>0$
	\item downwards if $a<0$
\end{itemize} 
We will see numerous examples that will help us demonstrate this. 

Remember that in the previous section we demonstrated three possible cases for the
number of solutions to quadratic equations:
\begin{enumerate}
	\item two real solutions if $b^2-4ac>0$
	\item one real \gls{solution} if $b^2-4ac=0$
	\item no real solutions if $b^2-4ac<0$
\end{enumerate} 

In terms of the graph of a quadratic, the points where $ax^2+bx+c=0$ represents the
points when the graph crosses the horizontal axis (if it does at all). In fact, the 
three cases we have just listed correspond precisely to three different behaviours in
the graph- we detail each in the following table.

\begin{longtable}{p{5cm}p{5cm}p{4.5cm}}
	\toprule
	$a>0$ (opens upwards) & $a<0$ (opens downwards) & comments \\ 
	\midrule
	\begin{adjustbox}{valign=t}
		\begin{tikzpicture}
			\begin{axis}[
					width=5cm,
					xmin=-5,xmax=5,
					ymin=-5,ymax=5,
					xtick={-10,...,-9},
					ytick={-10,...,-9},
				]
				\addplot expression[domain=-3:2,samples=100]{(x-1)*(x+2)};
				\addplot[cmhplot,soldot] coordinates{	(-2,0) 	(1,0)	};
			\end{axis}
		\end{tikzpicture}
	\end{adjustbox}
	& 
	\begin{adjustbox}{valign=t}
		\begin{tikzpicture}
			\begin{axis}[
					width=5cm,
					xmin=-5,xmax=5,
					xmin=-5,xmax=5,
					ymin=-5,ymax=5,
					xtick={-10,...,-9},
					ytick={-10,...,-9},
				]
				\addplot expression[domain=-3.19:2.19,samples=100]{-1*(x-1)*(x+2)};
				\addplot[cmhplot,soldot] coordinates{	(-2,0) (1,0)};
			\end{axis}
		\end{tikzpicture}
	\end{adjustbox}
	& 
	$b^2-4ac>0$ 
	\begin{itemize}
		\item {\em two} real solutions
		\item the graph cuts the horizontal axis {\em twice}
	\end{itemize} \\ 
	\begin{adjustbox}{valign=t}
		\begin{tikzpicture}
			\begin{axis}[
					width=5cm,
					xmin=-5,xmax=5,
					xmin=-5,xmax=5,
					ymin=-5,ymax=5,
					xtick={-10,...,-9},
					ytick={-10,...,-9},
				]
				\addplot expression[domain=-1.23:3.23,samples=100]{(x-1)^2};
				\addplot[cmhplot,soldot] coordinates{	(1,0) };
			\end{axis}
		\end{tikzpicture}
	\end{adjustbox}
	&
	\begin{adjustbox}{valign=t}
		\begin{tikzpicture}
			\begin{axis}[
					width=5cm,
					xmin=-5,xmax=5,
					xmin=-5,xmax=5,
					ymin=-5,ymax=5,
					xtick={-10,...,-9},
					ytick={-10,...,-9},
				]
				\addplot expression[domain=-1.23:3.23,samples=100]{-1*(x-1)^2};
				\addplot[cmhplot,soldot] coordinates{	(1,0) };
			\end{axis}
		\end{tikzpicture}
	\end{adjustbox}
	&
	$b^2-4ac=0$ 
	\begin{itemize}
		\item {\em one} real solution
		\item the graph {\em touches} the horizontal axis {\em once}
	\end{itemize} \\ 
	\begin{adjustbox}{valign=t}
		\begin{tikzpicture}
			\begin{axis}[
					width=5cm,
					xmin=-5,xmax=5,
					ymin=-5,ymax=5,
					xtick={-10,...,-9},
					ytick={-10,...,-9},
				]
				\addplot expression[domain=-2:2,samples=100]{x^2+1};
			\end{axis}
		\end{tikzpicture}
	\end{adjustbox}
	&
	\begin{adjustbox}{valign=t}
		\begin{tikzpicture}
			\begin{axis}[
					width=5cm,
					xmin=-5,xmax=5,
					xmin=-5,xmax=5,
					ymin=-5,ymax=5,
					xtick={-10,...,-9},
					ytick={-10,...,-9},
				]
				\addplot expression[domain=-2:2,samples=100]{-1*x^2-1};
			\end{axis}
		\end{tikzpicture}
	\end{adjustbox}
	& 
	$b^2-4ac<0$ 
	\begin{itemize}
		\item {\em no} real solutions
		\item the graph does {\em not} touch the horizontal axis at all
	\end{itemize} \\ 
\end{longtable}

We study each of the 3 cases in the following examples. 

\begin{myexample}
State the coordinates of the horizontal intercepts (as ordered pairs) for each of the given quadratic functions.
\end{myexample}

\begin{minipage}{.5\textwidth}
	\begin{tikzpicture}
		\begin{axis}[
				framed,
				xmin=-5,xmax=5,
				ymin=-5,ymax=5,
				xtick={-4,...,4},
				ytick={-4,...,4},
				grid=major,
				width=\textwidth,
			]
			\addplot expression[domain=-2.23:2.23,samples=100]{x^2-1};
			\addplot[cmhplot,soldot]coordinates{	(1,0) (-1,0) 	};
		\end{axis}
	\end{tikzpicture}
\end{minipage}%
\hfill
\begin{minipage}{.4\textwidth}
	$f(x)=x^2-1$
	
	$x$-intercepts: $(1,0)$, $(-1,0)$
	
	$a>0$, and the parabola opens upwards.
\end{minipage}%

\begin{minipage}{.5\textwidth}
	\begin{tikzpicture}
		\begin{axis}[
				framed,
				xmin=-5,xmax=5,
				ymin=-5,ymax=5,
				xtick={-4,...,4},
				ytick={-4,...,4},
				grid=major,
				width=\textwidth,
			]
			\addplot expression[domain=-2:4,samples=100]{(x-3)*(x+1)};
			\addplot[cmhplot,soldot] 	coordinates{	(3,0) 	(-1,0) 	};
		\end{axis}
	\end{tikzpicture}
\end{minipage}%
\hfill
\begin{minipage}{.4\textwidth}
	$f(x)=(x-3)(x+1)$
	    
	$x$-intercepts: $(3,0)$, $(-1,0)$
	
	$a>0$ and the parabola opens upwards.
\end{minipage}%

\begin{minipage}{.5\textwidth}
	\begin{tikzpicture}
		\begin{axis}[
				framed,
				xmin=-5,xmax=5,
				ymin=-5,ymax=5,
				xtick={-4,...,4},
				ytick={-4,...,4},
				grid=major,
				width=\textwidth,
			]
			\addplot expression[domain=-0.75:3.8,samples=100]{-1*(x-2)*(x-1)};
			\addplot[cmhplot,soldot] coordinates{	(2,0) (1,0) 	};
		\end{axis}
	\end{tikzpicture}
\end{minipage}%
\hfill
\begin{minipage}{.4\textwidth}
	$f(x)=-(x-2)(x-1)$
	    
	$x$-intercepts: $(2,0)$, $(1,0)$
	
	$a<0$ and the parabola opens upwards.
\end{minipage}%

\begin{minipage}{.5\textwidth}
	\begin{tikzpicture}
		\begin{axis}[
				framed,
				xmin=-5,xmax=5,
				ymin=-5,ymax=5,
				xtick={-4,...,4},
				ytick={-4,...,4},
				grid=major,
				width=\textwidth,
			]
			\addplot expression[domain=-0.25:4,samples=100]{-1*(x-2)^2};
			\addplot[cmhplot,soldot] coordinates{	(2,0) };
		\end{axis}
	\end{tikzpicture}
\end{minipage}%
\hfill
\begin{minipage}{.4\textwidth}
	$f(x)=-(x-2)^2$
	
	$x$-\gls{intercept}: $(2,0)$
	
	$a<0$ and the parabola opens downwards.
\end{minipage}%

\begin{minipage}{.5\textwidth}
	\begin{tikzpicture}
		\begin{axis}[
				framed,
				xmin=-5,xmax=5,
				ymin=-5,ymax=5,
				xtick={-4,...,4},
				ytick={-4,...,4},
				grid=major,
				width=\textwidth,
			]
			\addplot expression[domain=-4.75:-1,samples=100]{(x+3)^2+1};
		\end{axis}
	\end{tikzpicture}
\end{minipage}%
\hfill
\begin{minipage}{.4\textwidth}
	$f(x)=(x+3)^2+1$
	
	$x$-intercepts: none
	
	$a>0$ and the parabola opens upwards.
\end{minipage}


\begin{myexample}\label{ex:horizints1}
Find the $x$ intercepts of 
\[
	f(x) = x^2-9
\]
and state your answers as ordered pairs.
\end{myexample}
\begin{myProof}
	The $x$ intercepts are when $y=0$, so we have to \gls{solve} the equation
	\[
		0 = x^2-9
	\]
	We have studied this example previously (see \vref{ex:quadform1}) and know
	that the solutions are 
	\[
		x = \pm 3
	\]
	The horizontal intercepts of the given function are therefore
	\[
		(3,0), \qquad (-3,0)
	\]
	This is seen in \cref{fig:quad}; note in particular that the horizontal intercepts
	are shown with bullets.
			
	\begin{figure}[!ht]
		\centering
		\begin{tikzpicture}
			\begin{axis}[
					framed,
					xmin=-10,xmax=10,
					ymin=-10,ymax=10,
					xtick={-8,-6,...,8},
					minor xtick={-9,-7,...,9},
					ytick={-8,-6,...,8},
					minor ytick={-9,-7,...,9},
					grid=both,
					legend pos=south east,
				]
				\addplot expression[domain=-4.1:4.1,samples=100]{x^2-9};
				\addplot[cmhplot,soldot] coordinates{	(3,0) (-3,0) };
				\legend{$f(x)=x^2-9$};
			\end{axis}
		\end{tikzpicture}
		\caption{The quadratic function $f(x)=x^2-9$; note that there are 2 real solutions.}
		\label{fig:quad}
	\end{figure}
			
\end{myProof}


\begin{myexample}\label{ex:horizints2}
Find the $x$ intercepts of 
\[
	f(x) = x^2+2x+1
\]
and state your answers as ordered pairs.

\end{myexample}
\begin{myProof}
	We need to solve the equation
	\[
		0 = x^2+2x+1
	\]
	which we can do using any of the methods discussed so far. On factoring this \gls{expression} we obtain
	\[
		(x+1)^2=0
	\]
	and therefore $x=-1$ twice. We conclude that there is only one horizontal intercept, and it is at
	\[
		(-1,0)
	\]
	As there is only one $x$ intercept, the curve {\em touches} the horizontal axis at this \gls{point}, as shown 
	in \cref{fig:quad1}.
	\begin{figure}[!ht]
		\centering
		\begin{tikzpicture}
			\begin{axis}[
					framed,
					xmin=-10,xmax=10,
					ymin=-10,ymax=10,
					xtick={-8,-6,...,8},
					minor xtick={-9,-7,...,9},
					ytick={-8,-6,...,8},
					minor ytick={-9,-7,...,9},
					grid=both,
					legend pos=south east,
				]
				\addplot expression[domain=-4:2,samples=100]{x^2+2*x+1};
				\addplot[cmhplot,soldot] coordinates{	(-1,0) };
				\legend{$f(x)=x^2+2x+1$};
			\end{axis}
		\end{tikzpicture}
		\caption{The quadratic function $f(x)=x^2+2x+1$; note that there is only one horizontal intercept, which is located at $(-1,0)$.}
		\label{fig:quad1}
	\end{figure}
	{}
\end{myProof}

\FloatBarrier

\begin{myexample}\label{ex:horizints3}
Find the $x$ intercepts of 
\[
	f(x) = -x^2-2x-4
\]
and state your answers as an ordered pair.
\end{myexample}
\begin{myProof}
	As in the previous examples, to find the horizontal intercepts we need to set the function equal to zero. 
	We therefore need to solve the equation
	\begin{equation}\label{equn:norealsolns}
		0 = -x^2-2x-4
	\end{equation}
	We can solve this equation using any of the techniques that we have discussed so far. It does not
	\gls{factor}, so let's use the quadratic formula
	\begin{align*}
		x & =   \frac{-b\pm \sqrt{b^2-4ac}}{2a}                \\
		  & =   \frac{-(-2)\pm \sqrt{(-2)^2-4(-1)(-4)}}{2(-1)} \\
		  & =  \frac{2\pm\sqrt{4-16}}{-2}                      \\
		  & =  \frac{2\pm \sqrt{-12}}{-2}                      
	\end{align*} 
	{}
	Notice that we have an expression that contains the square root of a negative number, which is not real. 
	We therefore conclude that there are no real solutions to \cref{equn:norealsolns}, and there
	are no horizontal intercepts. This is shown graphically in \cref{fig:quad2}.
			
	\begin{figure}[!ht]
		\centering
		\begin{tikzpicture}
			\begin{axis}[
					framed,
					xmin=-10,xmax=10,
					ymin=-10,ymax=10,
					xtick={-8,-6,...,8},
					minor xtick={-9,-7,...,9},
					ytick={-8,-6,...,8},
					minor ytick={-9,-7,...,9},
					grid=both,
				]
				\addplot expression[domain=-3.4:1.5,samples=100]{-1*x^2-2*x-4};
				\legend{$f(x)=-x^2-2x-4$};
			\end{axis}
		\end{tikzpicture}
		\caption{The quadratic function $f(x)=-x^2-2x-4$; note that there are no horizontal intercepts.}
		\label{fig:quad2}
	\end{figure}
	{}
	\FloatBarrier
\end{myProof}


\subsection{Vertical intercept}
The vertical intercept is the point where the curve cuts the vertical axis. All quadratic functions have a vertical
intercept, and we find it by evaluating $f(0)$.

\begin{myexample}
Find the vertical intercept of
\[
	f(x) = x^2-x+6
\]
{}
\end{myexample}
\begin{myProof}
	We set $x=0$, and obtain
	\begin{align*}
		f(0) & =   (0)^2-0+6 \\
		     & =  6          \\
	\end{align*} 
	The vertical intercept of this function is at $(0,6)$; see \cref{fig:quad3}.
	\begin{figure}[!ht]
		\centering
		\begin{tikzpicture}
			\begin{axis}[
					framed,
					xmin=-5,xmax=5,
					ymin=-2,ymax=20,
					grid=major,
					xtick={-4,...,4},
					ytick={2,4,...,18},
				]
				\addplot expression[domain=-3.27:4.27,samples=100]{x^2-x+6};
				\addplot[cmhplot,soldot] coordinates{	(0,6) };
				\legend{$f(x)=x^2-x+6$};
			\end{axis}
		\end{tikzpicture}
		\caption{The quadratic function $f(x)=x^2-x+6$; note the vertical intercept at $(0,6)$. }
		\label{fig:quad3}
	\end{figure}
	\FloatBarrier
			
\end{myProof} 

\begin{myexample}
Find the vertical intercepts for \cref{ex:horizints1,ex:horizints2,ex:horizints3}
\end{myexample}
\begin{myProof}
	See if you can do this question on your own first! 
			
	The solutions are shown in the footnote\footnote{\cref{ex:horizints1}: $(0,9)$, \cref{ex:horizints2}:$(0,1)$, \cref{ex:horizints3}:$(0,-4)$}. 
\end{myProof}


\subsection{\Gls{vertex} coordinates}
Every quadratic has a vertex and an axis of symmetry. The vertex is the turning point of the parabola, and the axis of symmetry
is the line about which the graph is symmetric. Some examples are shown in \cref{fig:vertexdemo}.

\begin{figure}[ht]
	\begin{subfigure}{.45\textwidth}
		\centering
		\begin{tikzpicture}
			\begin{axis}[
					framed,
					width=\textwidth,
					xmin=-5,xmax=5,
					ymin=-5,ymax=5,
					xtick={-4,-2,...,4},
					ytick={-4,-2,...,4},
					minor xtick={-3,-1,...,3},
					minor ytick={-3,-1,...,3},
					grid=both,
				]
				\addplot expression[domain=-0.5:4.5,samples=100]{(x-2)^2-3};
				\addplot[cmhplot,soldot] coordinates{	(2,-3) 	};
			\end{axis}
		\end{tikzpicture}
		\caption{$a>0$: Vertex is at $(2,-3)$}
	\end{subfigure}%
	\hfill
	\begin{subfigure}{.45\textwidth}
		\centering
		\begin{tikzpicture}
			\begin{axis}[
					framed,
					width=\textwidth,
					xmin=-5,xmax=5,
					ymin=-5,ymax=5,
					xtick={-4,-2,...,4},
					minor xtick={-3,-1,...,3},
					minor ytick={-3,-1,...,3},
					ytick={-4,-2,...,4},
					grid=both,
				]
				\addplot expression[domain=-0.5:4.5,samples=100]{-1*(x-2)^2+3};
				\addplot[cmhplot,soldot] coordinates{	(2,3) };
			\end{axis}
		\end{tikzpicture}
		\caption{$a<0$: Vertex is at $(2,3)$}
	\end{subfigure}
	\caption{Notice that when $a>0$ the vertex is the lowest point on the parabola
	and when $a<0$ the vertex is the highest point on the parabola}
	\label{fig:vertexdemo}
\end{figure}

We can find the $x$ coordinate of the vertex of any quadratic using the formula
\[
	x = - \frac{b}{2a}
\]	
and the vertical coordinate is found by substituting this value in for $x$; we demonstrate
this with examples. 

\begin{myexample}
Find the coordinates of the vertex of
\begin{equation}\label{equn:vertex1}
	f(x) = x^2+x+5
\end{equation}
{}
\end{myexample}
\begin{myProof}
	We use the formula given above to find the $x$ coordinate of the vertex
	\begin{align*}
		x & =  - \frac{b}{2a}   \\
		  & =  - \frac{1}{2(1)} \\
		  & =  -\frac{1}{2}     
	\end{align*} 
	We find the $y$ coordinate by substituting this value of $x$ into \cref{equn:vertex1}
	\begin{align*}
		f\left(-\frac{b}{2a}\right) & =  \left(-\frac{1}{2}\right)^2 +\left(- \frac{1}{2}\right)+5 \\
		                            & =  \frac{1}{4}-\frac{1}{2}+5                                 \\
		                            & =  \frac{19}{4}                                              
	\end{align*} 
	Therefore, the coordinates of the vertex of this quadratic function are 
	\[
		\left( -\frac{1}{2}, \frac{19}{4}\right)
	\]
	which is shown in \cref{fig:quadvertex1}.
			
	\begin{figure}[!ht]
		\centering
		\begin{tikzpicture}
			\begin{axis}[
					framed,
					xmin=-5,xmax=5,
					ymin=-5,ymax=20,
					xtick={-4,...,4},
					ytick={-4,-2,...,18},
					grid=major,
				]
				\addplot expression[domain=-4.405:3.405,samples=100]{x^2+x+5};
				\addplot[cmhplot,soldot] coordinates{	(-0.5,4.75)	};
				\legend{$f(x)=x^2+x+5$};
			\end{axis}
		\end{tikzpicture}
		\caption{The quadratic function $f(x)=x^2+x+5$; notice that we have marked the vertex $\left(-\frac{1}{2}, \frac{19}{4}\right)$.}
		\label{fig:quadvertex1}
	\end{figure}
\end{myProof}

\FloatBarrier

\begin{myexample}
Find the coordinates of the vertex of
\[
	f(x) = -x^2 - 3x -4
\]
and state your answer as an ordered pair.
\end{myexample}
\begin{myProof}
	We follow the same procedure as illustrated previously, and begin by finding
	the $x$ coordinate
	\begin{align*}
		x & =  -\frac{b}{2a}     \\
		  & =  -\frac{-3}{2(-1)} \\
		  & =  -\frac{3}{2}      
	\end{align*} 
	The $y$ coordinate can be found by substituting this value of $x$ into the original
	quadratic function
	\begin{align*}
		f\left(-\frac{b}{2a}\right) & =  - \left(-\frac{3}{2}\right)^2 - 3\left(- \frac{3}{2}\right)-4 \\
		                            & =  -\frac{9}{4}+\frac{9}{2} - 4                                  \\
		                            & =  - \frac{7}{4}                                                 
	\end{align*} 
	This can be seen in \cref{fig:quadvertex2}.
			
	\begin{figure}[!ht]
		\centering
		\begin{tikzpicture}
			\begin{axis}[
					framed,
					xmin=-5,xmax=5,
					ymin=-10,ymax=5,
					xtick={-4,...,4},
					ytick={-9,...,4},
					grid=major,
				]
				\addplot expression[domain=-4.37:1.37,samples=100]{-1*x^2-3*x-4};
				\addplot[cmhplot,soldot]coordinates{	(-1.5,-1.75) 	};
				\legend{$f(x)=-x^2-3x-4$};
			\end{axis}
		\end{tikzpicture}
		\caption{The quadratic function $f(x)=-x^2-3x-4$; note that we have marked the vertex.}
		\label{fig:quadvertex2}
	\end{figure}
	\FloatBarrier
\end{myProof} 

\subsection{Plotting a quadratic function}
We now combine all of these ideas and plot a quadratic. We will find all of the features we have just 
described, together with at least 3 ordered pairs in order to help us plot an accurate graph. 

Note: the example given here will be plotted using a computer package, but it is intended that you perform
the exercises {by hand on graphing paper}.

\begin{myexample}
Plot the quadratic function
\[
	f(x) = x^2+2x-3
\]
{}
\end{myexample}
\begin{myProof}
	Note first of all that $a=1>0$, we therefore know that the graph will be a parabola that opens
	upwards.
			
	Our approach will be to find
	\begin{itemize}
		\item $x$ intercepts (if any):\\
		We find the $x$ intercepts by solving the equation
		\[
			x^2+2x-3=0
		\]
		At this stage our factoring skills should be suitable sharp that we can spot that this quadratic
		can be factored to
		\[
			(x+3)(x-1)=0
		\]
		and therefore $x=-3$ or $x=1$. This means that the $x$ intercepts are at 
		\[
			(-3,0), \qquad and \qquad (1,0)
		\]
		\item $y$ intercept:\\
		We find this by evaluating $f(0)$, and therefore
		\begin{align*}
			f(0) & =  0^2+2(0)-3 \\
			     & =  -3         
		\end{align*} 
		The vertical intercept is at $(0,-3)$.
		\item coordinates of the vertex:\\
		The $x$ coordinate of the vertex is
		\begin{align*}
			x & =  - \frac{b}{2a}   \\
			  & =  - \frac{2}{2(1)} \\
			  & =  - 1              
		\end{align*} 
		The $y$ coordinate of the vertex is 
		\begin{align*}
			f(-1) & =  (-1)^2+2(-1)-3 \\
			      & =  -4             
		\end{align*} 
		The coordinates of the vertex are $(-1,-4)$.
		\item three other ordered pairs:\\
		We can choose any $x$ values we wish, but it makes sense to try and choose them close to the points that 
		we have already
		\begin{center}
			\begin{tabular}{SS}
				\toprule
				{$x$} & {$y$} \\
				\midrule
				-4    & 5     \\
				-2    & -3    \\
				2     & 5     \\
				\bottomrule
			\end{tabular} 
		\end{center}
	\end{itemize}
	We now plot these points on a graph, see \cref{fig:plotpointsquad}.
			
	\tikzstyle{every pin}=[fill=white, draw=blue]
	\begin{figure}[!ht]
		\centering
		\begin{tikzpicture}
			\begin{axis}[
					framed,
					xmin=-10,xmax=10,
					ymin=-10,ymax=10,
					xtick={-8,-6,...,8},
					ytick={-8,-6,...,8},
					minor xtick={-9,-7,...,9},
					minor ytick={-9,-7,...,9},
					grid=both,
				]
				\node[coordinate,pin=left:{\tiny Vertex}]at (axis cs:-1,-4) {};
				\node[coordinate,pin=right:{\tiny Vertical Intercept}]at (axis cs:0,-3) {};
				\node[coordinate,pin=left:{\tiny Horizontal Intercept}]at (axis cs:-3,0) {};
				\node[coordinate,pin=right:{\tiny Horizontal Intercept}]at (axis cs:1,0) {};
				\addplot[cmhplot,soldot] coordinates{	(-4,5)(-3,0)(-2,-3)(-1,-4)(0,-3)(1,0)(2,5)};
			\end{axis}
		\end{tikzpicture}
		\caption{We begin by plotting the ordered pairs.}
		\label{fig:plotpointsquad}
	\end{figure}
	\FloatBarrier
			
	Once we have drawn the points, we can connect them together as a smooth curve (\cref{fig:connected}).
			
	\begin{figure}[!ht]
		\centering
		\begin{tikzpicture}
			\centering
			\begin{axis}[
					framed,
					xmin=-10,xmax=10,
					ymin=-10,ymax=10,
					xtick={-8,-6,...,8},
					ytick={-8,-6,...,8},
					minor xtick={-9,-7,...,9},
					minor ytick={-9,-7,...,9},
					grid=both,
					legend pos=south west,
				]
				\addplot expression[domain=-4.5:2.7,samples=100]{x^2+2*x-3};
				\addplot[cmhplot,soldot] coordinates{	(-4,5)(-3,0)(-2,-3)(-1,-4)(0,-3)(1,0)(2,5)};
				\legend{$f(x)=x^2+2x-3$};
			\end{axis}
		\end{tikzpicture}
		\caption{Now connect all of the points using a smooth curve.}
		\label{fig:connected}
	\end{figure}
	\FloatBarrier
			
\end{myProof} 


\pgfplotsset{every axis/.append style={width=8cm}}

\begin{myexample}
Match each of the following quadratic functions with one of the graphs 
below.
\begin{multicols}{2}
	\begin{itemize}
		\item $f(x)=-x^2$
		\item $f(x)=(x+3)(x-1)$
		\item $f(x)=-(x+2)^2$
		\item $f(x)=(x+2)(x-1)$
	\end{itemize}
\end{multicols}
\end{myexample}


\begin{minipage}{.5\textwidth}
	\begin{tikzpicture}
		\begin{axis}[
				framed,
				xmin=-5,xmax=5,
				ymin=-5,ymax=5,
				grid=major,
				xtick={-4,...,4},
				ytick={-4,...,4}
			]
			\addplot expression[domain=-4:2,samples=100]{(x+3)*(x-1)};
		\end{axis}
	\end{tikzpicture}
\end{minipage}%
\hfill
\begin{minipage}{.4\textwidth}
	$f(x)=$\solution{$(x+3)(x-1)$}
			
	$x$ intercepts: \solution{$\{(-3,0),(1,0)\}$}
			
	$y$ intercept:  \solution{$(0,-3)$}
			
	axis of symmetry:  \solution{$x=-1$}
			
	vertex:  \solution{$(-1,-4)$}
\end{minipage}

\begin{minipage}{.5\textwidth}
	\begin{tikzpicture}
		\begin{axis}[
				framed,
				xmin=-5,xmax=5,
				ymin=-5,ymax=5,
				grid=major,
				xtick={-4,...,4},
				ytick={-4,...,4}
			]
			\addplot expression[domain=-2.1:2.1,samples=100]{-1*x^2};
		\end{axis}
	\end{tikzpicture}
\end{minipage}%
\hfill
\begin{minipage}{.4\textwidth}
	$f(x)=$ \solution{$-x^2$}
			
	$x$ intercepts:  \solution{$(0,0)$}
			
	$y$ intercept:  \solution{$(0,0)$}
			
	axis of symmetry:  \solution{$x=0$}
			
	vertex:  \solution{$(0,0)$}\
\end{minipage}

\begin{minipage}{.5\textwidth}
	\begin{tikzpicture}
		\begin{axis}[
				framed,
				xmin=-5,xmax=5,
				ymin=-5,ymax=5,
				grid=major,
				xtick={-4,...,4},
				ytick={-4,...,4}
			]
			\addplot expression[domain=-3.1:2.1,samples=100]{-(x+2)*(x-1)};
		\end{axis}
	\end{tikzpicture}
\end{minipage}%
\hfill
\begin{minipage}{.4\textwidth}
	$f(x)=$ \solution{$(x+2)(x-1)$}
			
	$x$ intercepts:  \solution{$(-2,0),(1,0)$}
			
	$y$ intercept:  \solution{$(0,-2)$}
			
	axis of symmetry:  \solution{$x=-\frac{1}{2}$}
			
	vertex:  \solution{$\left(-\frac{1}{2}, -\frac{9}{4}\right)$}
\end{minipage}

\begin{minipage}{.5\textwidth}
	\begin{tikzpicture}
		\begin{axis}[
				framed,
				xmin=-5,xmax=5,
				ymin=-5,ymax=5,
				grid=major,
				xtick={-4,...,4},
				ytick={-4,...,4},
				width=\textwidth,
			]
			\addplot expression[domain=-4.1:1.01,samples=100]{-1*(x+2)^2};
		\end{axis}
	\end{tikzpicture}
\end{minipage}%
\hfill
\begin{minipage}{.4\textwidth}
	$f(x)=$ \solution{$-(x+2)^2$}
			
	$x$ intercepts:  \solution{$(-2,0)$}
			
	$y$ intercept:  \solution{$(0,-4)$}
			
	axis of symmetry:  \solution{$x=-2$}
			
	vertex:  \solution{$(-2,0)$}
\end{minipage}

\begin{myexample}
\drillandskill
Give a possible formula for a quadratic function that has zeros at
\begin{enumerate}
	\item $x=1,2$ \solution{$f(x)=(x-1)(x-2)$}
	\item $x=-1,1$ \solution{$f(x)=(x+1)(x-1)$} 
	\item $x=0,3$ \solution{$f(x)=x(x-3)$}
	\item $x=-3,4$ \solution{$f(x)=(x+3)(x-4)$}
\end{enumerate}
\end{myexample}

\subsection{Summary of plotting a quadratic}
When plotting a quadratic there are four main tasks:
\begin{itemize}
	\item decide if the graph opens upward ($a>0$) or downward ($a<0$)
	\item find the $x$ intercepts by setting $y=0$
	\item find the vertical intercept by setting $x=0$
	\item find the coordinates of the vertex
\end{itemize} 
We should also find three additional points so that we can draw a smooth curve. 


% need this hack to take page numbers off glossaries page
% http://tex.stackexchange.com/questions/19738/why-doesnt-pagestyleempty-work-on-the-first-page-of-a-chapter
\makeatletter
\renewcommand\chapter{\if@openright\cleardoublepage\else\clearpage\fi
	\thispagestyle{empty}% original style: plain
	\global\@topnum\z@
	\@afterindentfalse
	\secdef\@chapter\@schapter}
\makeatother

\pagestyle{empty}% original style: plain
\printglossary

% then close the 'fixthis' and 'sectionRef' files
\closeout\sortwrite
\closeout\sectionRefwrite

\end{document}
