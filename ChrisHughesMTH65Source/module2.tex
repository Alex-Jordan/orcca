%+*** mainfile.tex
% !arara: pdflatex: { files: [ mainfile.tex ] }
% arara: makechapters: { files:[mainfile], items: [module2], makeChapGlossaries: yes}
% !arara: indent: { overwrite: on, trace: yes, localSettings: on}
\chapter{Polynomials}
\minitoc
\section{Problem solving Using systems of equations}
\textref{5.4}{318}%
Nearly every person has trouble doing word problems at one time or another. It is mostly true that the hardest part of word problems is setting them up- that is, translating the words of the problem into mathematics. Once this is done, it is `simply' a case of using the skills we have acquired to \gls{solve} them. There are a few guidelines that will help us with these types of problems.

Remember these are guidelines for solving word problems involving systems of \gls{linear} \gls{equation}. 
\begin{steps}
	\item READ and study the problem
	\item Make a key, chart or sketch in order to familiarize yourself with the problem.
	\item Write down the equation, being careful to introduce all variables.
	\item Choose the type of method for solving (by graphing, substitution, or elimination).
	\item Solve for the first \gls{variable}. 
	\item Solve for the second variable.
	\item Write your answers in a complete sentence. 
	\item Check the answers mathematically, and for common sense.
\end{steps}
We will demonstrate some techniques with the following examples, but remember that sometimes
there are other ways to solve the problem; such an example is described in 
\href{http://math.stackexchange.com/questions/478212/is-there-another-simpler-method-to-solve-this-elementary-school-math-problem}{A systems problem involving 
bicycle wheels}.

\begin{myexample}
A local flower shop has sold 8 roses and 10 carnations for \$110 to the first customer of the day. Later that day, another customer bought a dozen roses and a dozen carnations for \$156. What is the cost of one rose at this floral shop?
\end{myexample}
\begin{myProof}
	First, let's do some experimentation. We are asked to find the cost of one rose, and one carnation at this shop. If one rose costs \$5, and one carnation costs \$10, then the first customer would have had to pay
	\[
		8(5)+10(10)=140
	\]
	which is clearly incorrect, as we are told that this customer was only charged \$110. We could use similar reasoning for an experiment with roses costing \$7, and carnations costing \$11, which would lead to the first customer being charged
	\[
		8(7)+10(11) = 166
	\]
	which is again incorrect. We could obviously proceed by trial and error in this way (which is valid in its own right), but our goal is to set up this problem algebraically.
		
	Now that we have a feel for the problem, let us declare our variables; let
	\begin{tightcenter}
		$r$= the cost of 1 rose, $c$ = cost of 1 carnation
	\end{tightcenter}
	From our previous work, the equation will be
	\begin{align*}
		8r + 10c & =		110 \\
		12r +12c & =		156 
	\end{align*}
	Now that we have set up the equation, we can solve them using any of the techniques illustrated in Module 1. Here we will demonstrate the method of addition
	\begin{alignat*}{2}
		\text{Multiply the top equation by 	\color{red}{3}:}      &   & \color{red}{3}(8r+10c)     & = \color{red}{3}(110)     \\
		\text{Multiply the bottom equation by \color{green}{-2}:} &   & \color{green}{-2}(12r+12c) & =  \color{green}{-2}(156) 
	\end{alignat*}
	This gives
	\begin{align*}
		24r+30c  & =		330  \\
		-24r-24c & =		-312 
	\end{align*}
	We can now add the two equation together to give
	\[
		-6c=-18
	\]
	from which we can solve for $c$ to give
	\[
		c=3
	\]
	We conclude by writing our \gls{solution} in a complete sentence:
	\begin{tightcenter}
		{\em At this floral shop, a rose cost \$10 and a carnation costs \$3.}
	\end{tightcenter}
	It remains for us to check the solution in the original problem. With $r=10$ and $c=3$, the first
	customer pays
	\[
		8(10)+10(3)=110,
	\]
	and the second customer pays
	\[
		12(10)+12(3)=156,
	\]
	both of which are as required.
\end{myProof}

\begin{myexample}
At a certain school event, an adult ticket cost \$8 while a child ticket is \$4. If 450 tickets were sold for a total of \$3000, how many
children attended this event?
\end{myexample}
\begin{myProof}
	A good way to record information is to use a chart. This works well for a variety of
	word problems. 
		
	We can summarize the information given to us in the problem as follows
	\begin{tightcenter}
		\begin{tabular}{lccc}
			\toprule
			         & Amount & Unit cost & Total value \\ 
			\midrule
			Adults   & $a$    & $8$       & $8a$        \\
			Children & $c$    & $4$       & $4c$        \\
			Totals   & $450$  &           & $3000$      \\
			\bottomrule
		\end{tabular}
	\end{tightcenter}
	We can read off two pieces of information from this chart; one from the `Amount' column, and one
	from the `Total Value' column; these give the following equation (respectively)
	\begin{align*}
		a+c   & =		450  \\
		8a+4c & =		3000 
	\end{align*}
	If we multiply the first equation by $-4$, and leave the second equation alone, then
	we obtain
	\begin{align*}
		-4a-4c & =		-1800 \\
		8a+4c  & =		3000  
	\end{align*}
	Adding the two equation together gives
	\[
		4a = 1200
	\]	
	and therefore 
	\[
		a = 300
	\]
	We can work out the number of children that attended by calculating $450-300=150$. We conclude
	that
	\begin{tightcenter}
		{\em At this school event 300 adults and 150 children attended.}
	\end{tightcenter}
	It is left for us to check these answers in the same that we did for the previous example. We
	can clearly see that a total of
	\[
		150+300=450
	\]
	tickets were sold (satisfying one of the pieces of information), and that a total of
	\[
		8(300)+4(150)=3000
	\]
	dollars was made from the sale of tickets.
\end{myProof}

\section{Polynomials}
\textref{6.1}{342}%
The algebraic \gls{expression} $-0.75d^4+3d^4+5$ describes the billions of viral particles in 
our bodies after $d$ days of invasion
	The expression enables us to determine the day on which there is a maximum
	number of viral particles and, consequently, the day we feel sickest.
	The expression is an example of a \gls{polynomial}

\begin{myDefinition}
	\begin{description}
		\item[polynomial] A single term or the sum (remember that subtraction can be written as a sum) of two or
		more terms containing variables only in the numerator, and all exponents are natural numbers.
		\item[Degree] The \gls{degree} of a polynomial is the highest power of any term of the polynomial.  
	\end{description}
		 
\end{myDefinition}

\begin{myexample}
Consider the polynomial
\[
	9x^6-5x^4+7x^2+3
\]
\end{myexample}
We note that the highest order term is $9x^6$. The polynomial therefore has degree 6.
\begin{myexample}
Consider the polynomial 
\[
	-4 + 6x^2-4x^5-8x^3
\]
\begin{enumerate}
	\item How many terms are there?
	\item What is the degree of each term?
	\item What is the degree of the polynomial (the highest degree of all the terms of the polynomial)
	\item State the polynomial in standard form.
\end{enumerate} 
\end{myexample}
\begin{myProof}
	\begin{enumerate}
		\item There are 4 terms: $-4$, $6x^2$, $-4x^5$, $-8x^3$
		\item $-4$ has degree $0$, $6x^2$ has degree 2, $-4x^5$ has degree 5, $8x^3$ has a degree of 3
		\item The degree of the polynomial is 5, since the highest power of $x$ is 5
		\item The standard form for this polynomial is
		\[
			-4x^5-8x^3+6x^2-4
		\]
		Note that the standard form of a polynomial is written with the highest power
		of $x$ written first, and then decreasing powers of $x$.
	\end{enumerate} 
\end{myProof} 
\begin{myDefinition}
	{\bfseries Special polynomials}: There are three forms of polynomial that arise frequently, and are worthy of their
	own terminology:
	\begin{enumerate}
		\item \gls{monomial}- a polynomial with one term
		\item \gls{binomial}- a polynomial with two terms
		\item \Gls{trinomial}- a polynomial with three terms
	\end{enumerate} 
\end{myDefinition}
\section{Exponent Rules}
There are 3 basic rules for exponents that will help you multiply polynomial
\begin{enumerate}
	\item The Product Rule- How to multiply exponential expression
	\item The Power Rule- How to raise an exponential expression to a power (power to power)
	\item The Products to Powers Rule- How to raise exponential products to a power.
\end{enumerate} 
We discuss each of these in turn.
\subsection{Multiplication and Exponents}
Consider the following:
\begin{itemize}
	\item $5+5+5=3(5)$ or 15 : repeated addition of the same number becomes multiplication
	\item $5\cdot 5\cdot 5=5^3$ or 125 : repeated multiplication of the same number becomes exponentiation It is important
to realize that $3(5)\ne 5^3$
\end{itemize}
In general
\begin{itemize}
	\item $x+x+x=3x$: repeated addition of the same number becomes multiplication.
	\item $x\cdot x\cdot x=x^3$: repeated multiplication of the same number becomes exponentiation.
\end{itemize}

It is important to realize that $3x\ne x^3$ in general.

\begin{myexample}
Exponents are used to indicate repeated multiplication. Fill in the blanks, then check your answers:
\begin{enumerate}
	\item In $2^4$, $2$ is the \underline{\phantom{base}} and 4 is the \underline{\phantom{exponent}} which indicates 2
	occurs as a \underline{\phantom{\gls{factor}}} 4 times.
	\item 
	$\begin{aligned}[t]
	2^4 &= 2\cdot 2\cdot 2\cdot 2 \\
	& = \underline{\phantom{16}}
	\end{aligned}$
\end{enumerate} 
\end{myexample}
\begin{myProof}
	\begin{enumerate}
		\item $2$ is the {\color{red}{base}} and 4 is the {\color{red}{exponent}} which indicates 2
		occurs as a {\color{red}{factor}} 4 times.
		\item 
		$\begin{aligned}[t]
		2^4 & = 2\cdot 2\cdot 2\cdot 2 \\
		&= {\color{red}{{16}}}
		\end{aligned}$
	\end{enumerate} 
	{}
		
\end{myProof}	
\subsection{1st Rule for Exponents: The Product Rule}
This rule tells us how to multiply exponential expression. We will demonstrate
this by {example}.

Without evaluating, \gls{simplify} the expression $2^4*2^3$
\begin{align*}
	2^4\cdot 2^3 & =	(2\cdot 2\cdot 2\cdot 2)\cdot (2\cdot 2\cdot 2) \\
	             & =	2^7                                             
\end{align*} 
Check by evaluating the left side $2^4*2^3$ should have the same value as the right $2^7$:
\begin{tightcenter}
	$2^4=16 \qquad 2^3=8\qquad 2^4*2^3=16(8)=128$ and $2^7=128$
\end{tightcenter}
Note that in order to simplify exponential expression using the product rule the base must be the
same. For {example}, there is no simpler form for
\[
	3^2 4^5
\]
We could evaluate the expression, but there are only 2 factors of 3, and 5 factors of 4. It would make
no sense here to add (or multiply) the exponents (or the bases).
\begin{myDefinition}
	The Product Rule for exponents is defined as
	\[
		b^m\cdot b^n=b^{m+n}
	\]
	When multiplying exponential expression with the same base, add the exponents.
\end{myDefinition}
\begin{myexample}
Multiply each expression using the product rule:
\begin{multicols}{5}
	\begin{enumerate}
		\item $3^2 3^5$
		\item $n^7 n^4$
		\item $x x^8$
		\item $y(y^8)(y^4)$
		\item $x^2 x^8 y^3 y^4$
	\end{enumerate} 
\end{multicols}
\end{myexample}
\begin{myProof}
	\begin{enumerate}
		\item 
		$\begin{aligned}[t]
		3^2 3^5&=3^{2+5}\\
		&=3^7
		\end{aligned}$ 
		
		The bases are the same, so we can add the exponents.
		\item $n^7 n^4=n^{11}$ The bases are the same, so we can add the exponents.
		\item 
		$\begin{aligned}[t]
		x \cdot x^8&=x^1\cdot x^8\\
		& = x^9
		\end{aligned}$ 
		
		The bases are the same; remember that $x$ is the same as $x^1$
		\item 
		$\begin{aligned}[t]
		y(y^8)(y^4)&=y^{1+8+4}\\
		&=y^{13}
		\end{aligned}$
		
		The bases are the same, so we can add the exponents.
		\item 
		$\begin{aligned}[t]
		x^2 x^8 y^3 y^4&=x^{2+8}y^{3+4}\\
		&=x^{10}y^7
		\end{aligned}$
		
		Note that we can only add powers of the same base.
	\end{enumerate} 
	\mbox{}
\end{myProof} 
Beware- it is very easy to  confuse multiplication with addition; consider the following example.
\begin{myexample}
Compare 
\begin{enumerate}
	\item $2^4\cdot 2^3$ with $2^4 + 2^3$
	\item $x^4 \cdot x^3$ with $x^4+x^3$
	\item $x^4 \cdot x^4$ with $x^4+x^4$
\end{enumerate} 
{}
\end{myexample}
\begin{myProof}
	\begin{enumerate}
		\item $2^4\cdot  2^3=2^7$ but $2^4 + 2^3$ has no simpler form, except to evaluate.
		\item $x^4 \cdot  x^3=x^7$ but $x^4+x^3$ has no simpler form- they are not like terms!
		\item $x^4 \cdot  x^4=x^8$ but $x^4+x^4=2x^4$. These are like terms
	\end{enumerate} 
\end{myProof} 

\subsection{2nd Rule for Exponents- The Power Rule}
This rule enables us to raise an exponential expression to a power (power to a power).

Consider the expression
\[
	(3^2)^5
\]
Without doing any actual arithmetic, what does it mean? The expression inside the parenthesis
will be raised to the $5^{th}$ power. That means $3^2$ will be used as a factor 5 times
\begin{align*}
	(3^2)^5 & =		3^23^23^23^23^2                                                  \\
	        & =		3(3)\cdot 3(3)\cdot 3(3)\cdot 3(3)\cdot 3(3)                     \\
	        & =		3\cdot 3\cdot 3\cdot 3\cdot 3\cdot 3\cdot 3\cdot 3\cdot 3\cdot 3 \\
	        & =		3^{10}                                                           
\end{align*} 
Or
\begin{align*}
	(3^2)^5 & =		3^2\cdot 3^2\cdot 3^2\cdot 3^2\cdot 3^2 \\
	        & =		3^{2+2+2+2+2}                           \\
	        & =		3^{10}                                  
\end{align*}
To summarize, we have found that
\[
	(3^2)^5 = 3^{10}
\]
This leads us to the power rule.
\begin{myDefinition}
	The Power Rule: Let $b$ be any real number, with $m$ and $n$ integers (in future Math classes
	you will consider the case when $m$, $n$ are any real numbers), then
	\[
		(b^m)^n = b^{mn}
	\]
	When an exponential expression is raised to a power, we multiply the exponents. Place
	the product of the exponents on the base and remove the parenthesis.
\end{myDefinition}
\begin{myexample}
Multiply each expression using the power rule
\begin{multicols}{4}
	\begin{enumerate}
		\item $(3^2)^5$
		\item $(n^7)^4$
		\item $(x^4)^4$
		\item $(x^2)^8(y^3)^4$
	\end{enumerate}
\end{multicols}
\end{myexample}
\begin{myProof}
	\begin{enumerate}
		\item $(3^2)^5=3^{2(5)}=3^{10}$  \hfill When raising a power to a power, multiply the exponents
		\item $(n^7)^4 = n^{7(4)} = n^{28}$
		\item $(x^4)^4 = x^{16}$
		\item Remember that the bases need to be the same in order for simplification to occur.
		\begin{align*}
			(x^2)^8 (y^3)^4 & =   x^{2(8)} y^{3(4)} \\
			                & =   x^{16} y^{12}     
		\end{align*} 
	\end{enumerate}
\end{myProof}
\subsection{3rd Rule for Exponents- Products to Powers Rule}
Suppose we have a product that is raised to a power such as
\[
	(3d)^4
\]
Hopefully you realize that the product $3d$ will be used as a factor 4 times:
\begin{align*}
	(3d)^4 & =		(3d)(3d)(3d)(3d) \\
	       & =		3(3)(3)(3)dddd   \\
	       & =		3^4d^4           
\end{align*}
We see that $(3d)^4 = 3^4 d^4$. 
\begin{itemize}
	\item It is important to realize that the product $3d$ is grouped inside 
	parenthesis, which is why it appears as a factor $4$ times. If there were
	no parenthesis, $3d^4$ would mean something different, i.e $3d^4 = 3 dddd$. Notice
	that the $3$ is not repeated as a factor $4$ times- only the $d$ is.
	\item This is the same principal for $(-5)^2$ versus $-5^2$. In $(-5)^2$ the $-5$ is 
	a grouping, and so $-5$ will appear as a factor 2 times
	\[
		(-5)^2 = (-5)(-5 = 25)
	\]
	In $-5^2$ the $-5$ is not grouped, and so only $5$ will appear as a factor 2 times
	\[
		-5^2 = -5(5) = -25
	\]
\end{itemize}
\begin{myDefinition}
	Products to Powers Rule: Let $a$ and $b$ be any real number, and $n$ any integer,
	then
	\[
		(ab)^n = a^n b^n
	\]
	when a product is raised to a power, raise each factor to the power.
\end{myDefinition}
\begin{myexample}
Demonstrate the products to powers rule with
\begin{multicols}{2}
	\begin{enumerate}
		\item $(2 \cdot 3)^5$
		\item $(2\cdot 3^2)^5$
	\end{enumerate}	
\end{multicols}
\end{myexample}
\begin{myProof}
	\begin{enumerate}
		\item Using the 3rd law of exponents, we have
		\begin{align*}
			(2\cdot 3)^5 & =		2^5\cdot 3^5 \\
			             & =		32 \cdot 243 \\
			             & =		7776         
		\end{align*}	
		You should also verify on your calculator that this is equal to $6^5$.
		\item We will need to use the both the 3rd law and the 2nd law of exponents here
		\begin{align*}
			(2 \cdot 3^2)^5 & =		2^5(3^2)^5 \\	
			                & =		2^5 3^{10} 
		\end{align*}
		We have simplified this expression, and we could evaluate it further by evaluating $2^5$
		and multiplying it by $3^{10}$.
	\end{enumerate}
\end{myProof}
\begin{myexample}
Simplify each expression using the products-to-powers rule:
\begin{multicols}{2}
	\begin{enumerate}
		\item $(-2n)^4$
		\item $(-4x^2)^3$
	\end{enumerate}
\end{multicols}
\end{myexample}
\begin{myProof}
	\begin{enumerate}
		\item Remember that since the entire group is being raised to the 4th power, the 
		factor of $-2$ appears as a factor 4 times. 
		\begin{align*}
			(-2n)^4 & =		(-2)^4(n^4) \\
			        & =		16 n^4      
		\end{align*}
		That is why it must still be inside parenthesis after applying the products to powers rule. 
		Since there is an even number of negative factors, $(-2)^4$ becomes 16.
		\item Remember that since the entire group is being raised to the 3rd power, the factor
		of $-4$ appears as a factor 3 times. 
		\begin{align*}
			(-4x^2)^3 & =		(-4)^3 (x^2)^3 \\
			          & =		-64 x^6        
		\end{align*}
		Since there are an odd number of negative
		factors, $(-4)^3$ becomes $-64$.
	\end{enumerate}	
	Notice that $(-4)^3=-4^3 = -64$. When the exponent is odd, the parenthesis are not necessary
	around a single term. However, when the exponent is even, the parenthesis are necessary; for
	example, $-2^4 = -16$ and $(-2)^4 = 16$.
\end{myProof}

\section{Adding and Subtracting Polynomials}
In this section we will discuss adding and subtracting polynomial. This is essentially
an exercise in combining like terms. There are two main formats for adding and subtracting
polynomial: vertical format, and horizontal format. 

We will demonstrate each technique with examples.
\begin{myexample}\label{ex:addpolyhoriz}
Add the polynomial
\begin{align*}
	7y^3+5y-1 & \qquad and \quad 2y^2-6y+3 
\end{align*}
using horizontal format.
\end{myexample}
\begin{myProof}
	We arrange the terms with highest power of $y$ at the front of the expression
	\begin{align*}
		(7y^3+5y-1) + (2y^2-6y+3) & =		7y^3+2y^2+5y-6y-1+3 \\
		                          & =		7y^3+2y^2-y+2       
	\end{align*}
	Note that the grouping brackets are not strictly necessary as we are {\em adding} the 2
	polynomial, but if the problem involved subtraction then they would most certainly be so 
	(see \cref{ex:subpoly1,ex:subpoly2}).
\end{myProof}

\begin{myexample}\label{ex:polyvert}
Repeat \cref{ex:addpolyhoriz} using vertical format.
\end{myexample}
\begin{myProof}
	The main \gls{point} when using this technique is to line up like terms, leaving space for any terms
	that have a \gls{coefficient} of $0$.
	\begin{equation*}
		\begin{array}{ccccc}
			  & 7y^3 &       & +5y & -1 \\
			  &      & 2y^2  & -6y & +3 \\\hline
			= & 7y^3 & +2y^2 & -y  & +2 
		\end{array}
	\end{equation*}
\end{myProof}

\begin{myexample}\label{ex:subpoly1}
Subtract the polynomial
\[
	y^2-8y+9 \text{ from } 6y^3+2y^2-y-11 
\]
using horizontal format.
\end{myexample}
\begin{myProof}
	We are very careful to include grouping brackets around the 2nd polynomial
	\begin{align*}
		(6y^3+2y^2-y-11) - (y^2-8y+9) & =		6y^3+2y^2-y-11 {\color{red}-}y^2 {\color{red}+} 8y {\color{red}-} 9 \\
		                              & =		6y^3+2y^2-y-11-y^2+8y-9                                             \\
		                              & =		6y^3+2y^2-y^2-y+8y-9-11                                             \\
		                              & =		6y^3+y^2+7y-20                                                      
	\end{align*}
	You may find it helpful to rewrite the problem as
	\[
		(6y^3+2y^2-y-11) - {\color{red}1}(y^2-8y+9)
	\]
\end{myProof}

\begin{myexample}\label{ex:subpoly2}
Repeat \cref{ex:subpoly1} using vertical format.
\end{myexample}
\begin{myProof}
	We line up terms as we did in \cref{ex:polyvert}
	\begin{equation*}
		\begin{array}{crrrr}
			  & 6y^3 & +2y^2 & -y  & -11 \\
			- &      & (y^2  & -8y & +9) 
		\end{array}
	\end{equation*}
	We now need to change the problem to an addition problem. This requires
	changing all the terms of the subtrahend (the 2nd polynomial) to their opposites. Then combine
	all like terms, using the rules for addition
	\begin{equation*}
		\begin{array}{crrrr}
			  & 6y^3 & +2y^2 & -y  & -11 \\
			+ &      & (-y^2 & +8y & -9) \\ \hline
			  & 6y^3 & +y^2  & +7y & -20 
		\end{array}
	\end{equation*}
\end{myProof}

\begin{myexample}
Simplify each of the following.
\end{myexample}
\begin{myProof}
	\drillandskill
	{\em Addition practise}:
	\begin{enumerate}
		\item $3x+1+(4x-1)$ \solution{$=7x$}
		\item $(3x+1)+(4x-1)$ \solution{$=7x$}
		\item $-2x-1+(x^2-3x+2)$ \solution{$=x^2-5x+1$}
		\item $x^2+3x+1+(2x^2-4x-5)$ \solution{$=3x^2-x-4$}
		\item $-3y^4+y^3-2y^2+1+ (y^5-y^4+y^2-3)$ \solution{$=y^5-4y^4+y^3-y^2-2$}
	\end{enumerate}
	\solution{\em Subtraction practise}:
	\begin{enumerate}
		\item $2x-1 - (4x+3)$ \solution{$=-2x-4$}
		\item $-x+10 - (2x+3)$ \solution{$=-3x+7$}
		\item $x^2+2x-3 - (3x^2+4x+1)$ \solution{$=-2x^2-2x-4$}
		\item $x^2+2x-3 - (-3x^2-4x-1)$ \solution{$=4x^2+6x-2$}
		\item $y^3+y^2+y - (3y^3-4y^2+y-1)$ \solution{$=-2y^3+5y^2+1$}
	\end{enumerate}
	{\em Harder addition}:
	\begin{enumerate}
		\item $3x+1+2(4x-1)$ \solution{$=11x-1$}
		\item $(3x+1)+3(4x-1)$ \solution{$=15x-2$}
		\item $-2x-1+2(x^2-3x+2)$ \solution{$=2x^2-8x+3$}
		\item $x^2+3x+1+4(2x^2-4x-5)$ \solution{$=9x^2-13x-19$}
		\item $-3y^4+y^3-2y^2+1+ 3(y^5-y^4+y^2-3)$ \solution{$=3y^5-6y^4+y^3+y^2-8$}
	\end{enumerate}
\end{myProof} 

\section{Multiplying Polynomials}
\textref{6.2}{350}%
There are 3 basic rules for exponents that we have discussed so far in this module that will 
helps us to multiply polynomial
\begin{enumerate}
	\item The Product Rule- how to multiply exponential expression
	\item The Power Rule- how to raise exponential expression to a power (power to a power)
	\item The Products to Powers Rule- how to raise exponential products to a power.
\end{enumerate}
We will demonstrate the techniques with a series of examples
\begin{enumerate}
	\item Multiply monomial
	\item Multiply a monomial and a polynomial with more than 1 term
	\item Multiply polynomial with more than 1 term
\end{enumerate}

\subsection{Multiplying Monomials}
To multiply monomial, we need to make use of the commutative and associative property of 
multiplication. 

We multiply two monomial (which are just polynomial with one term), by multiplying the coefficients
together, and then the algebraic terms. For example
\begin{align*}
	(-3x)(5x^2) & =		(-3)(5)x\cdot x^2 \\
	            & =		-15 x^3           
\end{align*}
Notice here that we have used the commutative property of multiplication here to switch the 
order of multiplication.

Notice also that this is clearly very different from the addition
\[
	(-3x)+(5x^2)=-3x+5x^2
\] 
which are not like terms and \emph{can not} be simplified.

\begin{myDefinition}
	To multiply monomial, multiply the coefficients and then multiply the variables. Use the product
	rule for exponents to multiply the variables.
\end{myDefinition}

\subsection{Multiplying a Monomial and a Polynomial with more than 1 term}
We use the distributive property to multiply a monomial and a polynomial that has more than one term.

\begin{myexample}
Multiply
\[
	{\color{red}3x^2}(2x^3+5x)
\]
\end{myexample}
\begin{myProof}
	Notice that the 1st factor is a monomial (a polynomial with one term), while the 2nd
	term is a binomial (a polynomial with two terms)
	\begin{align*}
		{\color{red}3x^2}(2x^3+5x) & =		{\color{red}3x^2}\cdot 2x^3+{\color{red}3x^2}\cdot 5x                                             \\
		                           & =		{\color{red}3}\cdot 2 \cdot {\color{red}x^2}x^3 + {\color{red}3}\cdot 5 \cdot {\color{red}x^2}x^1 \\
		                           & =		6x^5+15x^3                                                                                        
	\end{align*}
\end{myProof}

\subsection{Multiplying Polynomials which both have more than 1 term}
Multiplication over a group of terms introduces another level of difficulty because we need to 
remember to distribute.

\begin{myexample}
Multiply the following polynomial
\[
	(2x+3)(x^2+4x-5)
\]
\end{myexample}
\begin{myProof}
	We have a group of terms multiplying a group of terms. There are several ways to do this problem
	\begin{align*}
		(2x+3)(x^2+4x-5) & =		(2x+3)x^2+(2x+3)(4x)+(2x+3)(-5)        \\
		                 & =		(2x)x^2+3x^2+2x(4x)+3(4x)+2x(-5)+3(-5) \\
		                 & =		2x^3+3x^2+8x^2+12x-10x-15              \\
		                 & =		2x^3+11x^2+2x-15                       
	\end{align*}
	That's it. All the terms have different exponents, and we have written it with the highest
	power of $x$ at the front.
		
	We could also have multiplied the $2x$ throughout the trinomial, and then the $3$
	\[
		2x(x^2+4x-5)+3(x^2+4x-5)
	\]
	From here we can distribute each term, and collect like terms (exercise).
\end{myProof}

\begin{myexample}
Simplify the following and consider the differences between them
\begin{enumerate}
	\item $3x^3y-(4x^3y)$
	\item $3x^3y(-4x^3y)$
	\item $3x^3y(-4x^3+y)$
	\item $3x^3+y(-4x^3+y)$
	\item $(3x^3+y)(-4x^3+y)$
\end{enumerate}
\end{myexample}
\begin{myProof}
	\begin{itemize}
		\item The essential difference between a and b is the placement of the parenthesis. The first
		one is a subtraction problem, while the 2nd problem is a multiplication problem.
		\item The 3rd is a multiplication problem, and it is monomial times a binomial. We will need to distribute the monomial to the 2 terms inside the parenthesis.
		\item The 4th is an addition problem. In the 2nd term, the $y$ can be distributed to the
		2 terms inside the parenthesis. 
		\item The 5th problem is a multiplication problem. it is a binomial times a binomial. As we will soon see, the first
		group needs to be distributed to each term in the second group. Before we can correctly 
		simplify each of them, we need to be able to discern the differences.
	\end{itemize}
		
	\begin{enumerate}
		\item $3x^3y-(4x^3y) = -1x^3y = -x^3y$
		\item $3x^3y(-4x^3y) = -12x^6y^2$
		\item $3x^3y(-4x^3+y) = -12x^6y+3x^3y^2$
		\item $3x^3+y(-4x^3+y) = 3x^3-4x^3y+y^2$
		\item \label{item:distribute}
		$\begin{aligned}[t]
			{\color{red}(3x^3+y)}(-4x^3+y) & =	{\color{red}(3x^3+y)}(-4x^3)+{\color{red}(3x^3+y)}(y) \\
			                               & =	-12x^6-4x^3y+3x^3y+y^2                                \\
			                               & =	-12x^6-x^3y+y^2                                       
		\end{aligned}$
	\end{enumerate}
	Note that in \ref{item:distribute} we had to apply the distributive property twice. Each term in the first factor
	had to be multiplied by each term in the second factor. Or, as we saw above we could 
	have distributed the $3x^3$ to both terms, then the $y$ to both terms.
	
\end{myProof}

\begin{myexample}
Simplify the following.
\drillandskill
	
monomial with a monomial:
\begin{enumerate}
	\item $(x)(5x)$ \solution{$=5x^2$}
	\item $(2x)(6x)$ \solution{$=12x^2$}
	\item $(2x)(-6x^3)$ \solution{$=-12x^4$}
	\item $(-3x^{34})(x)$ \solution{$=-3x^{35}$}
\end{enumerate}
	
monomial with a binomial:
\begin{enumerate}
	\item $3x(x+1)$ \solution{$=3x^2+3x$}
	\item $-4x(2x^2+5x)$ \solution{$=-8x^3-20x^2$}
	\item $-4x(2x^2-5x)$ \solution{$=-8x^3+20x^2$}
	\item $x^2(-5x^3-2x)$ \solution{$=-5x^5-2x^3$}
\end{enumerate}
	
monomial with a trinomial:
\begin{enumerate}
	\item $4x(x^2+3x-1)$ \solution{$=4x^3+12x^2-4x$}
	\item $-6x(2x^3-x^2+1)$ \solution{$=-12x^4+6x^3-6x$}
\end{enumerate}
	
Binomial with a binomial:
\begin{enumerate}
	\item $(x+1)(x-2)$ \solution{$=x^2-x-2$}
	\item $(x-1)(x+3)$ \solution{$=x^2+2x-3$}
	\item $(x-4)(x+5)$ \solution{$=x^2+x-20$}
	\item $(2x-6)(x+2)$\solution{$=2x^2-2x-12$}
\end{enumerate}	
\end{myexample}


