%+*** mainfile.tex
% arara: pdflatex: { files: [ mainfile.tex ] }
% !arara: indent: { overwrite: on, trace: yes, localSettings: on}
\chapter{Quadratic equations}
\minitoc
\section{Rationalizing the denominator}
\subsection{Simple case}
\textref{9.4}{574}%
We have so far dealt with radical expressions such as
\[
	\sqrt{2} + 3\sqrt{2}
\]
There may be situations that force us to deal with radical symbols in the denominator, for example
\begin{equation}\label{eq:rationalize}
	\frac{1}{\sqrt{3}} + \sqrt{3}
\end{equation}
It is hard for a contemporary student to imagine a time without calculators, but be assured that 
it is only relatively recently that we have had impressive computing power at our finger tips. 

Historically if a student was asked to perform the calculation in \cref{eq:rationalize}
(by hand!) then they would find it significantly more simple if they could remove the radical symbol from the denominator. They would do so
as follows
\begin{align}
	\frac{1}{\sqrt{3}} + \sqrt{3} & =  \frac{1}{\sqrt{3}}\cdot {\color{red}\frac{\sqrt{3}}{\sqrt{3}}} + \sqrt{3}\label{eq:keysteprationalize} \\
	                              & =  \frac{\sqrt{3}}{3}+\sqrt{3} 	\nonumber                                                                 \\	
	                              & =  \frac{4\sqrt{3}}{3}\nonumber                                                                           
\end{align}
Notice that in \cref{eq:keysteprationalize} we multiply the term that has the radical in the denominator by a \gls{factor} of 1 (shown in red). We have used
the property of radicals that
\[
	\sqrt{3}\sqrt{3} = 3
\]
Even though the advent of technology means that we no longer need to perform such computations by hand, we still rationalize
the denominator as a matter of convention.

\begin{myexample}
Rationalize the denominator in each of the following expressions, and \gls{simplify}.
\begin{multicols}{3}
	\begin{enumerate}
		\item $\dd\frac{1}{\sqrt{2}}$
		\item $\dd\frac{2}{\sqrt{5}}+3\sqrt{3}$
		\item $\dd -\frac{3}{\sqrt{7}}+\sqrt{9}$
	\end{enumerate}
\end{multicols}
\end{myexample}
\begin{myProof}
	\begin{enumerate}
		\item 
		$\begin{aligned}[t]
			\frac{1}{\sqrt{2}} & =  \frac{1}{\sqrt{2}}\cdot \frac{\sqrt{2}}{\sqrt{2}} \\
			                   & =  \frac{\sqrt{2}}{2}                                
		\end{aligned}$
		\item 
		$\begin{aligned}[t]
			\frac{2}{\sqrt{5}}+3\sqrt{3} & =  \frac{2}{\sqrt{5}}\cdot \frac{\sqrt{5}}{\sqrt{5}}	+3 \sqrt{3} \\
			                             & =  \frac{2\sqrt{5}}{5}+ 3 \sqrt{3}                               
		\end{aligned}$
		\item 
		$\begin{aligned}[t]
			-\frac{3}{\sqrt{7}}+\sqrt{9} & =  -\frac{3}{\sqrt{7}}\cdot \frac{\sqrt{7}}{\sqrt{7}}+3 \\
			                             & =  -\frac{3\sqrt{7}}{7} + 3                             
		\end{aligned}$
	\end{enumerate}
	{}
\end{myProof}

\begin{myexample}
\drillandskill
Rationalize the denominator in each of the following.
\end{myexample}
\begin{multicols}{4}
	\begin{enumerate}
		\item $\dd\frac{1}{\sqrt{19}}$\solution{$=\dd\frac{\sqrt{19}}{19}$}
		\item $\dd\frac{2}{\sqrt{3}}$\solution{$=\dd\frac{2\sqrt{3}}{3}$}
		\item $\dd\frac{5}{\sqrt{30}}$\solution{$=\dd\frac{\sqrt{30}}{6}$}
		\item $\dd\frac{36}{\sqrt{40}}$\solution{$=\dd\frac{9\sqrt{40}}{10}$}
	\end{enumerate}
\end{multicols}

\subsection{Harder case}
We may sometimes wish to rationalize the denominator in expressions such as 
\begin{equation}\label{equn:hardrad}
	\frac{2}{3+\sqrt{7}}
\end{equation}
Our goal is the same as before- to remove the radical from the denominator. We 
do so using the conjugate, which we first introduced on \cpageref{def:conjugate}, 
but we state again for clarity:
\begin{myDefinition}
	The conjugate of
	\[
		\sqrt{a}-\sqrt{b}
	\]
	is 
	\[
		\sqrt{a}+\sqrt{b}
	\]
	The conjugate is useful because
	\[
		(\sqrt{a}+\sqrt{b})(\sqrt{a}-\sqrt{b}) = a-b
	\]
\end{myDefinition} 

We can use the conjugate to help us rationalize the denominator in \cref{equn:hardrad} as follows
\begin{align*}
	\frac{2}{3+\sqrt{7}} & =  \frac{2}{3+\sqrt{7}}\cdot {\color{red}\frac{3-\sqrt{7}}{3-\sqrt{7}}} \\
	                     & =  \frac{6-2\sqrt{7}}{9-7}                                              \\
	                     & =  \frac{6-2\sqrt{7}}{2}                                                \\
	                     & =  3 - \sqrt{7}                                                         
\end{align*} 
Notice that this process has removed the radical symbol from the denominator.

\begin{myexample}
\drillandskill
Rationalize the denominator (using the conjugate) in each of the following 
examples.
\end{myexample}
\begin{multicols}{2}
	\begin{enumerate}
		\item $\dd\frac{1}{\sqrt{2}+1}$\solution{$=\dd\sqrt{2}-1$}
		\item $\dd\frac{3}{\sqrt{5}+3}$\solution{$=\dd\frac{9}{4}-\frac{3\sqrt{5}}{4}$}
		\item $\dd\frac{10}{\sqrt{8}-\sqrt{2}}$\solution{$=\dd 5\sqrt{2}$}
		\item $\dd\frac{5}{\sqrt{7}-\sqrt{2}}$\solution{$=\dd \sqrt{2}+\sqrt{7}$}
	\end{enumerate}
\end{multicols}

\section{Solving quadratic equations by factoring}
\textref{7.6}{454}%
In the previous modules we have considered solving \gls{quadratic} equations by factoring;
we have also manipulated radical expressions. In this module we will combine the two
principles to help us \gls{solve} quadratic equations that can not be factored. We begin
this module with a review of factoring: 
\begin{enumerate}
	\item Use the zero-product principle
	\item Solve quadratic equations by factoring
	\item Solve problems involving quadratic equations
\end{enumerate} 
As a reminder, remember that equations that can be written in the form
\[
	ax^2 + bx+c =0
\]
where $a\ne 0$, $b$, and $c$ are real numbers, are called {\em Quadratic Equations} in $x$. A 
quadratic \gls{equation} in $x$ is also called a second \gls{degree} \gls{polynomial} equation in $x$.

\begin{myexample}
Solve
\begin{equation}\label{equn:reviewquad}
	x^2-7x+10 = 0
\end{equation}
\end{myexample}
\begin{myProof}
	As we have previously mentioned, because of the $x^2$ term this is not a \gls{linear} equation. If we factor the left side
	of the equation, then we obtain
	\[
		(x-5)(x-2) = 0
	\]
	We now have a product that equals 0. If a quadratic equation has a zero on one side and a factored \gls{expression} (multiplication) on 
	the other, it can be solved by the Zero product principle which we state as a reminder:
		
	\begin{myDefinition}
		{Zero product principle}: 
		This rule says that if the product of two numbers is 0, then either the first one equals 0, or the second one
		equals 0, or they both equal 0. 
				
		Mathematically, we say that if
		\[
			AB=0
		\]
		then either $A=0$ or $B=0$.
	\end{myDefinition}
		
	According to the zero product principle, this product can be zero only if at least one of the factors is equal
	to 0. We set each individual factor equal to 0, and solve each resulting equation for $x$. 
	Therefore
	\[
		x-5 = 0 \qquad {\mathrm{or}}\qquad  x-2=0
	\]
	and so either
	\[
		x=5 \qquad {\mathrm{or}} \qquad x=2
	\]
	We have 2 proposed solutions: $x=5$ and $x=2$. We need to check that both of these values satisfy \cref{equn:reviewquad},
	which we leave as an exercise for the reader.
\end{myProof} 

\subsection{Summary of Solving Quadratic equations by factoring}
\begin{steps}
	\item If necessary, rewrite the equation in the form $ax^2+bx+c=0$ by moving all terms to one side, thereby obtaining
	zero on the other side.
	\item  Factor
	\item Apply the zero-product principle, setting each factor equal to zero.
	\item Solve the equations in step 3.
	\item Check the solutions in the original equation. 
\end{steps} 
Please see the work in our earlier modules for more review.

\section{Solving quadratic equations using the square root property}
\textref{10.1}{600}%
We have so far considered solving quadratic equations by factoring. We may encounter quadratic equations
that can not be factored, and so we will need some alternative approaches. In this section we will discuss the square root property
and Pythagoras` Theorem. We will include examples to help describe each.
\begin{myDefinition}
	The square root property: If $x^2=d$, where $d>0$ is some real number, then
	\[
		x=\sqrt{d} \qquad or \qquad x=-\sqrt{d}
	\]
	{}
	We will use the square root property to help us solve quadratic equations.
\end{myDefinition}

\begin{myexample}
Solve
\[
	x^2-9=0
\]
{}
\end{myexample}
\begin{myProof}
	Notice firstly that this is a quadratic equation, and the left hand side is the difference of 2 perfect squares.
	Therefore, we could factor the left hand side to obtain
	\[
		(x-3)(x+3) = 0
	\]
	and conclude, by the zero product principle, that $x=\pm 3$. However, our intention in this example is to use 
	the square root property. 
		
	We begin by isolating $x^2$ which we do by adding $9$ to both sides
	\[
		x^2=9
	\]
	and now by the square root property, 
	\[
		x = \sqrt{9} \qquad or \qquad x=-\sqrt{9}
	\]
	Since $9$ is a perfect square, we can conclude that
	\[
		x = 3 \qquad or \qquad x=-3
	\]
	{}
\end{myProof} 

\begin{myexample}
Solve 
\[
	x^2-5=0
\]
{}
\end{myexample}
\begin{myProof}
	We proceed on the same lines as in the previous example by first isolating the square term
	\[
		x^2=5
	\]
	We can now solve this equation by using the square root property to obtain
	\[
		x = \pm \sqrt{5}
	\]
	{}
\end{myProof} 

\begin{myexample}
Solve the following quadratic equation using the square root property
\[
	(x+3)^2 = 13
\]
{}
\end{myexample}
\begin{myProof}
	Notice that the squared term is already isolated, so we can use the square root property
	to obtain
	\[
		x+3 = \pm \sqrt{13}
	\]
	At this stage we can subtract 3 from both sides, and conclude that
	\[
		x = -3 \pm \sqrt{13}
	\]
	{}
\end{myProof} 
\begin{myexample}
Solve the following quadratic equation using the square root property
\[
	(2x-1)^2 - 60=0
\]
{}
\end{myexample}
\begin{myProof}
	We first need to isolate the squared term, which we do by adding 60 to both sides
	\[
		(2x-1)^2=60
	\]
	We can now use the square root property to obtain
	\begin{equation}\label{equn:squarerootprop}
		2x-1 = \pm \sqrt{60}
	\end{equation}
	and we need to simplify the right hand side- we note that we can write $\sqrt{60}$ as
	\begin{align*}
		\sqrt{60} & =  \sqrt{4\cdot 15} \\
		          & =  2\sqrt{15}       
	\end{align*} 
	We can now continue to solve \cref{equn:squarerootprop} for $x$ by first adding 1 to both sides
	\[
		2x = 1 \pm 2\sqrt{15}
	\]
	and finally dividing both sides by 2
	\begin{align*}
		x & =   \frac{1\pm 2\sqrt{15}}{2} \\
		  & =  \frac{1}{2}\pm \sqrt{15}   
	\end{align*} 
\end{myProof}

\begin{myexample}
\drillandskill
Solve the following quadratic equations, reducing radicals as much as possible.
\end{myexample}

{\em Straight forward}:
\begin{multicols}{4}
	\begin{enumerate}
		\item $x^2 = 49$  \solution{$\{\pm 7\}$}
		\item $x^2 = 81$  \solution{$\{\pm 9\}$}
		\item $x^2 = 121$ \solution{$\{\pm 11\}$}
		\item $x^2 = 144$ \solution{$\{\pm 12\}$}
	\end{enumerate}
\end{multicols}

{\em Slightly harder}:
\begin{multicols}{2}
	\begin{enumerate}
		\item $2x^2 = 34$ \solution{$\{\pm\sqrt{17}\}$}
		\item $9x^2 = 25$ \solution{$\left\{\pm\frac{5}{4}\right\}$}
		\item $7y^2 = 49$ \solution{$\{\pm\sqrt{7}\}$}
		\item $8w^2 = 36$ \solution{$\left\{\pm\frac{3\sqrt{2}}{2}\right\}$}
	\end{enumerate}
\end{multicols}

{\em More involved}:
\begin{multicols}{2}
	\begin{enumerate}
		\item $(3x-5)^2 = 45$\solution{$\left\{\frac{5}{3}\pm\sqrt{5}\right\}$}
		\item $(4x+1)^2 = 100$\solution{$\left\{\frac{9}{4},-\frac{11}{4}\right\}$}
		\item $(1-2w)^2 = 80$\solution{$\left\{\frac{1}{2}\pm\sqrt{5}\right\}$}
		\item $(1+3y)^2 = 75$\solution{$\left\{-\frac{1}{3}\pm \frac{5}{3}\sqrt{3}\right\}$}
	\end{enumerate}
\end{multicols}


\subsection{Pythagoras' Theorem}
Consider the following right angled triangle
\begin{center}
	\begin{tikzpicture}[inner sep=0mm,scale=3]
		\node (a) at (0,0) {};
		\node (b) at (1,0) {};
		\node (c) at (0,1) {};
		\draw rectangle (.15,.15);
		\draw (a)--(b) node [midway,below left] {$c$};
		\draw (b)--(c) node [midway,above right] {$a$};
		\draw (c)--(a) node [midway,left] {$b$};
	\end{tikzpicture}
\end{center}
Then the following relationship is true
\[
	a^2=b^2+c^2	
\]
The side $a$ opposite the right angle (illustrated with a square on the diagram) is called the 
{\em hypotenuse}. 

There are many proofs of this theorem, some using more advanced mathematics than others. It is a useful
theorem in many applications, and will usually require us to use the square root property.

\begin{myexample}
A baseball diamond is a square with 90 foot sides. Find the distance from home plate to second base.
\end{myexample}
\begin{myProof}
	A typical baseball diamond is shown below
	\begin{center}
		\begin{tikzpicture}
			\node (mycenter) at (0,0) {};
			\node[draw,rectangle,fill=red] (b1) [left = of mycenter] {};
			\node[draw,rectangle,fill=red] (b2) [below = of mycenter] {};
			\node[draw,rectangle,fill=red] (b3) [right = of mycenter] {};
			\node[draw,rectangle,fill=red] (b4) [above= of mycenter] {};
			%\draw (b1)--(b2)--(b3)--(b4)--(b1);
			\node [node distance=0cm] (b11) [left=of b1] {Home plate};
			\node [node distance=0cm] (b21) [below=of b2] {First base};
			\node [node distance=0cm] (b31) [right=of b3] {Second base};
			\node [node distance=0cm] (b41) [above=of b4] {Third base};
			\draw[<->] (b1)--(b2) node [midway,below left] {90ft};
			\draw[<->] (b2)--(b3) node [midway,below right] {90ft};
			\draw[<->] (b3)--(b4) node [midway,above right] {90ft};
			\draw[<->] (b4)--(b1) node [midway,above left] {90ft};
			\draw[dotted] (b1) -- (b3) node [midway] {$x$};
		\end{tikzpicture}
	\end{center}
		
	Let $x$ represent the distance, in feet, from home plate to 2nd base. Using Pythagoras' Theorem
	\[
		90^2+90^2 = x^2
	\]
	and therefore
	\[
		16200 = x^2
	\]
	Using the square root property, we see that
	\[
		x = \pm \sqrt{16200}
	\]
	Since $x$ represents distance, we reject the negative \gls{solution} and conclude that
	\begin{align*}
		x & =  \sqrt{16200} \\
		  & \approx  127.3  
	\end{align*} 
	The distance from home plate to 2nd base is approximately 127.3 feet.
\end{myProof} 

\begin{myexample}
An object is dropped from a building that is 100ft high, and its height $h$ (in feet) at time $t$ (in seconds)
is given by
\[
	h = -16t^2+100
\] 
At what time does the object hit the ground?
\end{myexample}
\begin{myProof}
	When the object hits the ground, the height of the object is 0. This means that we have to solve the equation
	\[
		-16t^2+100 = 0
	\]
	To solve this equation for $t$ we subtract 100 from both sides to give
	\[
		-16t^2=-100
	\]
	On dividing by $-16$, we achieve
	\[
		t^2 = \frac{25}{4}
	\]
	from which we can use the square root property to give
	\[
		t = \pm \frac{5}{2}
	\]
	When dealing with application problems such as this, we have to ask ourselves if there are two possible
	values for $t$. In this case, $t$ represents time, and since we can not have negative time we reject the negative
	solution. 
		
	Therefore, the object hits the ground when $t=\frac{5}{2}$ seconds. The check is left as an exercise for the reader.
\end{myProof}
