%+*** mainfile.tex
% !arara: pdflatex: { files: [ mainfile.tex ] }
% arara: makechapters: { files:[mainfile], items: [module1], makeChapGlossaries: yes}
% !arara: indent: { overwrite: on, trace: yes, localSettings: on}
\chapter{Systems of equations}
\minitoc
\section{Review of Graphing lines}\label{sec:reviewgraphlines}
Our starting points in Math 65 requires us to have a good foundation of graphing of lines. It may
have been a while since you have studied them, and this section is to remind you about them.

There are 3 main pieces of terminology that we will need for this section:
\begin{enumerate}
	\item \gls{slope} describes the way a line slants ($m$)
	\item the $x$-\gls{intercept} is where the line crosses the $x$ - axis.
	\item the $y$-intercept is where the line crosses the $y$ - axis; $(0, b)$
\end{enumerate}
There are various ways to write the \gls{equation} of a line
\begin{itemize}[leftmargin=6cm,font=\bfseries]
	\item[Standard form	]	$Ax+By=C$, where $A$, $B$, and $C$ are integers	
	\item[Slope-intercept form]	\parbox[t]{7cm}{$y=mx+b$, where $m$ is slope and $b$ is the $y$ coordinate of the $y$ intercept. $(0,b)$ is the $y$ intercept} 	
	\item[Point-slope form]	\parbox[t]{7cm}{$y-y_1=m(x-x_1)$ where $m$ is the slope and $(x_1,y_1)$ is a \gls{point} on the line}
\end{itemize}
The most common form is the slope-intercept form, $y = mx + b$. In this case the given equation
is in terms of $y$; the slope $m$, and the $y$ -intercept $(0,b)$ can readily be determined.
There are many ways to find the slope of a line which is needed to find any equation of the line.
The various ways to find slope are given and illustrated below.

Given two point on a line, we can find the slope by using the slope formula
\[
	m = \frac{y_2-y_1}{x_2-x_1}
\]
where the two point $(x_1,y_1)$ and $(x_2,y_2)$ are given.
	
\begin{myexample}
Find the slope of the line that connects $(-5, 4)$ and $(-1, 0)$
\end{myexample}
\begin{myProof}
	We begin with the formula for the slope
	\[
		m = \frac{y_2-y_1}{x_2-x_1}
	\]
	and we substitute $(x_1,y_1) = (-5,4)$, and $(x_2,y_2) = (-1,0)$ and obtain
	\begin{align*}
		m & =	 	\frac{0-4}{-1-(-5)} \\
		  & =		\frac{-4}{4}         \\
		  & =		-1                   
	\end{align*}
	Notice how we have lined up the $=$ symbols, and have worked vertically- this is the way
	that mathematics should be presented, and it is required for you to adopt this.
\end{myProof}
				
\begin{myexample}
Find the slope for the line represented by the equation $2x+3y=6$
\end{myexample}
\begin{myProof}
	In this example we use the equation of the line to find the slope. We first need to rearrange the equation for $y$ by moving the $2x$ to the other side by subtracting $2x$ from 
	both sides to give
	\[
		3y = 6- 2x
	\]
	Now we divide all terms by $3$ which gives
	\[
		\frac{3y}{3} = \frac{6-2x}{3}
	\]
	which can be simplified to
	\begin{align*}
		y = 2-\frac{2x}{3} ~\textrm{or}~ y = -\frac{2x}{3}+2 ~\textrm{or}~ y=-\frac{2}{3}x+2 
	\end{align*}
	Of the three forms given here, the most preferable is the last one, that is
	\[
		y = -\frac{2}{3}x +2
	\]
	The reason this is most preferable is that it separates the {\em \gls{coefficient}} of $x$ from the \gls{variable}. 
	This allows us to read off that this line has a slope of $-\frac{2}{3}$.
\end{myProof}
									
\begin{myexample}
Another way to find the slope of the line $2x+3y=6$
\end{myexample}
\begin{myProof}
	When you find three point that are on the line and are getting ready to graph it, you
	then have a couple more choices on how to find the slope. 
																							
	We begin by finding 3 point on the line by choosing values for $x$ and then finding the corresponding
	value of $y$
	\begin{tightcenter}
		\begin{tabular}{SS}
			\toprule
            {$x$} & {$y$}  \\
			\midrule
			0 & 2  \\
			3 & 0  \\
			6 & -2 \\
			\bottomrule
		\end{tabular}
	\end{tightcenter}
	Remember that slope is also known as $\frac{rise}{run}$. The change in values in the second
	column is the rise and the corresponding change of values in the first column is the run. Therefore, 
	the slope of this line is $\dd\frac{-2}{3}$ or $\dd-\frac{2}{3}$, which is what we found in the last example.
\end{myProof}
									
\begin{myexample}\label{ex:firstgraph}
And another way to find the slope of the line $2x+3y=6$.
\end{myexample}
\begin{myProof}
	Find $3$ point on the line by choosing values for $x$ and then finding $y$
	\begin{tightcenter}
		\begin{tabular}{SS}
			\toprule
            {$x$} & {$y$}  \\
			\midrule
			0 & 2  \\
			3 & 0  \\
			6 & -2 \\
			\bottomrule
		\end{tabular}
	\end{tightcenter}
	Now graph these point, and connect the point to graph your line as shown in \cref{fig:firstline}.
	\begin{figure}
		\begin{subfigure}{.5\textwidth}
			\centering
			\begin{tikzpicture}
				\begin{axis}[%
						framed,
						xmin=-5,xmax=7,
						ymin=-5,ymax=5,
						grid=major,
						xtick={-4,...,6},
						ytick={-4,...,4},
						width=.9\textwidth,
					]
					\addplot+[soldot] coordinates{	(0,2)	(3,0) (6,-2) };
				\end{axis}
			\end{tikzpicture}
			\caption{Graph the point}
		\end{subfigure}%
		\begin{subfigure}{.5\textwidth}
			\centering
			\begin{tikzpicture}
				\begin{axis}[
						framed,
						xmin=-5,xmax=7,
						ymin=-5,ymax=5,
						grid=major,
						xtick={-4,...,6},
						ytick={-4,...,4},
						width=.9\textwidth,
					]
					\addplot expression[domain=-4.5:7]{2-2/3*x};
					\legend{$3y+2x=6$};
					\addplot[cmhplot,soldot] 	coordinates{	(0,2) (3,0) (6,-2) };
					\draw (axis cs: -3,4) -- node[anchor=east]{Rise} (axis cs: -3,2);
					\draw (axis cs: -3,2) -- node[anchor=north]{Run} (axis cs: 0,2);
				\end{axis}
			\end{tikzpicture}
			\caption{Draw a line through them}
		\end{subfigure}%
		\caption{The line defined in \cref{ex:firstgraph}}
		\label{fig:firstline}
	\end{figure}
																							
	Next make any right triangle using any two point on the line, and remember that the rise is
	how many units you go up (or down, in which case the rise is negative) from the first point
	to the second point, and the run is how many units you go the right from the first point.
																							
	In this case, we see that we go
	\begin{tightcenter}
		down 2
	\end{tightcenter}
	which means that the rise is $-2$. 
																										
	We also see that we go
	\begin{tightcenter}right 3\end{tightcenter}
	which means that the run is 3. 
																														
	Therefore
	\begin{align*}
		m & =		\frac{\mathrm{rise}}{\mathrm{run}} \\
		  & =		-\frac{2}{3}                       
	\end{align*}
	So in the last three examples, we obtained the same slope using three different methods. We also reviewed
	how to graph a line. 
																														
	Please use your text to review more about graphing.
\end{myProof}
																			
\begin{myexample}\label{ex:graphbyintercepts}
Graph 
\[
	y = 3x - 5
\]
by finding the $x$ and $y$ intercept.
\end{myexample}
\begin{myProof}
	We need to find the horizontal and vertical intercept. 
	\begin{itemize}[leftmargin=4cm]
		\item[$x$-intercept]
		We find the $x$ intercept by setting $y=0$ and solving for $x$
		\[
			0 = 3 x - 5
		\]
		which gives 
		\[
			x = \frac{5}{3}
		\]
		The horizontal intercept is therefore $\left(\frac{5}{3},0\right)$.
		\item[$y$-intercept]
		We find the $y$ intercept by setting $x=0$ and solving for $y$
		\[
			y = -5
		\]
		The vertical intercept is therefore $(0,-5)$. 
	\end{itemize} 
	We plot the intercept  and line in \cref{fig:graphintercepts}.
	\begin{figure}
		\begin{subfigure}{.5\textwidth}
			\centering
			\begin{tikzpicture}
				\begin{axis}[
						framed,
						xmin=-10,xmax=10,
						ymin=-10,ymax=10,
						xtick={-8,-6,...,8},
						minor xtick={-9,-7,...,9},
						ytick={-8,-6,...,8},
						minor ytick={-9,-7,...,9},
						grid=both,
						width=0.9\textwidth,
					]
					\addplot+[soldot]coordinates{ (1.66,0)  (0,-5)  };
				\end{axis}
			\end{tikzpicture}
			\caption{Graph the intercept}
		\end{subfigure}
		\begin{subfigure}{.5\textwidth}
			\centering
			\begin{tikzpicture}
				\begin{axis}[
						framed,
						xmin=-10,xmax=10,
						ymin=-10,ymax=10,
						xtick={-8,-6,...,8},
						minor xtick={-9,-7,...,9},
						ytick={-8,-6,...,8},
						minor ytick={-9,-7,...,9},
						grid=both,
						legend pos=north west,
						width=0.9\textwidth,
					]
					\addplot expression[domain=-1.5:5]{3*x-5};
					\legend{$y=3x-5$};
					\addplot[cmhplot,soldot] coordinates{ (1.66,0)  (0,-5)  };
				\end{axis}
			\end{tikzpicture}
			\caption{Draw a line through them}
		\end{subfigure}
		\caption{The line defined in \cref{ex:graphbyintercepts}.}
		\label{fig:graphintercepts}
	\end{figure}
																																								
	\mbox{}
\end{myProof} 
																													
\section{Solving systems of equations by graphing}
\textref{5.1}{290}%
A \gls{system} of \gls{linear} equation (in two variable) consists of two equation in two unknowns, $x$ and
$y$ say. Examples of such system are
\begin{equation*}
	\begin{cases}
		3x+2y = 1 &   \\
		4x-y=2    &   
	\end{cases}
\end{equation*}
Our goal in this first part of the course is to \gls{solve} such a system by finding unique values for $x$ and
$y$ that satisfy both equation. There are a number of ways that we can do this; the method that we
describe in the section is by graphing.
																												
Solving system of equation using graphing is a good way to represent the idea visually. For
those of you who are visual learners, this will indeed convince you of the various possibilities, and
add to the understanding of why the answers appear as they do. In general the way to do this is as
follows:
\begin{enumerate}
	\item Graph the first equation using your preferred method (see \cref{sec:reviewgraphlines}).
	\item Graph the second equation using your preferred method (see \cref{sec:reviewgraphlines}).
	\item See if the lines are:
	\begin{enumerate}
		\item intersecting in one point $–$ this is the \gls{solution}
		\item parallel - there is no solution because they are inconsistent
		\item the same line - All Real Numbers is the answer and they are dependent lines.
	\end{enumerate}	 
	\item If there is a solution of a point, label the coordinates on the graph. Then write the
	coordinates as your final answer.
\end{enumerate}
We demonstrate the techniques with examples.
																												
\begin{myexample}
Solve the system by graphing
\begin{equation}\label{ex:firstsystem}
	\begin{cases}
		x+y = 2 &   \\
		x-y = 4 &   
	\end{cases}
\end{equation}
\end{myexample}
\begin{myProof}
	We can choose {\em any} values for $x$ that we wish, we will just be finding different point on the line
	\begin{tightcenter}
		\begin{minipage}{4cm}
			{\color{violet}
				\begin{tabular}{SS}
					\toprule
					{$x$} & {$y = 2-x$} \\	
					\midrule
					0 & 2       \\
					2 & 0       \\
					4 & -2      \\
					\bottomrule
				\end{tabular}
			}
		\end{minipage}
		\begin{minipage}{4cm}
			{\color{red}
				\begin{tabular}{SS}
					\toprule
					{$x$} & {$y = x-4$} \\	
					\midrule
					0 &  -4     \\
					2 &  -2     \\
					5 &  1      \\
					\bottomrule
				\end{tabular}
			}
		\end{minipage}
	\end{tightcenter}
	We now graph these point on a graph, and connect the point. We label each line
	with its equation. Note, they can labelled in their original form if you prefer $x+y=2$ and
	$x-y=4$, or in slope intercept form as demonstrated.
																																									
	\begin{figure}[!h]
		\begin{subfigure}{.5\textwidth}
			\centering
			\begin{tikzpicture}
				\begin{axis}[
						framed,
						xmin=-5,xmax=6,
						ymin=-5,ymax=5,
						grid=major,
						xtick={-4,...,5},
						ytick={-4,...,4},
						width=0.9\textwidth,
					]
					\addplot+[soldot] coordinates{	(0,2)  (2,0)  (4,-2) };
					\addplot+[soldot] coordinates{	(0,-4)  (2,-2)  (5,1)  (3,-1)  };
				\end{axis}
			\end{tikzpicture}
			\caption{Graph the point}
		\end{subfigure}
		\begin{subfigure}{.5\textwidth}
			\centering
			\begin{tikzpicture}
				\begin{axis}[
						framed,
						xmin=-5,xmax=6,
						ymin=-5,ymax=5,
						grid=major,
						xtick={-4,...,5},
						ytick={-4,...,4},
						width=0.9\textwidth,
					]
					\addplot expression[domain=-1:6]{x-4};
					\addplot expression[domain=-3:6]{2-x};
					\legend{$y=x-4$, $y=2-x$};
					\addplot+[soldot] coordinates{ (0,-4)  (2,-2)  (5,1)  (3,-1) };
					\addplot+[soldot] coordinates{	(0,2)  (2,0)  (4,-2) };
				\end{axis}
			\end{tikzpicture}
			\caption{Draw a line through them}
		\end{subfigure}
		\caption{The system of equation defined in \cref{ex:firstsystem}.}
	\end{figure}
	\FloatBarrier
																																									
	We can see that these lines intersect at the point $(3,-1)$. This is the {\em only}
	point that both equation have in common, and $(3,-1)$ will make both of the equation
	true; in other words $(3,-1)$ is the solution to the original system of linear equation. It
	remains for us to check that this is true.
																																									
	Once we have found our solution, we must always check it by plugging the numbers
	back into the original system.
																																									
	If $x=3$ and $y=-1$ then in the first equation
	\begin{align*}
		x+y & =	3+(-1) \\
		    & =	2,     
	\end{align*}
	which is as required. In the second equation
	\begin{align*}
		x-y & =	3-(-1) \\
		    & =	3+1    \\
		    & =	4,     
	\end{align*}
	which is again as required. 
																																									
	We must always {\em answer the question in a sentence}. The solution to the system $(3,-1)$. 
	As there is only one solution, we say that the system is {\em consistent}.
\end{myProof}
																											
\begin{myexample}
Solve the system by graphing
\begin{equation}\label{ex:secondsystem}
	\begin{cases}
		y	=	2x+1  &   \\
		y	=	-2x-3 &   
	\end{cases}
\end{equation}
\end{myexample}
\begin{myProof}
	We proceed as in the previous example by making a table of values for each of the two equation
	\begin{tightcenter}
		\begin{minipage}{4cm}
			{\color{red}
				\begin{tabular}{SS}
					\toprule
					{$x$}  & {$y = 2x+1$} \\	
					\midrule
					0  & 1        \\
					2  & 5        \\
					-2 & -3       \\
					\bottomrule
				\end{tabular}
			}
		\end{minipage}
		\begin{minipage}{4cm}
			{\color{violet}
				\begin{tabular}{SS}
					\toprule
					{$x$}  & {$y = -2x-3$} \\	
					\midrule
					0  & -3        \\
					1  & -5        \\
					-2 & 1         \\
					\bottomrule
				\end{tabular}
			}
		\end{minipage}
	\end{tightcenter}
	Note: the $y$ value that corresponds to $x=0$ is the vertical intercept of the graph. 
																																									
	We graph the point, then connect them, and label the equation. We also mark the 
	point of intersection, find the coordinates and write them as our answer.
																																									
	\begin{figure}[!h]
		\centering
		\begin{tikzpicture}
			\begin{axis}[
					framed,
					xmin=-5,xmax=5,
					ymin=-6,ymax=6,
					grid=major,
					xtick={-4,...,4},
					ytick={-5,...,5},
					legend pos=south east,
				]
				\addplot expression[domain=-3.5:2.5]{2*x+1};
				\addplot expression[domain=-4.5:1.5]{-2*x-3};
				\legend{$y=2x+1$, $y=-2x-3$};
				\addplot+[soldot] coordinates{	(-2,-3)  (0,1)  (2,5) };
				\addplot+[soldot] coordinates{	(-2,1)  (0,-3)  (1,-5)  (-1,-1)  };
			\end{axis}
		\end{tikzpicture}
		\caption{The system defined in \cref{ex:secondsystem}.}
	\end{figure}
	\FloatBarrier
																																									
	The point of intersection is $(-1,-1)$, and this is the solution to the system. As before, 
	we need to check to make sure that this point satisfies both of the original equation (an exercise for you).
																																									
	As there is only one solution, we say that the system is {\em consistent}.
\end{myProof}
																											
\begin{myexample}
Solve the following system by graphing
\begin{equation}\label{eq:thirdsystem}
	\begin{cases}
		y		=	3x-1 &   \\
		y		=	3x+2 &   
	\end{cases}
\end{equation}
{}
\end{myexample}
																											
\begin{myProof}
	We proceed using the established routine from the previous two examples. 
																																									
	We find some sample point:
	\begin{tightcenter}
		\begin{minipage}{4cm}
			{\color{violet}
				\begin{tabular}{SS}
					\toprule
					{$x$}  & {$y = 3x-1$} \\
					\midrule
					-1 & -4       \\
					0  & -1       \\
					1  & 2        \\
					\bottomrule
				\end{tabular}
			}
		\end{minipage}
		\begin{minipage}{4cm}
			{\color{red}
				\begin{tabular}{SS}
					\toprule
					{$x$}  & {$y = 3x+2$} \\	
					\midrule
					-1 & -1       \\
					0  & 2        \\
					1  & 5        \\
					\bottomrule
				\end{tabular}
			}
		\end{minipage}
	\end{tightcenter}
	We graph the point, then connect them, and label the equation. We also mark the point
	of intersection, find the coordinates and write them as our answer.
																																									
	There is no point of intersection because they are parallel lines. We should have
	noted this when we first saw the equation, as they both have the same slope $m=3$ .
																																									
	\begin{figure}[!h]
		\centering
		\begin{tikzpicture}
			\begin{axis}[
					framed,
					xmin=-5,xmax=5,
					ymin=-6,ymax=6,
					grid=major,
					xtick={-4,...,4},
					ytick={-5,...,5},
					legend pos=north west,
				]
				\addplot expression[domain=-2.6:1.3]{3*x+2};
				\addplot expression[domain=-1.6:2.1]{3*x-1};
				\legend{$y=3x+2$, $y=3x-1$};
				\addplot+[soldot] coordinates{	(-1,-1)  (0,2)  (1,5) };
				\addplot+[soldot] coordinates{	(-1,-4)  (0,-1)  (1,2)  };
			\end{axis}
		\end{tikzpicture}
		\caption{The system of equation defined in \cref{eq:thirdsystem}.}
	\end{figure}
	\FloatBarrier
																																									
	Therefore this system has {\em no solution}. As there is no solution, we say that the
	system is {\em inconsistent}.
\end{myProof}
																											
\begin{myexample}
Solve the following system by graphing
\begin{equation}\label{eq:consistentsys}
	\begin{cases}
		2x-y		=	0 &   \\
		y		=	2x   &   
	\end{cases}
\end{equation}
\end{myexample}
\begin{myProof}
	Proceeding as in the previous examples, we achieve the graph shown in \cref{fig:consistentsyst}.
																																									
	\begin{figure}[!h]
		\centering
		\begin{tikzpicture}
			\begin{axis}[
					framed,
					xmin=-5,xmax=5,
					ymin=-5,ymax=5,
					grid=major,
					xtick={-4,...,4},
					ytick={-4,...,4},
					legend pos=north west
				]
				\addplot expression[domain=-10:10]{2*x};
				\addplot expression[domain=-2.5:2.5]{2*x};
				\legend{$y=2x$, $2x-y=0$};
			\end{axis}
		\end{tikzpicture}
		\caption{The system defined in \cref{eq:consistentsys}. Notice that there is only 1 line showing,
		as both equation represent the same line.}
		\label{fig:consistentsyst}
	\end{figure}
	%\FloatBarrier
																																									
	In this case, we notice that the equation represent the same line, in which case
	any ordered pair that satisfies one of the equation will automatically satisfy the other. In
	this case the system has infinitely many solution. In this case, the system is consistent,
	but there is not a {\em unique} solution.
\end{myProof}
																											
\section{Solving systems using substitution}
\textref{5.2}{301}%
We saw in the previous section that we can solve a system of linear equation by graphing. Another
way to solve a system of equation is by using the substitution method. The main point in this
method is to eliminate one variable from one of the equation, and substitute it into the other.
Sometimes it is easy to see why the substitution method would be the best. Other times, a different
method might be preferable (see next section).
																											
Here is an outline of the steps to solving a system of equation by substitution.
\begin{steps}
	\item See if one of the variable in one of the equation has a coefficient of 1. This is the ideal
	situation and occurs quite often. If this does not occur, you still can continue with this
	method.
	\item Solve one of the equation for either variable- choose whichever seems like the most
	simple.
	\item Now you substitute this \gls{expression} into the OTHER equation.
	\item Solve the equation for the variable.
	\item Now substitute this number in either original equation and find the value for the other
	variable.
	\item Now that you have the values for both of the variable, write the answer as an ordered pair.
	\item Check to make sure that the ordered pair makes both of the original equation true.
\end{steps}
We demonstrate this technique with a series of examples.
																											
\begin{myexample}
Solve this system by the substitution method:
\begin{align*}
	x+y & =	2 \\
	x-y & =4  
\end{align*}
\end{myexample}
\begin{myProof}
	\begin{itemize}
		\item Note that the coefficients of all the variable are 1, which means that rearranging for
		either variable $x$ or $y$ will be straightforward.
		\item   We choose the second equation to solve for $x$ as this will leave no negative signs. (We
		could just as well use the first equation if we choose.) We add $y$ to both sides to obtain
		\begin{equation*}
			x= y +4
		\end{equation*}
		\item   We substitute this expression for $x$ into the first equation to obtain
		\begin{equation*}
			(y+4 )+ y=2 ,
		\end{equation*}
		which we \gls{simplify} to
		\begin{equation*}
			2 y+4=2 ,
		\end{equation*}
		and therefore, subtracting 4 from both sides gives
		\begin{equation*}
			2 y=- 2 ,
		\end{equation*}
		which on dividing by 2 gives
		\begin{equation*}
			y =- 1 .
		\end{equation*}
		\item   Now that we have a value for y we substitute this into the expression we have for $x$,
		which gives
		\begin{align*}
			x & = -1 +4 \\
			  & = 3     
		\end{align*}
		\item According to our work, the answer is $(3, -1)$. But we need to check that this is true by
		substituting it into the original equation (exercise).
	\end{itemize}
\end{myProof}
																											
\begin{myexample}
Solve this system by the substitution method:
\begin{align*}
	x+2y  & =	7  \\
	4x-2y & =	-2 
\end{align*}
\end{myexample}
\begin{myProof}
	\begin{itemize}
		\item   Note that in the first equation the coefficient of $x$ is 1; this means that rearranging for $x$ from
		this equation will be neater. If we had chosen to rearrange for $y$ in the first equation, then
		we would need to divide by 2, which would leave fractions in the expressions- this is not as
		neat as using our first observation.
		\item   We solve the first equation for $x$ by subtracting $2 y$ from both sides
		\begin{equation*}
			x = 7-2y
		\end{equation*}
		\item   We now substitute this into the second equation which gives
		\begin{align*}
			4(7-2y)-2y & =		-2  \\
			28-8y-2y   & =		-2  \\
			28-10y     & =		-2  \\
			30         & =		10y 
		\end{align*}
		and therefore
		\begin{equation*}
			y=3
		\end{equation*}
		\item   We obtain $x$ by substituting this value of $y$ back into our original expression for $x$
		\begin{align*}
			x & =		7-2(3) \\
			  & =		1      
		\end{align*}
		\item According to our work, the answer is $(1,3)$ . We need to check that this is true by
		substituting it into the original equation (exercise).
	\end{itemize}
\end{myProof}
																											
\section{Solving systems by addition}
\textref{5.3}{309}%
Remember that any time you add opposites (also known as additive inverses) you get 0. This will make it possible
to solve system of equation using a third method called the {\em addition method} (or the elimination method depending on the
text and/or instructor). Our goal in this method is to eliminate one of the variable by using addition. This may 
be achieved by using the equation in their given form, or by multiplying one (or both) equation(s) by a number to make
the elimination occur. This will be made clearer in the examples in this section.
																											
Here is an outline of the steps for using the addition method:
\begin{steps}
	\item Write the equation underneath each other in standard form, and line up like terms and the '=' symbols
	\item See if one of the variable will `add out' as the equation by seeing if either of the variable have
	coefficients that are opposites (for example $2x$ and $-2x$)
	\item If not, you will need to multiply one or both equation by a coefficient so that one variable term will `add out'
	\item Add the two equation together. (In some cases you might have to multiply one of the equation by -1 to get
	things to `add out')
	\item Solve for the remaining variable
	\item Substitute this value in either equation
	\item Solve for the other variable
	\item Check your result by plugging the numbers back into the original system
	\item Write your answer as an ordered pair.
\end{steps} 
																											
We demonstrate this technique with a series of examples.
																											
\begin{myexample}
Solve the system by the addition method. If there is no solution or an number of solution, state so
\[
	\begin{cases}
		x+y  =6  &   \\
		x-y  =-2 &   
	\end{cases}
\]
{}
\end{myexample}
\begin{myProof}
	\begin{steps}
		\item We begin by writing the equation underneath each other
		\begin{alignat*}{3}
			x+y & {}={} & 6  \\
			x-y & {}={} & -2 
		\end{alignat*}
		\item If we add them together, will either variable add out? {\color{red} Yes, so we need to do \#3}
		\item When we add them together we get
		\[
			2x = 4
		\]
		\item Solve
		\[
			x = 2
		\]
		\item Substitute this value in one equation
		\[
			2+y=6
		\]		
		\item Solve for the other variable
		\[
			y = 4
		\]
		\item Check in both equation
		\begin{align*}
			x-y & \stackrel{?}{=} -2 & x+y & \stackrel{?}{=}	 6 \\
			2-4 & \stackrel{?}{=} -2 & 2+4 & \stackrel{?}{=}	 6 \\
			-2  & \stackrel{?}{=} -2 & 6   & \stackrel{?}{=}	 6 
		\end{align*} 
		Both of which are true.
		\item Write the answer as an ordered pair: $(2,4)$.
	\end{steps} 
	Naturally, since you have already taken Math 60, you know the problems are not always going to be this clear.
	Practicing these types of problems is a good way to get the method firm in your mind.
\end{myProof} 
																																						
\begin{myexample}
Find the solution to the system
\[
	\begin{cases}
		2x+y  =1 &   \\
		3x+y  =0 &   
	\end{cases}
\]
\end{myexample}
\begin{myProof}
	\begin{steps}
		\item Write the equation underneath each other
		\begin{align*}
			3x+y & =0 \\
			2x+y & =1 
		\end{align*} 
		\item Will one of the variable `add out' as is? {\color{red} NO}
		\item However, the $y$ terms have the same coefficients, but not opposite signs. If we
		multiply the bottom equation by -1, they will add out easily. (Note, you could multiply the top equation
		by -1 instead)
		\begin{align*}
			-2x-y & =	-1 \\
			3x+y  & =	0  
		\end{align*} 
		\item Now add the 2 equation
		\[
			1x = -1
		\]	
																																																																						
		\item Solve
		\[
			x = -1
		\]
		\item Substitute into one equation
		\[
			3(-1)+y=0
		\]
		\item Solve for $y$
		\[
			y=3
		\]
		\item Check (in both equation)
		\begin{align*}
			3x+y=0  & \stackrel{?}{=} 0 & 2x+y    & \stackrel{?}{=}	 1 \\
			3(-1)+3 & \stackrel{?}{=} 0 & 2(-1)+3 & \stackrel{?}{=}	 1 \\
			-3+3    & \stackrel{?}{=} 0 & -2+3    & \stackrel{?}{=}	 1 \\
			0       & =  0              & 1       & =			 1             
		\end{align*} 
		\item Finally we write our answer as an ordered pair: $(-1,3)$.
	\end{steps} 
\end{myProof}
																																																	
\begin{myexample}
Solve the following system by using the addition method
\[
	\begin{cases}
		2x-y  =-7 &   \\
		3x+2y =0  &   
	\end{cases}
\]
\end{myexample}
\begin{myProof}
	\begin{steps}
		\item Write the equation underneath each other
		\begin{align*}
			2x-y  & =-7 \\
			3x+2y & =0  
		\end{align*} 
		\item No variable will add out as the equation stand.
		\item Since the $y$ terms already have opposite signs, we will work
		with them first. We multiply the top equation by 2.
		\[
			2(2x-y)=2(-7)
		\]
		and therefore
		\[
			4x-2y=-14
		\]
		\item We now add the equation together
		\begin{align*}
			4x-2y & =-14 \\
			3x+2y & =0   
		\end{align*}
		which gives
		\[
			7x = -14
		\]
		\item Solve:
		\[
			x = -2
		\]
		\item Substitute into the other equation
		\[
			3(-2)+2y=0
		\]
		\item and now solve for $y$
		\begin{align*}
			-6+2y & = 0 \\
			2y    & = 6 \\
			y     & = 3 
		\end{align*} 
		{}
		\item We now check the solution in the other equation
		\[
			2(-2)-3 = -7
		\]
		which is true.
		\item We conclude by saying that the solution to the system is $(-2,3)$.
	\end{steps} 
\end{myProof} 
																																																														
\begin{myexample}
Solve the following system by using the addition method
\[
\begin{cases}
		5x+3y  =27 &\\
		7x-2y  =13 &
      \end{cases}
\]
{}
\end{myexample}
\begin{myProof}
	\begin{steps}
		\item Write equation underneath each other
		\begin{align*}
			5x+3y & =27 \\
			7x-2y & =13 
		\end{align*} 
		\item Nothing adds out.
		\item Since the $y$ terms have opposite signs, we will 
		try to get them to add out. We multiply the top equation by $2$
		and multiply the second equation by $3$
		\begin{align*}
			2(5x+3y) & =2(27) \\
			3(7x-2y) & =3(13) 
		\end{align*} 
		which gives
		\begin{align*}
			10x+6y & = 54 \\ 
			21x-6y & = 39 
		\end{align*}
		\item Adding the equation will eliminate the $y$ term
		\[
			31x = 93
		\]
		\item And we can now solve for $x$
		\[
			x = 3
		\]
		\item Substitute into the other equation
		\[
			5(3)+3y = 27
		\]	
		\item And now solve for $y$
		\[
			y = 4
		\]
		\item Check in the first equation
		\[
			5(3)+3(4)\stackrel{?}{=} 27
		\]
		which is a true statement.
		\item We conclude by stating that the solution to the system is $(3,4)$.
	\end{steps} 
\end{myProof} 
																																																																										
\begin{myexample}
Consider the following miscellaneous examples. In each case, decide which method
you would use to solve the system.
\end{myexample}
\begin{tabular}{p{3cm}cp{8cm}}
	\toprule
	Example & Choice                         & Why                                                                      \\ 
	\midrule
	$
	\begin{cases}
	x+y =6 &	\\
	2x+y =8 & 
	\end{cases}
	$
	        & Substitution {\em or }Addition & Because it is easy to solve for $x$ in the 1st equation                  
	{\em or}
	Because it is easy to multiply the first equation by -1 \\ 
	$
	\begin{cases}
	3x-7y =13 &\\
	6x+5y =7  & 
	\end{cases}
	$
	        & Addition                       & Because there is no obvious way to use substitution                      \\ 
	$
	\begin{cases}
	x-7y  =13 & \\
	6x+5y =7 &   
	\end{cases}
	$
	        & Substitution                   & Because the first equation can easily be solved for one of the variable \\
	\bottomrule
\end{tabular}
																																																																										
We have discussed three approaches to solving a system of linear equation in two
variable:
\begin{enumerate}
	\item graphing
	\item substitution
	\item addition
\end{enumerate} 
From this point on you will be able to choose which method works best depending
on your equation, unless you are specifically instructed to use a certain method.
